\section{تعاریف کلی}

\subsection{بارالکتریکی}
بار الکتریکی یک خاصیت ماده است که باعث می‌شود هنگامی که جسمی باردار در مجاورت جسم باردار دیگری قرار می‌گیرد به آن نیرو وارد شود. بار الکتریکی می‌تواند مثبت یا منفی باشد؛ که این مثبت یا منفی بودن را میزان الکترون‌های موجود در هسته جسم در مقایسه با پروتون‌های آن تعیین می کند.
در شرایط عادی تعداد الکترونها و پروتون‌ها با هم برابرند. اما اگر تعداد الکترون‌های جسم از تعداد پروتون‌های آن بیشتر باشد، جسم دارای بار الکتریکی منفی است و در صورتی که تعداد الکترون‌ها از تعداد پروتون‌ها کمتر باشد، بار الکتریکی جسم مثبت است. واحد بار الکتریکی کولن(C) است.

\noindent
\textbf{توجه:}

\begin{itemize}
\item
در این جزوه برای نمایش بار الکتریکی مثبت از حرف 
Q
و برای نمایش بار الکتریکی متغیر با زمان از حرف 
q
استفاده می‌شود.
\item
در بعضی ترجمه‌ها واحد بار الکتریکی کولمب ترجمه شده است.
\end{itemize}

\subsection{جریان الکتریکی}
از حرکت دسته جمعی الکترون‌ها جریان الکتریکی به وجود می‌آید. به توضیح علمی‌تر تعداد بارهایی که در واحد زمان از یک سطح مشخص عبور می‌کنند
\textbf{جریان الکتریکی}
گفته می‌شود. واحد حریان الکتریکی 
\textbf{آمپر}
است و جریان الکتریکی از رابطه‌های زیر محاسبه می‌شود:

\begin{gather}
I = \frac{Q}{t} \Rightarrow \text{\rl{محاسبه جریان ثابت}}\\
i(t) = \frac{dq}{qt} \Rightarrow \text{\rl{محاسبه جریان متغیر با زمان}}
\end{gather}

\subsubsection*{جهت جریان}
بصورت قراردادی در هر المان الکتریکی از طرف قطب مثبت به طرف قطب منفی یعنی خلاف جهت حرکت الکترون‌ها.
