\chapter{تعاریف کلی}

\section{بارالکتریکی}
بار الکتریکی یک خاصیت ماده است که باعث می‌شود هنگامی که جسمی باردار در مجاورت جسم باردار دیگری قرار می‌گیرد به آن نیرو وارد شود. بار الکتریکی می‌تواند مثبت یا منفی باشد؛ که این مثبت یا منفی بودن را میزان الکترون‌های موجود در هسته جسم در مقایسه با پروتون‌های آن تعیین می‌کند.
در شرایط عادی تعداد الکترونها و پروتون‌ها با هم برابرند. اما اگر تعداد الکترون‌های جسم از تعداد پروتون‌های آن بیشتر باشد، جسم دارای بار الکتریکی منفی است و در صورتی که تعداد الکترون‌ها از تعداد پروتون‌ها کمتر باشد، بار الکتریکی جسم مثبت است. واحد بار الکتریکی کولن(C) است.

\vspace{0.5cm}
\noindent
\textbf{توجه:}

\begin{itemize}
\item
در این جزوه برای نمایش بار الکتریکی ثابت از حرف 
Q
و برای نمایش بار الکتریکی متغیر با زمان از حرف 
q
استفاده می‌شود.
\item
در بعضی ترجمه‌ها واحد بار الکتریکی کولمب ترجمه شده است.
\end{itemize}

\section{جریان الکتریکی}
از حرکت دسته جمعی الکترون‌ها جریان الکتریکی به وجود می‌آید. به توضیح علمی‌تر تعداد بارهایی که در واحد زمان از یک سطح مشخص عبور می‌کنند
\textbf{جریان الکتریکی}
گفته می‌شود. واحد جریان الکتریکی 
\textbf{آمپر}
است و جریان الکتریکی از رابطه‌های زیر محاسبه می‌شود:

\begin{gather}
I = \frac{Q}{t} \Rightarrow \text{\rl{محاسبه جریان ثابت}}\\
i(t) = \frac{dq}{qt} \Rightarrow \text{\rl{محاسبه جریان متغیر با زمان}}
\end{gather}

\subsection*{جهت جریان}
بصورت قراردادی در هر المان الکتریکی از طرف قطب مثبت به طرف قطب منفی یعنی خلاف جهت حرکت الکترون‌ها. برای مثال در مقاومت زیر جهت جریان از سمت مثبت به منفی می‌باشد.

\begin{center}
	\begin{circuitikz}
		\draw (-2,0) to[R,i=$I$] (2,0);
		\node[xshift=-0.75cm,yshift=0.25cm] () {+};
		\node[xshift=0.75cm,yshift=0.25cm] () {-};
	\end{circuitikz}
\end{center}

\subsection*{انواع جریان الکتریکی}
جریان‌های الکتریکی دو نوع هستند:
\begin{description}
	\item[مستقیم]
	جهت و مقدار الکترون‌های عبوری نسبت به زمان ثابت می‌ماند. نمودار جریان الکتریکی مستقیم نسبت به زمان به صورت زیر است:
	\begin{center}
		\begin{pspicture}(-3,-3)(3,3)
			\psset{algebraic}
			\psaxes[ticks=none,labels=none,]{->}(0,0)(-0.5,-0.5)(2.5,2.5)
			\psplot[linecolor=red]{0}{1.5}{1}
		\end{pspicture}
	\end{center}
	
	
	\item[متناوب]
	جهت حرکت و مقدار جریان در فواصل زمانی معین تغییر می‌کند. جریان متناوب خود نیز با توجه به نوع تغییرات به انواعی از سینوسی، دندان اره‌ای، مربعی(پالسی) و ... تقسیم می‌شود. برای درک شهودی نیز می‌توان به نمودار جریان متناوب سینوسی توجه کرد.
	\begin{center}
		\begin{pspicture}(-3,-3)(3,3)
			\psset{algebraic}
			\psaxes[ticks=none,labels=none,]{->}(0,0)(-2.5,-2.5)(2.5,2.5)
			\psplot[linecolor=green]{-1.5}{1.5}{sin(3*x)}
		\end{pspicture}
	\end{center}
\end{description}

\section{اختلاف پتانسیل(ولتاژ)}
عاملی است برای حرکت الکترون‌ها از نقطه‌ای به نقطه‌ی دیگر. واحد آن 
\textbf{ولت(v)}
 است و از فرمول زیر محاسبه می‌شود:
\begin{gather}
	V= \frac{W}{Q}
\end{gather}

\section{توان}
به زبان ساده به معنای سرعت انجام کار است یعنی مقدار کاری که یک دستگاه در واحد زمان انجام می‌دهد. واحد آن 
\textbf{وات(w)}
است و برای محاسبه‌ی آن می‌توان از رابطه‌های زیر استفاده کرد:
\begin{align}
	W = P.t \Rightarrow V = \frac{P.t}{Q} \Rightarrow P = \frac{V.Q}{t} \Rightarrow \boxed{P=VI}
\end{align}

\begin{example}


یک منبع 220 ولتی، یک لامپ رشته‌ای 200 واتی را تغذیه می‌کند:

الف)جریان لامپ را بدست آورید.

ب)بار الکتریکی عبوری از مدار در زمان یک ساعت را بدست آورید.

\end{example}

\begin{solu}
	الف.
	\begin{gather*}
		V = 220v \qquad P=200w \\
		P = VI \rightarrow I = \frac{P}{V} = \frac{200}{220} = \frac{10}{11}
	\end{gather*}
	ب.
	\begin{gather*}
		I = \frac{Q}{t} \Rightarrow \frac{10}{11} = \frac{Q}{3600} \Rightarrow Q = \frac{3600\times 10}{11}
	\end{gather*}
	

\end{solu}
	
\begin{example}
	در صورتی که بار الکتریکی عبوری از یک سیم بصورت 
	$q(t)=5t^2$
	کولن باشد.جریان عبوری از این سیم در ثانیه 
	$t=2$
	چقدر است؟
\end{example}

\begin{solu}
	\begin{align*}
		I(t) = \frac{dq}{dt}=10t=20 
	\end{align*}
\end{solu}