\section*{پیش گفتار}
از ابتدای پیدایش بشر یکی از دغدغه‌های مهم انسان(و شاید مهم‌ترین آنها) یافتن معنی، هدف یا انگیزه‌ای برای زیستن روی کره خاکی بوده است. هدفی که تمامی ادیان بر روی آن تمرکز داشته‌اند، و متفکران مختلف در طول سالیان راه‌کارهای گوناگونی برای دست‌یابی به آن ارائه کرده‌اند. چیزی که اکثر ادیان و مکاتب در آن اتفاق نظر داشته‌اند نقش خود فرد در تعیین هدف زندگی است، که از طرف بسیاری از پیروان این مکاتب نادیده گرفته می‌شود. هیچ مکتبی نمی‌تواند دستورالعملی جامع برای دست‌یابی تک‌تک افراد به تعالی ارائه دهد، چرا که \textbf{هر کس در آفرینش منحصر به فرد است و باید نقشی متفاوت در این دنیا ایفا کند}، و درک این نقش میسر نخواهد بود مگر از طریق تفکر و تعقل خود فرد(حقیقتی که تمامی ادیان الهی به آن تأکید دارند).

حتماً تا به حال با افرادی مواجه شده‌اید که صرفاً به خاطر فشار اطرافیان، احساس ناتوانی یا دلایل دیگر، در زمینه‌ای تحصیل یا کار می‌کنند که به آن علاقه ندارند. این تلاش معمولاً نتیجه‌ای ندارد جز سرخوردگی و استعدادهای درونی فرد. نیت از نگارش این مقدمه هم  چیزی نیست جز واداشتن خودم و شما به لحظه‌ای تأمل. تأمل به مسیری که در پیش گرفته‌ایم و در آن گام برمیداریم. تأمل در مورد اینکه آیا این مسیر ما را به رشد و تعالی نزدیک می‌کند یا نه. اینکه آیا تاکنون سعی کرده‌ایم به فلسفه وجودی خود پی ببریم، یا ما هم جزء افرادی هستیم که ناخواسته و بدون هدف در مسیری که دیگران برای ما تعیین کرده‌اند گام برمیداریم.

امروزه به دلیل بیماری کرونا ناچاراً به شیوه‌های آموزش مجازی روی آوردیم. در این شیوه‌ی آموزشی یکی از دغدغه‌های اساسی دانشجویان، نکته‌برداری و دسترسی به جزوه اساتید محترم است.  از این رو تصمیم گرفتم که کمی از بار مشکلات دوستان بکاهم و این جزوه را تکمیل و در اختیار همکلاسی‌های عزیز قرار دهم. تمام دغدغه این‌جانب در نوشتن این جزوه، انتقال صحیح مطالب ارائه شده در کلاس درس بوده است. در این راستا فیلم ضبط شده را چندین بار مشاهده کرده تا بتوانم این مهم را به درستی انجام دهم. امیدوارم این تلاش نتیجه‌بخش بوده و گامی باشد هر چند ناچیز در راستای موفقیت شما عزیزان.

این کمترین هیچ ادعایی در امتیاز و حتی مقایسه کرده‌ی خود با تلاشی که اساتید صرف تدریس نموده‌اند ندارم؛ به کمبودها و اشتباهات فراوانی که در این جزوه خواهد بود تصریح داشته و از دیده‌ی نکته‌یاب اهل فضل و کرم امید اغماض دارم، تذکرات سودمند شما را به دیده‌ی منت می‌طلبم.

\begin{flushleft}
	محمد رستمی\\
	بهار ۱۴۰۰
\end{flushleft}