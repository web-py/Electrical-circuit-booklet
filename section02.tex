\chapter{عناصر مدار و قوانین تجربی}

\section{منبع}
وسیله‌ای که بتواند انرژی غیرالکتریکی را به انرژی الکتریکی و بالعکس تبدیل کند. انواع منبع‌ها عبارتند از:
\begin{description}
	\item[مستقل]
	جریان یا ولتاژ به ساختار داخلی خود منبع مرتبط است.
	\item[وابسته]
	جریان و ولتاژ برای خودشان نیست و وابسته به قسمت دیگری از مدار است. مانند منبع تغذیه مادربورد
\end{description}

\subsection*{ولتاژ مستقل}

منبعی که ولتاژ دو سر آن کاملاً مستقل از جریان عبوری از آن باشد، به طوری که با افزایش یا کاهش جریان، ولتاژ دو سر آن همواره ثابت بماند.

\begin{minipage}{0.3\textwidth}
	\begin{center}
		\begin{circuitikz}
			\draw (0,-1) to[american voltage source] (0,1);
		\end{circuitikz}
	\end{center}
\end{minipage}
\begin{minipage}{0.3\textwidth}
	\begin{center}
		\begin{circuitikz}
			\draw (0,-1) to[battery1] (0,1);
		\end{circuitikz}
	\end{center}
\end{minipage}
\subsection*{جریان مستقل}

منبعی است که جریان عبوری از آن، همواره مستقل از ولتاژ دو سر آن است.
\begin{center}
	\begin{circuitikz}
		\draw (0,-1) to[american current source] (0,1);
	\end{circuitikz}
\end{center}

\subsection*{ولتاژ وابسته}
منبعی که ولتاژ آن، به ولتاژ یا جریان قسمت دیگری از مدار وابسته است. دو نوع می‌باشد:
\begin{itemize}
	\item
	وابسته به ولتاژ شاخه دیگر
	\item
	وابسته به جریان شاخه دیگر
\end{itemize}

\begin{minipage}{0.5\textwidth}
		\begin{center}
				کنترل شده با ولتاژ
				
				$V = \alpha V$
				
				\begin{circuitikz}
				\draw (0,-1) to[american controlled voltage source]  (0,1);
			\end{circuitikz}
		\end{center}
\end{minipage}
\begin{minipage}{0.5\textwidth}
	\begin{center}
		کنترل شده با جریان
		
		$V = \beta I$
		
		\begin{circuitikz}
			\draw (0,-1) to[american controlled voltage source]  (0,1);
		\end{circuitikz}
	\end{center}
\end{minipage}

\subsection*{جریان وابسته}
منبعی که جریان آن به جریان یا ولتاژ قسمت دیگری از مدار وابسته است. دو نوع می‌باشد:
\begin{itemize}
	\item
	کنترل شده با ولتاژ
	\item
	کنترل شده با جریان
\end{itemize}

\begin{minipage}{0.5\textwidth}
	\begin{center}
		کنترل شده با جریان
		
		$I = \alpha I$
		
		\begin{circuitikz}
			\draw (0,-1) to[american controlled current source]  (0,1);
		\end{circuitikz}
	\end{center}
\end{minipage}
\begin{minipage}{0.5\textwidth}
	\begin{center}
		کنترل شده با ولتاژ
		
		$I = \beta V$
		
		\begin{circuitikz}
			\draw (0,-1) to[american controlled current source]  (0,1);
		\end{circuitikz}
	\end{center}
\end{minipage}


\section{مقاومت}

یک عنصر دوسر که با عبور جریان الکتریکی از آن یک اختلاف ولتاژ در دو سر آن اتفاق می‌افتد. واحد آن اهم
($\Omega$)
 می‌باشد.
 
\begin{minipage}{0.5\textwidth}
	\begin{center}
		\begin{align}
		\Rightarrow	V= RI
		\end{align}
	\end{center}
\end{minipage}
\begin{minipage}{0.3\textwidth}
	\begin{center}
		\begin{circuitikz}
			\draw (-2,0) to[R,i=$I$] (2,0);
			\node[xshift=-0.75cm,yshift=0.25cm] () {+};
			\node[xshift=0.75cm,yshift=0.25cm] () {-};
		\end{circuitikz}
	\end{center}
\end{minipage}

\begin{minipage}{0.5\textwidth}
	\begin{center}
		\begin{align}
			\Rightarrow	V= -RI
		\end{align}
	\end{center}
\end{minipage}
\begin{minipage}{0.3\textwidth}
	\begin{center}
		\begin{circuitikz}
			\draw (-2,0) to[R,i<=$I$] (2,0);
			\node[xshift=-0.75cm,yshift=0.25cm] () {+};
			\node[xshift=0.75cm,yshift=0.25cm] () {-};
		\end{circuitikz}
	\end{center}
\end{minipage}

\subsection*{رسانایی الکتریکی}

\begin{align}
	G = \frac{1}{R} = \frac{I}{V}
\end{align}

\textbf{توجه:}

\begin{itemize}
	\item
	در مدارهای الکتریکی داغ شدن به معنای وجود مقاومت است.
	\item
	مقاومت الکتریکی یک عنصر مصرف کننده است. یعنی جریان الکتریکی را به صورت گرما به محیط می‌دهد.
	\item
	منبع هم تولید کننده و هم مصرف کننده است. مصرف کنندگی به دلیل مقاومت درونی است.
\end{itemize}

\subsection*{توان تلف شده در مقاومت}

\begin{align}
	P = RI^2 = (RI)I = VI = V(\frac{V}{R}) = \frac{V^2}{R}
\end{align}

\section{قوانین مداری ولتاژ و جریان}

\begin{description}
	\item[گره:]
	به محل اتصال دو یا چند عنصر به یکدیگر در یک مدار گفته می‌شود.
	
	\begin{remark}
		همیشه بین دو گره مدار حداقل یک المان وجود دارد.
	\end{remark}
	\item[حلقه:]
	مسیری از یک مدار را حلقه گویند، در صورتی که اگر از گره‌ای دلخواه از روی این مسیر شروع به حرکت کنیم از عناصر عبور کنیم بدون اینکه از هیچ یک از گره‌های میان راه، بیش از یک بار بگذریم و دوباره به گره آغازین برگردیم.
	
	\begin{figure}[!h]
		\centering
		\begin{circuitikz}
			
			\draw (-2,-2) to[generic,-*] (-2,2);
			\node[xshift=-2.2cm,yshift=0.75cm]{+};
			\node[xshift=-2.2cm,yshift=-0.75cm]{-};
			\node[xshift=-2.5cm,yshift=0cm]{$V_1$};
			
			\draw (2,-2) to[generic,*-*] (2,2);
			\node[xshift=2.2cm,yshift=0.75cm]{-};
			\node[xshift=2.2cm,yshift=-0.75cm]{+};
			\node[xshift=2.5cm,yshift=0cm]{$V_3$};
			
			\draw (6,-2) to[generic , -*] (6,2);
			\node[xshift=6.2cm,yshift=0.75cm]{-};
			\node[xshift=6.2cm,yshift=-0.75cm]{+};
			\node[xshift=6.5cm,yshift=0cm]{$V_5$};
			
			\draw (-2,2) to[generic] (2,2);
			\node[xshift=0.75cm,yshift=2.2cm]{+};
			\node[xshift=-0.75cm,yshift=2.2cm]{-};
			\node[xshift=0cm,yshift=2.5cm]{$V_2$};
			
			\draw (-2,-2) -- (2,-2);
			
			
			\draw (2,2) to[generic] (6,2);
			\node[xshift=4.75cm,yshift=2.2cm]{-};
			\node[xshift=3.25cm,yshift=2.2cm]{+};
			\node[xshift=4cm,yshift=2.5cm]{$V_4$};
			
			\draw (2,-2) -- (6,-2);
			
			\draw[<-,red] (1,0.5) arc  (30:330:1);
			\draw[<-,green] (5,0.5) arc  (30:330:1);
		\end{circuitikz}
		\caption{گره و حلقه‌ها در مدار}
		\label{loop}
	\end{figure}
	
	
\end{description}

\textbf{کنجکاوی:}
در مدار بالا سه حلقه داریم، دوتا از حلقه‌ها مشخص شده‌اند. حلقه سوم را پیدا کنید.

\subsection{قانون مداری جریان(KCL)}
جمع جبری همه جریان‌ها در گره برابر صفر است. به عبارت دیگر جمع جریان‌های وارد شده به هر گره با جمع جریان‌های خارج شده از آن گره برابرند.
\subsection{قانون مداری ولتاژ(KVL)}
جمع جبری همه‌ی ولتاژها حول یک حلقه، برابر صفر است. برای به کاربر بردن قانون ولتاژها ابتدا جهتی قراردادی به صورت دلخواه در حلقه تعیین می کنیم. ولتاژ عناصری که جهت قراردادی آنها با جهت قراردادی حلقه یکی است را با علامت مثبت و بقیه را با علامت منفی در نظر می‌گیریم. برای مثال قانون ولتاژها برای حلقه‌های موجود در شکل 
\ref{loop}
 عبارتند از:

\begin{gather*}
	\begin{cases}
		-V_2 - V_3 - V_1 = 0 \\
		-V_2-V_4-V_5-V_1 = 0 \\
		-V_4 -V_5 + V_3 = 0
	\end{cases}
\end{gather*}


\section{تحلیل مدار تک حلقه‌ای}

\textbf{اهداف:}
\begin{enumerate}
	\item
	محاسبه جریان یا ولتاژ مقاومت
	\item
	محاسبه توان جذب شده یا تلف شده توسط هر عنصر
\end{enumerate}

\textbf{فرض اولیه:}
	 مقادیر مقاومت‌ها و منبع معلوم است.

\vspace{1cm}
\noindent
مراحل انجام تحلیل عبارتند از:
\begin{description}
	\item[مرحله اول]
	انتخاب یک جهت برای جریان مجهول
	\item[مرحله دوم] 
	گذاشتن علامت ولتاژ برای هر کدام از مقاومت‌ها.
	\begin{remark}
برای سمتی که جریان وارد می‌شود علامت مثبت در نظر گرفته می‌شود.
	\end{remark}

	\item[مرحله سوم] 
	استفاده از قانون 
	\lr{KVL}
\end{description}

\begin{example}
	در مدار شکل زیر ولتاژ روی هر مقاومت و توان جذب شده توسط هر عنصر را بدست آورید.
	
	\begin{center}
		\begin{circuitikz}
			\draw (0,1) to[battery1,l=$12V$] (0,-1);
			\draw (3,1) to[battery1,l=$3V$] (6,1);
			\draw (0,1) to[R,l=$3\Omega$,i>=$I$] (3,1);
			\draw (6,-1) to[R,l=$1.5\Omega$] (6,1);
			
		
			\draw (0,-1) -- (6,-1);
		\end{circuitikz}
	\end{center}
	
\end{example}

\begin{solu}
	\begin{description}
		\item[گام اول] انتخاب جهت جریان مجهول
	
		\begin{center}
		\begin{circuitikz}
			\draw (0,1) to[battery1,l=$12V$] (0,-1);
			\draw (3,1) to[battery1,l=$3V$] (6,1);
			\draw (0,1) to[R,l=$3\Omega$,i>=$I$] (3,1);
			\draw (6,-1) to[R,l=$1.5\Omega$] (6,1);
			
			
			\draw (0,-1) -- (6,-1);
			\draw[<-,color=red] (3.5,0.25) arc (30:330:0.5);
		\end{circuitikz}
	\end{center}
	
	\item[گام دوم] 
	تعیین علامت ولتاژ هر یک از مقاومت‌ها
		\begin{center}
		\begin{circuitikz}
			\draw (0,1) to[battery1,l=$12V$] (0,-1);
			\draw (3,1) to[battery1,l=$3V$] (6,1);
			\draw (0,1) to[R,l=$3\Omega$,i>=$I$] (3,1) node[xshift=-2.25cm,yshift=0.25cm,color=red]{+}
			node[xshift=-0.75cm,yshift=0.25cm,color=red]{-};
			\draw (6,-1) to[R,l=$1.5\Omega$] (6,1) node[xshift=0.25cm,yshift=-0.25cm,color=red]{+}
			node[xshift=0.25cm,yshift=-1.5cm,color=red]{-};
			
			
			\draw (0,-1) -- (6,-1);
			\draw[<-,color=red] (3.5,0.25) arc (30:330:0.5);
		\end{circuitikz}
	\end{center}
	
	\item[گام سوم] 
	استفاده از قانون KVL
	\begin{gather*}
		+3I + 3 + 1.5I -12 = 0 \\
		4.5I - 9 = 0 \Rightarrow I = 2A \\
		V_3 = 3 \times 2 = 6v \\
		V_{1.5} = 1.5 \times 2 = 3v \\
		P_{12v} = -VI = 12 \times 2 = -24w \\
		P_{3\Omega} = VI = 6 \times 2 = 12w \\
		P_{3v} = VI = 3 \times 2 = 6w \\
		P_{1.5\Omega} = VI = 3 \times 2 = 6w \\
	\end{gather*}
	
\end{description}
\end{solu}

\begin{remark}
	هر المانی که جذب کننده(مصرف کننده) باشد، توان جذب شده علامت مثبت دارد و تولید کننده‌ها توان جذب شده علامت منفی دارد.
\end{remark}

\begin{minipage}{0.5\textwidth}
	\begin{center}
		\begin{circuitikz}
			\draw (0,1) to[battery1,i=$I$] (0,-1);
			\node[xshift=2cm] (node name) {\text{\rl{منبع تولیدکننده}}};
		\end{circuitikz}
	\end{center}
\end{minipage}
\begin{minipage}{0.5\textwidth}
	\begin{center}
		\begin{circuitikz}
			\draw (0,1) to[battery1,i>=$I$] (0,-1);
			\node[xshift=2cm] (node name) {\text{\rl{منبع مصرف کننده}}};
		\end{circuitikz}
	\end{center}
\end{minipage}


برای ساده کردن مدارها به مدارهای تک حلقه‌ای می‌توان مقاومت‌ها را با یکدیگر ترکیب کرد. این ترکیب مقاومت‌ها به سه صورت انجام می‌شود:
\begin{enumerate}
	\item
	آرایش سری(متوالی)
	\item
	آرایش موازی
	\item
	آرایش ستاره-مثلث
\end{enumerate}




\subsection*{آرایش سری(متوالی)}

\begin{center}
	\begin{circuitikz}
		\draw (0,1) to[american voltage source,l_=$V_s$] (0,-1);
		\draw (0,1) to[R,l=$R_1$] (2,1);
		\draw (2,1) to[R,l=$R_2$] (4,1);
		\draw (5,1) to[R,l=$R_n$] (7,1);
		\node[xshift=4.5cm,yshift=1cm] (node name) {$\cdots$};
		
		\draw (0,-1) -- (7,-1) -- (7,1);
		\draw[->] (1,-1.5) -- (1,-0.5) -- (1.5,-0.5) node[xshift=-1cm,yshift=-1cm]{$R_{eq}$};
	\end{circuitikz}
\end{center}
\begin{gather*}
	R_{eq}=R_1 + R_2 + \cdots + R_n \rightarrow \text{\rl{مقاومتی که منبع می‌بیند.}}
\end{gather*}




\subsection*{آرایش موازی}
یعنی ولتاژ دو سر مقاومت یکسان است به عبارت دیگر دو سر آنها گره‌های یکسانی است.

\begin{center}
	\begin{circuitikz}
		\draw (0,1) to[american voltage source,l_=$V_s$] (0,-1);
		\draw (2,1) to[R,l_=$R_1$] (2,-1);
		\draw (4,1) to[R,l_=$R_2$] (4,-1);
		\draw (6,1) to[R,l=$R_n$] (6,-1);
		\node[xshift=5cm,yshift=0cm] (node name) {$\cdots$};
		\node[xshift=5cm,yshift=1cm] (node name) {$\cdots$};
		\node[xshift=5cm,yshift=-1cm] (node name) {$\cdots$};
		
		\draw (0,1) -- (4.5,1);
		\draw (0,-1) -- (4.5,-1);
		
		\draw (5.5,1) -- (6,1);
		\draw (5.5,-1) -- (6,-1);
	\end{circuitikz}
\end{center}
\begin{gather*}
	\frac{1}{R_{eq}} = \frac{1}{R_1} + \frac{1}{R_1} + \cdots + \frac{1}{R_n}
\end{gather*}



\subsection*{آرایش ستاره-مثلث}
این نوع مدار را با علامت 
$\Delta - Y $
نیز نشان می‌دهد. در این مدار به اسامی دقت کنید چرا که در تبدیل مهم است.

\begin{minipage}{0.5\textwidth}
	\begin{center}
		\begin{circuitikz}
			\draw (0,0) to[R,l=$R_c$, *-*] (3,0) node[yshift=0.35cm]{b};
			\draw (3,0) to[R,l=$R_a$, *-*] (1.5,-3) node[yshift=-0.35cm]{c};
			\draw (1.5,-3) to[R,l=$R_b$, *-*] (0,0) node[yshift=0.35cm]{a};
		\end{circuitikz}
	\end{center}
\end{minipage}
\begin{minipage}{0.5\textwidth}
		\begin{center}
		\begin{circuitikz}
			\draw (1.5,-1.5) to[R,l=$R_1$, *-*] (0,0) node[yshift=0.35cm]{a};
			\draw (1.5,-1.5) to[R,l_=$R_2$, *-*] (3,0) node[yshift=0.35cm]{b};
			\draw (1.5,-1.5) to[R,l=$R_3$, *-*] (1.5,-3) node[yshift=-0.35cm]{a};
		\end{circuitikz}
	\end{center}
\end{minipage}


\begin{gather*}
	\text{\rl{تبدیل ستاره به مثلث}} \rightarrow
	\begin{cases}
		R_a = \frac{R_1R_2 + R_1R_3 + R_2R_3}{R_1}\vspace{0.5cm} \\
		R_b = \frac{R_1R_2 + R_1R_3 + R_2R_3}{R_2} \vspace{0.5cm}\\
		R_c = \frac{R_1R_2 + R_1R_3 + R_2R_3}{R_3} \\
	\end{cases}
\end{gather*}

\begin{gather*}
	\text{\rl{تبدیل مثلث به ستاره}} \rightarrow
	\begin{cases}
		R_1 = \frac{R_bR_c}{R_a + R_b + R_c} \vspace{0.5cm}\\ 
		R_2 = \frac{R_aR_c}{R_a + R_b + R_c} \vspace{0.5cm} \\ 
		R_3 = \frac{R_aR_b}{R_a + R_b + R_c}\\ 
	\end{cases}
\end{gather*}

\begin{example}
	در مدار شکل زیر جریان 
	$I_s$
	را بدست آورید.
	
	\begin{center}
		\begin{circuitikz}
			\draw (0,1.5) to[american voltage source,l_=$12v$] (0,-1.5);
			\draw (3,-1.5) to[R,l=$18\Omega$] (3,1.5);
			\draw (6,-1.5) to[R,l=$6\Omega$] (6,1.5);
			\draw (0,1.5) to[R,l=$4\Omega$,i=$I_s$] (3,1.5);
			\draw (3,1.5) to[R,l=$3\Omega$] (6,1.5);
			
			\draw (0,-1.5) -- (6,-1.5);
		\end{circuitikz}
	\end{center}
\end{example}
\begin{solu}
	\ \\
	\begin{gather*}
		R_{3,6} = R_3 + R_6 = 3+6=9\Omega
	\end{gather*}
\begin{minipage}{0.6\textwidth}
		\begin{center}
		\begin{circuitikz}
			\draw (0,1.5) to[american voltage source,l=$12v$] (0,-1.5);
			\draw (3,-1.5) to[R,l=$18\Omega$] (3,1.5);
			\draw (6,-1.5) to[R,l=$9\Omega$,color=red] (6,1.5);
			\draw (0,1.5) to[R,l=$4\Omega$,i=$I_s$] (3,1.5);
			\draw (3,1.5) -- (6,1.5);
			
			\draw (0,-1.5) -- (6,-1.5);
		\end{circuitikz}
	\end{center}
\end{minipage}
\begin{minipage}{0.6\textwidth}
	\begin{center}
		\begin{circuitikz}
			\draw (0,1.5) to[american voltage source,l_=$12v$] (0,-1.5);
			\draw (3,-1.5) to[R,l=$18\Omega$] (3,1.5);
			\draw (6,-1.5) to[R,l=$6\Omega$,color=red] (6,1.5);
			\draw (0,1.5) to[R,l=$4\Omega$,i=$I_s$] (3,1.5);
			\draw (3,1.5) to[R,l=$3\Omega$,color=red] (6,1.5);
			
			\node [xshift=7cm,yshift=0cm]{$\Rightarrow$};
			
			\draw (0,-1.5) -- (6,-1.5);
		\end{circuitikz}
	\end{center}
\end{minipage}

\begin{gather*}
	\frac{1}{R_{eq}} = \frac{1}{9} + \frac{1}{18} \Rightarrow R_{eq} = 6\Omega
\end{gather*}

\begin{minipage}{0.4\textwidth}
	\begin{center}
		\begin{circuitikz}
			\draw (0,1.5) to[american voltage source,l=$12v$] (0,-1.5);
			\draw (3,-1.5) to[R,l=$6\Omega$,color=red] (3,1.5);
		
			\draw (0,1.5) to[R,l=$4\Omega$,i=$I_s$] (3,1.5);
			
			
			\draw (0,-1.5) -- (3,-1.5);
		\end{circuitikz}
	\end{center}
\end{minipage}
\begin{minipage}{0.6\textwidth}
	\begin{center}
	\begin{circuitikz}
		\draw (0,1.5) to[american voltage source,l=$12v$] (0,-1.5);
		\draw (3,-1.5) to[R,l=$18\Omega$,color=red] (3,1.5);
		\draw (6,-1.5) to[R,l=$9\Omega$,color=red] (6,1.5);
		\draw (0,1.5) to[R,l=$4\Omega$,i=$I_s$] (3,1.5);
		\draw (3,1.5) -- (6,1.5);
		
		\node [xshift=7cm,yshift=0cm]{$\Rightarrow$};
		
		\draw (0,-1.5) -- (6,-1.5);
	\end{circuitikz}
\end{center}
\end{minipage}

\begin{gather*}
	R_{eq} = 4 + 6 = 10\Omega
\end{gather*}


\begin{minipage}{0.4\textwidth}
	\begin{center}
		\begin{circuitikz}
			\draw (0,1.5) to[american voltage source,l_=$12v$] (0,-1.5);
			\draw (3,-1.5) to[R,l_=$10\Omega$,color=red] (3,1.5);
			
			\draw (0,1.5) to[short,i=$I_s$] (3,1.5);
			
			\draw[->,color=blue] (1,-0.25) arc (210:-30:0.5);
			\draw (0,-1.5) -- (3,-1.5);
		\end{circuitikz}
	\end{center}
\end{minipage}
\begin{minipage}{0.4\textwidth}
	\begin{center}
		\begin{circuitikz}
			\draw (0,1.5) to[american voltage source,l=$12v$] (0,-1.5);
			\draw (3,-1.5) to[R,l=$6\Omega$,color=red] (3,1.5);
			
			\draw (0,1.5) to[R,l=$4\Omega$,i=$I_s$,color=red] (3,1.5);
			
			\node [xshift=4.5cm,yshift=0cm]{$\Rightarrow$};
			\draw (0,-1.5) -- (3,-1.5);
		\end{circuitikz}
	\end{center}
\end{minipage}

\begin{gather*}
	-12 + 10 I_s = 0 \rightarrow 10 I_s = 12 \rightarrow I_s = \frac{12}{10} = 1.2 A
\end{gather*}
\end{solu}

\vspace{1cm}
\begin{example}
	در مدار شکل زیر جریان I را بدست آورید.
	
	\begin{center}
		\begin{circuitikz}
			\draw (0,2) to[battery1,l=$40v$] (0,-2);
			\draw (0,2) to[R,l=$54\Omega$] (5,2);
			\draw (5,1.5) to[R,l=$30\Omega$] (7,0);
			\draw (5,1.5) to[R,l_=$20\Omega$] (3,0);
			\draw (3,0) to[R,l_=$50\Omega$] (5,-1.5);
			\draw (7,0) to[R,l=$15\Omega$] (5,-1.5);
			\draw (3,0) to[R,l=$50\Omega$] (7,0);
			
			\draw (0,-2) -- (5,-2) -- (5,-1.5);
			\draw (5,2) -- (5,1.5);
		\end{circuitikz}
	\end{center}
\end{example}


\begin{solu}
	قسمت مشخص شده در مدار زیر به فرم ستاره‌ای تبدیل می‌شود.
	
	\begin{minipage}{0.5\textwidth}
		\begin{gather*}
			\begin{cases}
				R_1 = \frac{20 \times 30 }{20+30+50} = 6\Omega \vspace{0.5cm} \\
				R_2 = \frac{30 \times 50 }{20+30+50} = 15\Omega \vspace{0.5cm} \\
				R_3 = \frac{50 \times 20 }{20+30+50} = 10 \Omega  \vspace{0.5cm} \\
			\end{cases}
		\end{gather*}
	\end{minipage}
	\begin{minipage}{0.5\textwidth}
		\begin{center}
			\begin{circuitikz}
				\draw (0,2) to[battery1,l=$40v$] (0,-2);
				\draw (0,2) to[R,l=$54\Omega$] (5,2);
				\draw (5,1.5) to[R,l=$30\Omega$] (7,0);
				\draw (5,1.5) to[R,l_=$20\Omega$] (3,0);
				\draw (3,0) to[R,l_=$50\Omega$] (5,-1.5);
				\draw (7,0) to[R,l=$15\Omega$] (5,-1.5);
				\draw (3,0) to[R,l=$50\Omega$] (7,0);
				
				\node[rectangle,draw,dashed,color=red,minimum height=2cm,minimum width=4cm,xshift=5cm,yshift=0.75cm]  {};
				\node[yshift=-3cm,xshift=3cm] (node name) {$\Downarrow$};
				
				\draw (0,-2) -- (5,-2) -- (5,-1.5);
				\draw (5,2) -- (5,1.5);
			\end{circuitikz}
		\end{center}
	\end{minipage}



	\begin{minipage}{0.4\textwidth}
	\begin{gather*}
		\begin{cases}
			R_{3,4} =10+50 = 60\Omega \vspace{0.5cm} \\
			R_{2,5} = 15+15 = 30\Omega \\
		\end{cases}
	\end{gather*}
\end{minipage}
\begin{minipage}{0.6\textwidth}
	\begin{center}
		\begin{circuitikz}
			\draw (0,3) to[battery1,l=$40v$] (0,-3);
			\draw (0,3) to[R,l=$54\Omega$] (5,3);
			\draw (5,3) to[R,l=$R_1 \text{=} 6\Omega$] (5,1);
			\draw (4,1) to[R,l_=$R_3 \text{=}10\Omega$,color=red] (4,-1);
			\draw (6,1) to[R,l=$R_2 \text{=}15\Omega$,color=blue] (6,-1);
			\draw (4,-1) to[R,l_=$R_4 \text{=}50\Omega$,color=red] (4,-2.5);
			\draw (6,-1) to[R,l=$R_5 \text{=}15\Omega$,color=blue] (6,-2.5);
			
			\draw (4,1) -- (6,1);
			\draw (4,-2.5) -- (6,-2.5);
			\draw (0,-3) -- (5,-3) -- (5,-2.5);
		\end{circuitikz}
	\end{center}
\end{minipage}


\vspace{0.5cm}
	\begin{minipage}{0.4\textwidth}
	\begin{gather*}
			\frac{1}{R_{2,3}} = \frac{1}{30} + \frac{1}{60} \Rightarrow R_{2,3} = 20\Omega\\
	\end{gather*}
\end{minipage}
\begin{minipage}{0.6\textwidth}
	\begin{center}
		\begin{circuitikz}
			\draw (0,3) to[battery1,l=$40v$] (0,-3);
			\draw (0,3) to[R,l=$54\Omega$] (5,3);
			\draw (5,3) to[R,l=$R_1 \text{=} 6\Omega$] (5,1);
			\draw (4,1) to[R,l_=$R_3 \text{=}60\Omega$,color=red] (4,-2.5);
			\draw (6,1) to[R,l=$R_2 \text{=}30\Omega$,color=red] (6,-2.5);
			
	
			
			\draw (4,1) -- (6,1);
			\draw (4,-2.5) -- (6,-2.5);
			\draw (0,-3) -- (5,-3) -- (5,-2.5);
		\end{circuitikz}
	\end{center}
\end{minipage}

\vspace{0.5cm}
	\begin{minipage}{0.4\textwidth}
	\begin{gather*}
		 R_{1,2} = 6 + 20 = 26\Omega\\
	\end{gather*}
\end{minipage}
\begin{minipage}{0.6\textwidth}
	\begin{center}
		\begin{circuitikz}
			\draw (0,2) to[battery1,l=$40v$] (0,-2);
			\draw (0,2) to[R,l=$54\Omega$] (3,2);
			\draw (3,2) to[R,l=$R_1 \text{=} 6\Omega$,color=red] (3,0);
			\draw (3,0) to[R,l=$R_2 \text{=}20\Omega$,color=red] (3,-2);
				
			\draw (0,-2) -- (3,-2);
		\end{circuitikz}
	\end{center}
\end{minipage}


\vspace{0.5cm}
	\begin{minipage}{0.4\textwidth}
	\begin{gather*}
		R_{eq} = 54+26 = 80\Omega\\
	\end{gather*}
\end{minipage}
\begin{minipage}{0.6\textwidth}
	\begin{center}
		\begin{circuitikz}
			\draw (0,1) to[battery1,l_=$40v$] (0,-1);
			\draw (0,1) to[R,l=$54\Omega$,color=red] (2,1);
			\draw (2,1) to[R,l=$R_1 \text{=} 26\Omega$,color=red] (2,-1);
			
			\draw (0,-1) -- (2,-1);
		\end{circuitikz}
	\end{center}
\end{minipage}

\vspace{0.5cm}
	\begin{minipage}{0.4\textwidth}
	\begin{gather*}
		I = \frac{40}{80} = 0.5 A\\
	\end{gather*}
\end{minipage}
\begin{minipage}{0.6\textwidth}
	\begin{center}
		\begin{circuitikz}
			\draw (0,1) to[battery1,l_=$40v$] (0,-1);
			\draw (0,1) to[short,i=$I$] (2,1);
			\draw (2,1) to[R,l=$ 80\Omega$] (2,-1);
			
			\draw (0,-1) -- (2,-1);
		\end{circuitikz}
	\end{center}
\end{minipage}

\end{solu}


\vspace{1cm}
\subsection*{مدار تقسیم ولتاژ}

\begin{minipage}{0.5\textwidth}
	\begin{center}
		\begin{gather*}
			V_1 = \frac{R_1}{R_1 + R_2 + \cdots + R_n} \times V \vspace{0.75cm} \\
			V_2 = \frac{R_2}{R_1 + R_2 + \cdots + R_n} \times V \vspace{0.75cm} \\
			\vdots \vspace{0.5cm} \\
			V_n = \frac{R_n}{R_1 + R_2 + \cdots + R_n} \times V \vspace{0.75cm} \\
		\end{gather*}
	\end{center}
\end{minipage}
\begin{minipage}{0.5\textwidth}
	\begin{center}
		\begin{circuitikz}
			\draw (0,3) to[american voltage source,v<=$V$] (0,-3);
			
			\draw (2,3) to[R,l=$R_1$] (2,1);
			\draw (2,1) to[R,l=$R_2$] (2,0);
			\node[xshift=2cm,yshift=-0.5cm]() {$\vdots$};
			\draw (2,-1) to[R,l=$R_n$] (2,-3);
			
			\draw (0,3) -- (2,3);
			\draw (0,-3) -- (2,-3);
		\end{circuitikz}
	\end{center}
\end{minipage}

\subsection*{مدار تقسیم جریان}

\begin{center}
	\begin{circuitikz}
		\draw (0,-1.5) to[american current source] (0,1.5);
		\draw (2,-1.5) to[R,l=$R_1$,i<=$I_1$] (2,1.5);
		\draw (4,-1.5) to[R,l=$R_2$,i<=$I_2$] (4,1.5);
		
		\draw (0,1.5) -- (4,1.5);
		\draw (0,-1.5) -- (4,-1.5);
	\end{circuitikz}
\end{center}


\begin{gather*}
	I_1 = \frac{R_2}{R_1 + R_2} I \qquad 	I_2 = \frac{R_1}{R_1 + R_2} I 
\end{gather*}

\begin{remark}
	\begin{itemize}
	
	\item
	اگر مدار شما به صورت زیر بود می‌توان طرف راست و چپ آن را به صورت مدارهای جداگانه تحلیل کرد.
	
	\begin{center}
		\begin{circuitikz}
			\draw (0,1.5) to[american voltage source, l_= $10v$] (0,-1.5);
			\draw (2,-1.5) to[american voltage source,l=$15v$] (2,1.5);
			\draw (4,-1.5) to[R,l_=$2\Omega$] (4,1.5);
			\draw (0,1.5) to[R,l=$5\Omega$] (2,1.5);
			\draw (2,1.5) to[R,l=$3\Omega$] (4,1.5);
		
			\draw (0,-1.5) -- (4,-1.5);
		\end{circuitikz}
	\end{center}

	\item
	اگر خواسته‌ی سؤال محاسبه‌ی توان مصرفی باشد، تولیدکننده‌ها علامت منفی و مصرف‌کننده‌ها علامت مثبت می‌گیرند.
	\item
	اگر خواسته‌ی سؤال محاسبه توان تولیدی باشد، تولید‌کننده‌ها علامت مثبت و مصرف کننده‌ها در مدار علامت منفی می‌گیرند.
\end{itemize}
\end{remark}


\newpage
\vspace{1cm}
\section*{مسائل}
\rule{\textwidth}{0.1cm}
\vspace{1cm}

\begin{enumerate}
	\item 
	در شکل 
	\ref{fig_sec02_01}
	 ولتاژ v را پیدا کنید.
	\begin{figure}[!h]
		\centering
		\includegraphics{images/sec02/01}
			\caption{}
		\label{fig_sec02_01}
	\end{figure}

		\item 
	در شکل 
	\ref{fig_sec02_02}
	 ولتاژ دو سر مقاومت 
	$3\Omega$
	 را پیدا کنید.
	\begin{figure}[!h]
		\centering
		\includegraphics{images/sec02/02}
			\caption{}
		\label{fig_sec02_02}
	\end{figure}

		\item 
	در شکل
	\ref{fig_sec02_03}
	توانی که منبع جریان به مدار تحویل می‌دهد به دست آورید.
	\begin{figure}[!h]
		\centering
		\includegraphics{images/sec02/03}
			\caption{}
		\label{fig_sec02_03}
	\end{figure}

		\item 
	در شکل 
	\ref{fig_sec02_04}
مقدار v را پیدا کنید.
	\begin{figure}[!h]
		\centering
		\includegraphics{images/sec02/04}
			\caption{}
		\label{fig_sec02_04}
	\end{figure}

	\item 
	در شکل 
	\ref{fig_sec02_05}
	مقاومت معادل 
	$R_{eq}$
	را پیدا کنید.
	\begin{figure}[!h]
		\centering
		\includegraphics{images/sec02/05}
			\caption{}
		\label{fig_sec02_05}
	\end{figure}

		\item 
	در شکل 
	\ref{fig_sec02_06}
	مقاومت معادل 
	$R_{eq}$
	را پیدا کنید.
	\begin{figure}[!h]
		\centering
		\includegraphics{images/sec02/06}
		\caption{}
		\label{fig_sec02_06}
	\end{figure}

		\item 

	مقاومت معادل شکل 
	\ref{fig_sec02_07}
	را پیدا کنید.
	\begin{figure}[!h]
		\centering
		\includegraphics{images/sec02/07}
		\caption{}
		\label{fig_sec02_07}
	\end{figure}

	\item 

با ترکیب مقاومت‌ها و تقسیم ولتاژ،
$V_{ab}$
مدار شکل
\ref{fig_sec02_08}
را پیدا کنید.
	\begin{figure}[!h]
		\centering
		\includegraphics{images/sec02/08}
		\caption{}
		\label{fig_sec02_08}
	\end{figure}

		\item 
	
	با ترکیب مقاومت‌ها و تقسیم جریان،
	$I_x$
	مدار شکل
	\ref{fig_sec02_09}
	را پیدا کنید.
	\begin{figure}[!h]
		\centering
		\includegraphics{images/sec02/09}
		\caption{}
		\label{fig_sec02_09}
	\end{figure}


		\item 
	در مدار شکل
	\ref{fig_sec02_10}
	$i=5A$
	مقدار v را پیدا کنید.
	\begin{figure}[!h]
		\centering
		\includegraphics{images/sec02/10}
		\caption{}
		\label{fig_sec02_10}
	\end{figure}

		\item 
	با کاربرد مستقیم قوانین کیرشهف جریان i را در مدار شکل
	\ref{fig_sec02_11}
	بدست آورید.
	\begin{figure}[!h]
		\centering
		\includegraphics{images/sec02/11}
		\caption{}
		\label{fig_sec02_11}
	\end{figure}

		\item 
	با کاربرد مستقیم قوانین کیرشهف جریان
	$I_x$
	را در مدار شکل
	\ref{fig_sec02_12}
	بدست آورید.
	\begin{figure}[!h]
		\centering
		\includegraphics{images/sec02/12}
		\caption{}
		\label{fig_sec02_12}
	\end{figure}

		\item 
	در مدار شکل
	\ref{fig_sec02_13}
	مقاومت معادل
	$R_{eq}$
	را پیدا کنید.
	\begin{figure}[!h]
		\centering
		\includegraphics{images/sec02/13}
		\caption{}
		\label{fig_sec02_13}
	\end{figure}

		\item 
	با استفاده از تبدیل مثلث به ستاره جریان i را در شکل
	\ref{fig_sec02_14}
	بدست آورید.
	\begin{figure}[!h]
		\centering
		\includegraphics{images/sec02/14}
		\caption{}
		\label{fig_sec02_14}
	\end{figure}

		\item 
	با استفاده از تبدیل مثلث-ستاره جریان i را در شکل
	\ref{fig_sec02_15}
	بدست آورید.
	\begin{figure}[!h]
		\centering
		\includegraphics{images/sec02/15}
		\caption{}
		\label{fig_sec02_15}
	\end{figure}

\end{enumerate}
