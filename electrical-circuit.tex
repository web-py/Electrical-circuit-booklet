\documentclass[12pt]{article}
\usepackage[labelsep=space]{caption}
\usepackage{fancyhdr}
\fancyfoot[C]{\thepage}
\fancyhead[L]{مدارهای منطقی و الکتریکی}
\fancyhead[R]{\leftmark}
\usepackage[hidelinks]{hyperref}
\linespread{1.5}
%\usepackage{background}
%\backgroundsetup{contents=webpy.ir, opacity=0.45, scale=3}
\usepackage{amsmath}
\usepackage{amsthm}
\usepackage{pstricks,pst-plot}
\usepackage{tikz}
\usepackage{circuitikz}
\usepackage{xepersian}
\usepackage{MnSymbol}
\settextfont{XB VNiloofar}
\setdigitfont{XB VNiloofar}

%%%%%%%%%%%%%%%%%%%%%%%%%%%%%%%%%%%%
% دستورهایی برای نحوه ظاهر شدن تعاریف، قضایا، مثال‌ها و... در متن
% باید توجه داشت که متن محیط‌های ادعایی مثل قضیه، لم، گزاره، نتیجه و ملاحظه باید به صورت ایتالیک باشد؛ ولی
% محیط مثال و تعریف، باید به صورت ایستاده باشند. همچنین شماره تمام این محیط‌ها برای پیدا کردن راحت‌تر آن‌ها
% باید از یکدیگر پیروی کنند.
\theoremstyle{definition}
\newtheorem{definition}{تعریف}[section]
\theoremstyle{plain}
\newtheorem{theorem}[definition]{قضیه}
\newtheorem{lemma}[definition]{کنجکاوی}
\newtheorem{proposition}[definition]{گزاره}
\newtheorem{corollary}[definition]{نتیجه}
%\newtheorem{remark}[definition]{توجه}
\newtheorem*{remark}{توجه}
\theoremstyle{definition}
\newtheorem{example}[definition]{مثال}
\newtheorem*{solu}{حل}
\renewcommand\proofname{{\rm \textbf{برهان}}}



\begin{document}
	\pagestyle{empty}
	%\NoBgThispage
	\includegraphics[height=12cm]{images/besm}
	\title{جزوه مدار الکتریکی و الکترونیکی }
	\author{همکلاسی}
	\maketitle
	\vspace*{1.5cm}
	\begin{abstract}
		
	\end{abstract}
	\section*{پیش گفتار}
از ابتدای پیدایش بشر یکی از دغدغه‌های مهم انسان(و شاید مهم‌ترین آنها) یافتن معنی، هدف یا انگیزه‌ای برای زیستن روی کره خاکی بوده است. هدفی که تمامی ادیان بر روی آن تمرکز داشته‌اند، و متفکران مختلف در طول سالیان راه‌کارهای گوناگونی برای دست‌یابی به آن ارائه کرده‌اند. چیزی که اکثر ادیان و مکاتب در آن اتفاق نظر داشته‌اند نقش خود فرد در تعیین هدف زندگی است، که از طرف بسیاری از پیروان این مکاتب نادیده گرفته می‌شود. هیچ مکتبی نمی‌تواند دستورالعملی جامع برای دست‌یابی تک‌تک افراد به تعالی ارائه دهد، چرا که \textbf{هر کس در آفرینش منحصر به فرد است و باید نقشی متفاوت در این دنیا ایفا کند}، و درک این نقش میسر نخواهد بود مگر از طریق تفکر و تعقل خود فرد(حقیقتی که تمامی ادیان الهی به آن تأکید دارند).

حتماً تا به حال با افرادی مواجه شده‌اید که صرفاً به خاطر فشار اطرافیان، احساس ناتوانی یا دلایل دیگر، در زمینه‌ای تحصیل یا کار می‌کنند که به آن علاقه ندارند. این تلاش معمولاً نتیجه‌ای ندارد جز سرخوردگی و استعدادهای درونی فرد. نیت از نگارش این مقدمه هم  چیزی نیست جز واداشتن خودم و شما به لحظه‌ای تأمل. تأمل به مسیری که در پیش گرفته‌ایم و در آن گام برمیداریم. تأمل در مورد اینکه آیا این مسیر ما را به رشد و تعالی نزدیک می‌کند یا نه. اینکه آیا تاکنون سعی کرده‌ایم به فلسفه وجودی خود پی ببریم، یا ما هم جزء افرادی هستیم که ناخواسته و بدون هدف در مسیری که دیگران برای ما تعیین کرده‌اند گام برمیداریم.

امروزه به دلیل بیماری کرونا ناچاراً به شیوه‌های آموزش مجازی روی آوردیم. در این شیوه‌ی آموزشی یکی از دغدغه‌های اساسی دانشجویان، نکته‌برداری و دسترسی به جزوه اساتید محترم است.  از این رو تصمیم گرفتم که کمی از بار مشکلات دوستان بکاهم و این جزوه را تکمیل و در اختیار همکلاسی‌های عزیز قرار دهم. تمام دغدغه این‌جانب در نوشتن این جزوه، انتقال صحیح مطالب ارائه شده در کلاس درس بوده است. در این راستا فیلم ضبط شده را چندین بار مشاهده کرده تا بتوانم این مهم را به درستی انجام دهم. امیدوارم این تلاش نتیجه‌بخش بوده و گامی باشد هر چند ناچیز در راستای موفقیت شما عزیزان.

این کمترین هیچ ادعایی در امتیاز و حتی مقایسه کرده‌ی خود با تلاشی که اساتید صرف تدریس نموده‌اند ندارم؛ به کمبودها و اشتباهات فراوانی که در این جزوه خواهد بود تصریح داشته و از دیده‌ی نکته‌یاب اهل فضل و کرم امید اغماض دارم، تذکرات سودمند شما را به دیده‌ی منت می‌طلبم.

\begin{flushleft}
	محمد رستمی\\
	بهار ۱۴۰۰
\end{flushleft}
	\newpage
	\tableofcontents
	\newpage
	\pagestyle{fancy}	
	\include{intro}
	\section{تعاریف کلی}

\subsection{بارالکتریکی}
بار الکتریکی یک خاصیت ماده است که باعث می‌شود هنگامی که جسمی باردار در مجاورت جسم باردار دیگری قرار می‌گیرد به آن نیرو وارد شود. بار الکتریکی می‌تواند مثبت یا منفی باشد؛ که این مثبت یا منفی بودن را میزان الکترون‌های موجود در هسته جسم در مقایسه با پروتون‌های آن تعیین می کند.
در شرایط عادی تعداد الکترونها و پروتون‌ها با هم برابرند. اما اگر تعداد الکترون‌های جسم از تعداد پروتون‌های آن بیشتر باشد، جسم دارای بار الکتریکی منفی است و در صورتی که تعداد الکترون‌ها از تعداد پروتون‌ها کمتر باشد، بار الکتریکی جسم مثبت است. واحد بار الکتریکی کولن(C) است.

\noindent
\textbf{توجه:}

\begin{itemize}
\item
در این جزوه برای نمایش بار الکتریکی مثبت از حرف 
Q
و برای نمایش بار الکتریکی متغیر با زمان از حرف 
q
استفاده می‌شود.
\item
در بعضی ترجمه‌ها واحد بار الکتریکی کولمب ترجمه شده است.
\end{itemize}

\subsection{جریان الکتریکی}
از حرکت دسته جمعی الکترون‌ها جریان الکتریکی به وجود می‌آید. به توضیح علمی‌تر تعداد بارهایی که در واحد زمان از یک سطح مشخص عبور می‌کنند
\textbf{جریان الکتریکی}
گفته می‌شود. واحد حریان الکتریکی 
\textbf{آمپر}
است و جریان الکتریکی از رابطه‌های زیر محاسبه می‌شود:

\begin{gather}
I = \frac{Q}{t} \Rightarrow \text{\rl{محاسبه جریان ثابت}}\\
i(t) = \frac{dq}{qt} \Rightarrow \text{\rl{محاسبه جریان متغیر با زمان}}
\end{gather}

\subsubsection*{جهت جریان}
بصورت قراردادی در هر المان الکتریکی از طرف قطب مثبت به طرف قطب منفی یعنی خلاف جهت حرکت الکترون‌ها.

	\section{عناصر مدار و قوانین تجربی}

\subsection{منبع}
وسیله‌ای که بتواند انرژی غیرالکتریکی را به انرژی الکتریکی و بالعکس تبدیل کند. انواع منبع‌ها عبارتند از:
\begin{description}
	\item[مستقل]
	جریان یا ولتاژ به ساختار داخلی خود منبع مرتبط است.
	\item[وابسته]
	جریان و ولتاژ برای خودشان نیست و وابسته به قسمت دیگری از مدار است. مانند منبع تغذیه مادربورد
\end{description}

\subsubsection*{ولتاژ مستقل}

منبعی که ولتاژ دو سر آن کاملاً مستقل از جریان عبوری از آن باشد، به طوری که با افزایش یا کاهش جریان، ولتاژ دو سر آن همواره ثابت بماند.

\begin{minipage}{0.3\textwidth}
	\begin{center}
		\begin{circuitikz}
			\draw (0,-1) to[american voltage source] (0,1);
		\end{circuitikz}
	\end{center}
\end{minipage}
\begin{minipage}{0.3\textwidth}
	\begin{center}
		\begin{circuitikz}
			\draw (0,-1) to[battery1] (0,1);
		\end{circuitikz}
	\end{center}
\end{minipage}
\subsubsection*{جریان مستقل}

منبعی است که جریان عبوری از آن، همواره مستقل از ولتاژ دو سر آن است.
\begin{center}
	\begin{circuitikz}
		\draw (0,-1) to[american current source] (0,1);
	\end{circuitikz}
\end{center}

\subsubsection*{ولتاژ وابسته}
منبعی که ولتاژ آن، به ولتاژ یا جریان قسمت دیگری از مدار وابسته است. دو نوع می‌باشد:
\begin{itemize}
	\item
	وابسته به ولتاژ شاخه دیگر
	\item
	وابسته به جریان شاخه دیگر
\end{itemize}

\begin{minipage}{0.5\textwidth}
		\begin{center}
				کنترل شده با ولتاژ
				
				$V = \alpha V$
				
				\begin{circuitikz}
				\draw (0,-1) to[american controlled voltage source]  (0,1);
			\end{circuitikz}
		\end{center}
\end{minipage}
\begin{minipage}{0.5\textwidth}
	\begin{center}
		کنترل شده با جریان
		
		$V = \beta I$
		
		\begin{circuitikz}
			\draw (0,-1) to[american controlled voltage source]  (0,1);
		\end{circuitikz}
	\end{center}
\end{minipage}

\subsubsection*{جریان وابسته}
منبعی که جریان آن به جریان یا ولتاژ قسمت دیگری از مدار وابسته است. دو نوع می‌باشد:
\begin{itemize}
	\item
	کنترل شده با ولتاژ
	\item
	کنترل شده با جریان
\end{itemize}

\begin{minipage}{0.5\textwidth}
	\begin{center}
		کنترل شده با جریان
		
		$I = \alpha I$
		
		\begin{circuitikz}
			\draw (0,-1) to[american controlled current source]  (0,1);
		\end{circuitikz}
	\end{center}
\end{minipage}
\begin{minipage}{0.5\textwidth}
	\begin{center}
		کنترل شده با ولتاژ
		
		$I = \beta V$
		
		\begin{circuitikz}
			\draw (0,-1) to[american controlled current source]  (0,1);
		\end{circuitikz}
	\end{center}
\end{minipage}


\subsection{مقاومت}

یک عنصر دوسر که با عبور جریان الکتریکی از آن یک اختلاف ولتاژ در دو سر آن اتفاق می‌افتد. واحد آن اهم
($\Omega$)
 می باشد.
 
\begin{minipage}{0.5\textwidth}
	\begin{center}
		\begin{align}
		\Rightarrow	V= RI
		\end{align}
	\end{center}
\end{minipage}
\begin{minipage}{0.3\textwidth}
	\begin{center}
		\begin{circuitikz}
			\draw (-2,0) to[R,i=$I$] (2,0);
			\node[xshift=-0.75cm,yshift=0.25cm] () {+};
			\node[xshift=0.75cm,yshift=0.25cm] () {-};
		\end{circuitikz}
	\end{center}
\end{minipage}

\begin{minipage}{0.5\textwidth}
	\begin{center}
		\begin{align}
			\Rightarrow	V= -RI
		\end{align}
	\end{center}
\end{minipage}
\begin{minipage}{0.3\textwidth}
	\begin{center}
		\begin{circuitikz}
			\draw (-2,0) to[R,i<=$I$] (2,0);
			\node[xshift=-0.75cm,yshift=0.25cm] () {+};
			\node[xshift=0.75cm,yshift=0.25cm] () {-};
		\end{circuitikz}
	\end{center}
\end{minipage}

\subsubsection*{رسانایی الکتریکی}

\begin{align}
	G = \frac{1}{R} = \frac{I}{V}
\end{align}

\textbf{توجه:}

\begin{itemize}
	\item
	در مدارهای الکتریکی داغ شدن به معنای وجود مقاومت است.
	\item
	مقاومت الکتریکی یک عنصر مصرف کننده است. یعنی جریان الکتریکی را به صورت گرما به محیط می دهد.
	\item
	منبع هم تولید کننده و هم مصرف کننده است. مصرف کنندگی به دلیل مقاومت درونی است.
\end{itemize}

\subsubsection*{توان تلف شده در مقاومت}

\begin{align}
	P = RI^2 = (RI)I = VI = V(\frac{V}{R}) = \frac{V^2}{R}
\end{align}

\subsection{قوانین مداری ولتاژ و جریان}

\begin{description}
	\item[گره:]
	به محل اتصال دو یا چند عنصر به یکدیگر در یک مدار گرفته گفته می‌شود.
	\item[حلقه:]
	مسیری از یک مدار را حلقه گویند، در صورتی که اگر از گره‌ای دلخواه از روی این مسیر شروع به حرکت کنیم از عناصر عبور کنیم بدون اینکه از هیچ یک از گره‌های میان راه، بیش از یک بار بگذریم و دوباره به گره آغازین برگردیم.
	
	\begin{figure}[!h]
		\centering
		\begin{circuitikz}
			
			\draw (-2,-2) to[generic,-*] (-2,2);
			\node[xshift=-2.2cm,yshift=0.75cm]{+};
			\node[xshift=-2.2cm,yshift=-0.75cm]{-};
			\node[xshift=-2.5cm,yshift=0cm]{$V_1$};
			
			\draw (2,-2) to[generic,*-*] (2,2);
			\node[xshift=2.2cm,yshift=0.75cm]{-};
			\node[xshift=2.2cm,yshift=-0.75cm]{+};
			\node[xshift=2.5cm,yshift=0cm]{$V_3$};
			
			\draw (6,-2) to[generic , -*] (6,2);
			\node[xshift=6.2cm,yshift=0.75cm]{-};
			\node[xshift=6.2cm,yshift=-0.75cm]{+};
			\node[xshift=6.5cm,yshift=0cm]{$V_5$};
			
			\draw (-2,2) to[generic] (2,2);
			\node[xshift=0.75cm,yshift=2.2cm]{+};
			\node[xshift=-0.75cm,yshift=2.2cm]{-};
			\node[xshift=0cm,yshift=2.5cm]{$V_2$};
			
			\draw (-2,-2) -- (2,-2);
			
			
			\draw (2,2) to[generic] (6,2);
			\node[xshift=4.75cm,yshift=2.2cm]{-};
			\node[xshift=3.25cm,yshift=2.2cm]{+};
			\node[xshift=4cm,yshift=2.5cm]{$V_4$};
			
			\draw (2,-2) -- (6,-2);
			
			\draw[<-,red] (1,0.5) arc  (30:330:1);
			\draw[<-,green] (5,0.5) arc  (30:330:1);
		\end{circuitikz}
		\caption{گره و حلقه‌ها در مدار}
	\end{figure}
	
	
\end{description}

\textbf{کنجکاوی:}
در مدار بالا سه حلقه داریم، دوتا از حلقه‌ها مشخص شده‌اند. حلقه سوم را پیدا کنید.

\subsubsection{قانون مداری جریان}
جمع جبری همه جریان‌ها در گره برابر صفر است. به عبارت دیگر جمع جریان‌های وارد شده به هر گره با جمع جریان‌های خارج شده از آن گره برابرند.
\subsubsection{قانون مداری ولتاژ}
جمع جبری همه‌ی ولتاژها حول یک حلقه، برابر صفر است. برای به کاربر بردن قانون ولتاژها ابتدا جهتی قراردادی به صورت دلخواه در حلقه تعیین می کنیم. ولتاژ عناصری که جهت قراردادی آنها با جهت قراردادی حلقه یکی است را با علامت مثبت و بقیه را با علامت منفی در نظر می‌گیریم. برای مثال قانون ولتاژها برای حلقه‌های موجود در شکل فلان عبارتند از:

\begin{gather*}
	\begin{cases}
		-V_2 - V_3 - V_1 = 0 \\
		-V_2-V_4-V_5-V_1 = 0 \\
		-V_4 -V_5 + V_3 = 0
	\end{cases}
\end{gather*}





	\section{روش‌های تحلیل مدار}

\subsection{روش تحلیل گره}

هدف: یافتن ولتاژ گره‌ها

\begin{description}
	\item[گام اول] 
	مشخص کردن گره های مدار و نام گذاری آنها
	\item[گام دوم] 
	مشخص کردن یک گره به عنوان گره مبنا(ولتاژ گره مبنا صفر فرض می‌شود.)
	\item[گام سوم] 
	نوشتن KCL برای هر گره(بجز گره مبنا)
\end{description}

\begin{example}
	ولتاژ گره‌های مدار زیر را بیابید.
	\begin{center}
	\begin{circuitikz}
		\draw (0,2) to[american voltage source,l_=$4v$] (0,0);
		\draw (2,2) to[R,l=$4\Omega$] (2,0);
		\draw (4,2) to[R,l=$8\Omega$] (4,0);
		\draw (6,0) to[american current source,l_=$2A$] (6,2);
		
		\draw (0,2) to[R,l=$1\Omega$] (2,2);
		\draw (2,2) to[R,l=$1\Omega$] (4,2);
		
		
		\draw (0,0) -- (6,0);
		\draw (4,2) -- (6,2);
	\end{circuitikz}
	\end{center}
\end{example}

\begin{solu}
	
	\begin{description}
		\item[گام اول] مشخص کردن گره‌ها و نامگذاری
		
			\begin{center}
			\begin{circuitikz}
				\draw (0,2) to[american voltage source,l_=$4v$] (0,0) node[xshift=-0.25cm,yshift=2.25cm,color=red]{1};
				\draw (2,2) to[R,l=$4\Omega$,-*] (2,0) node[xshift=0.25cm,yshift=-0.25cm,color=red]{4};
				\draw (4,2) to[R,l=$8\Omega$] (4,0);
				\draw (6,0) to[american current source,l_=$2A$] (6,2);
				
				\draw (0,2) to[R,l=$1\Omega$, *-*] (2,2) node[yshift=0.25cm ,color=red]{2};
				\draw (2,2) to[R,l=$1\Omega$,*-*] (4,2) node[yshift=0.25cm ,color=red]{3};
				
				
				\draw (0,0) -- (6,0);
				\draw (4,2) -- (6,2);
			\end{circuitikz}
		\end{center}
	
	\item[گام دوم] مشخص کردن گره مبنا
	
	\begin{center}
		\begin{circuitikz}
			\draw (0,2) to[american voltage source,l_=$4v$] (0,0) node[xshift=-0.25cm,yshift=2.25cm,color=red]{1};
			\draw (2,2) to[R,l=$4\Omega$,-*] (2,0) node[xshift=0.25cm,yshift=-0.25cm,color=red]{4};
			\draw (4,2) to[R,l=$8\Omega$] (4,0);
			\draw (6,0) to[american current source,l_=$2A$] (6,2);
			
			\draw (0,2) to[R,l=$1\Omega$, *-*] (2,2) node[yshift=0.25cm ,color=red]{2};
			\draw (2,2) to[R,l=$1\Omega$,*-*] (4,2) node[yshift=0.25cm ,color=red]{3};
			
			\draw (2,0) to[ground] (2,-1);
			
			\draw[color=red] (1.5,-1) -- (2.5,-1);
			\draw[color=red] (1.75,-1.1) -- (2.25,-1.1);
			\draw[color=red] (1.9,-1.2) -- (2.1,-1.2);
			
			\draw[->] (1.9,-0.1) -- (1,-0.5) node[xshift=-0.75cm]{$V_4 = 0$};
			
			\draw (0,0) -- (6,0);
			\draw (4,2) -- (6,2);
		\end{circuitikz}
	\end{center}
		
		\item[گام سوم] 
		نوشتن KCL هر گره
		
				\begin{minipage}{0.5\textwidth}
			
			\begin{gather*}
				\Rightarrow
				\begin{cases}
					\frac{9}{4}V_2 - V_3 = 4 \vspace{1cm}\\
					-V_2 + \frac{9}{8}V_3 = 2
				\end{cases}
			\end{gather*}
		\end{minipage}
		\begin{minipage}{0.5\textwidth}
			\begin{gather*}
				\begin{cases}
					\textcircled{1} V_1 - 0 = 4 \rightarrow V_1 = 4v \vspace{1cm}\\
					\textcircled{2} \frac{V_2 - V_1}{1} + \frac{V_2-0}{4} + \frac{V_2-V_3}{1} = 0 \vspace{1cm}\\
					\textcircled{3} \frac{V_3-V_2}{1} + \frac{V_3-0}{8} -2=0
				\end{cases}
			\end{gather*}
		\end{minipage}
		
		
		
		
		
		\begin{minipage}{0.5\textwidth}
			\begin{gather*}
				V_3 = \frac{\begin{vmatrix}
						\frac{9}{4} & 4 \\
						-1 & 2\\
				\end{vmatrix}}{\begin{vmatrix}
						\frac{9}{4} & -1 \\
						-1 & \frac{9}{8} \\
				\end{vmatrix}} = 6.51v
			\end{gather*}
		\end{minipage}
		\begin{minipage}{0.5\textwidth}
			\begin{gather*}
				V_2 = \frac{\begin{vmatrix}
						4 & -1 \\
						2 & \frac{9}{8} \\
				\end{vmatrix}}{\begin{vmatrix}
						\frac{9}{4} & -1 \\
						-1 & \frac{9}{8} \\
				\end{vmatrix}} = 4.13v
			\end{gather*}
		\end{minipage}
		
	\end{description}
	
	

\end{solu}


\vspace{0.5cm}
\textbf{توجه:}
\begin{itemize}
	\item
	همیشه بین دو گره حداقل یک المان وجود دارد.
	\item
	معمولاً پایین ترین گره به عنوان گره مبنا در نظر گرفته می‌شود و زمین می‌شود.(گره ۴)
	\item
هنگام نوشتن KCL	برای هر گره فرض می کنیم که جریان‌ها خارج شونده هستند مگر اینکه منبع جریان داشته باشیم.(گره ۳)
\end{itemize}

\begin{example}
	ولتاژ گره‌ها را بدست آورید.
	
	\begin{center}
		\begin{circuitikz}
			\draw (-3,0) to[R,l=$3\Omega$] (3,0);
			\draw (-3,-1) to[R,l=$2\Omega$] (0,-1);
			\draw (0,-1) to[R,l=$4\Omega$] (3,-1);
			\draw (-3,-4) to[american current source,l=$10A$] (-3,-1);
			\draw (0,-4) to[R,l=$5\Omega$] (0,-1);
			\draw (3,-4) to[R,l=$6\Omega$] (3,-1);
			
			\draw (-3,-4) -- (3,-4);
			\draw (-3,0) -- (-3,-1);
			\draw (3,0) -- (3,-1);
		\end{circuitikz}
	\end{center}
	
\end{example}


\begin{solu}
	\ \\
		\begin{center}
		\begin{circuitikz}
			\draw (-3,0) to[R,l=$3\Omega$] (3,0);
			\draw (-3,-1) to[R,l=$2\Omega$,*-*] (0,-1) node[red, yshift=0.25cm]{2} node[red, xshift=-3.25cm,yshift=0.25cm]{1};
			\draw (0,-1) to[R,l=$4\Omega$,-*] (3,-1) node[xshift=0.25cm,red]{3};
			\draw (-3,-4) to[american current source,l=$10A$] (-3,-1);
			\draw (0,-1) to[R,l=$5\Omega$,-*] (0,-4) node[red, yshift=-0.25cm,xshift=0.25cm]{4};
			\draw (3,-4) to[R,l=$6\Omega$] (3,-1);
			\draw (0,-4) node[cground]{} (0,-5);
			
			\draw (-3,-4) -- (3,-4);
			\draw (-3,0) -- (-3,-1);
			\draw (3,0) -- (3,-1);
		\end{circuitikz}
	\end{center}

	\begin{gather*}
		\textcircled{1} \Longrightarrow -10 + \frac{V_1-V_2}{2} + \frac{V_1-V_3}{3} = 0
	\end{gather*}
	\begin{gather*}
		\textcircled{2} \Longrightarrow  \frac{V_2-V_1}{2} + \frac{V_2-0}{5} + \frac{V_2 - V_3}{4} = 0
	\end{gather*}
	\begin{gather*}
		\textcircled{3} \Longrightarrow  \frac{V_3-V_2}{4} + \frac{V_3}{6} + \frac{V_3 - V_1}{3} = 0
	\end{gather*}
	
	\begin{gather*}
		\begin{cases}
			\frac{5}{6}V_1 - \frac{1}{2}V_2 - \frac{1}{3}V_3 = 10 \vspace{0.25cm}\\
			-\frac{1}{2}V_1 + \frac{19}{20}V_2 - \frac{1}{4}V_3 = 0 \vspace{0.25cm} \\
			-\frac{1}{3}V_1 -\frac{1}{4}V_2 + \frac{3}{4}V_3 = 0
		\end{cases}
	\end{gather*}
در ادامه جواب را می‌توانید با استفاده از یکی از روش‌های حل دستگاه‌ها بدست آورید من از سایت محاسبه آنلاین برای بدست آوردن جواب استفاده کردم، راست و دروغ آن گردن خودشان.
\begin{gather*}
	\begin{cases}
		V_1 = 39.328 \\
		V_2 = 27.731 \\
		V_3 = 26.723
	\end{cases}
\end{gather*}
\end{solu}


\begin{remark}
	هر گاه میان دو گره اصلی یک منبع ولتاژ(مستقل یا وابسته) قرار داشته باشد ترکیب این دوگره و منبع به صورت یک گره در نظر گرفته می‌شود و به آن ابرگره می‌گوییم.
\end{remark}

\begin{example}
	ولتاژ گره‌ها را محاسبه کنید.
	\begin{center}
		\begin{circuitikz}
			\draw (-3,2) to[american voltage source] (-3,0);
			\draw (-1,0) to[R, l=$4\Omega$, i<=$i_x$] (-1,2);
			\draw (1,0) to[R, l=$2\Omega$] (1,2);
			\draw (3,0) to[american current source, l=$2A$] (3,2);
			\draw (-3,2) to[R,l=$1\Omega$] (-1,2);
			\draw (1,2) to[american controlled voltage source,l_=$3i_x$] (-1,2);
			\draw (1,2) to[short] (3,2);
			\draw (-3,0) to[short] (3,0);
		\end{circuitikz}
	\end{center}
\end{example}

\begin{solu}
	\ \\
	\begin{center}
		\begin{circuitikz}
			\draw (-3,2) to[american voltage source,*-] (-3,0) node[red,yshift=2.25cm]{1};
			\draw (-1,0) to[R, l=$4\Omega$, i<=$i_x$] (-1,2);
			\draw (1,0) to[R, l=$2\Omega$] (1,2);
			\draw (3,0) to[american current source, l=$2A$] (3,2);
			\draw (-3,2) to[R,l=$1\Omega$,-*] (-1,2) node[red,yshift=0.25cm]{2};
			\draw (1,2) to[american controlled voltage source,l_=$3i_x$] (-1,2);
			\draw (0,0) node[cground]{\textcolor{red}{4}} (0,-1);
			\draw (3,2) to[short,-*] (1,2) node[red,yshift=0.25cm]{3};
			\draw (-3,0) to[short] (3,0);
		\end{circuitikz}
	\end{center}
	\begin{gather*}
		\begin{cases}
			V_3 - V_2 = 3i_x = \frac{3}{4} V_2 \vspace{0.25cm}\\
			i_x = \frac{V_2}{4} \vspace{0.25cm} \\
			\frac{V_2 - V_1}{1} + \frac{V_2}{4} + \frac{V_3}{2} - 2 = 0
		\end{cases}
	\end{gather*}
		\begin{gather*}
		\begin{cases}
			-\frac{7}{4} V_2 + V_3 = 0\vspace{0.25cm}\\
			\frac{5}{4} V_2 +\frac{1}{2} V_3 = 6
		\end{cases}
	\end{gather*}
	\begin{gather*}
		\begin{cases}
			V_2 = 2.82\vspace{0.25cm}\\
			V_3 = 4.94
		\end{cases}
	\end{gather*}
\end{solu}

\begin{remark}
	علامت‌ها در تحلیل گره:
	\begin{center}
		\begin{circuitikz}
			\draw (-2,0) to[R,l=$R$,*-*] (2,0) node[xshift=-4cm,yshift=0.25cm]{$V_2$} node[yshift=0.25cm]{$V_3$};
			\draw[->] (0,-1) -- (1,-1) node[yshift=-0.5cm,xshift=-0.5cm]{$\frac{V_2-V_3}{R}$};
			\draw[->] (1,1) -- (0,1) node[yshift=0.5cm,xshift=0.5cm]{$\frac{V_3-V_2}{R}$};
		\end{circuitikz}
	\end{center}
\end{remark}

\subsection{تحلیل حلقه(خانه‌ای)}
هدف: بدست آوردن جریان شاخه‌ها(حلقه‌ها)
\begin{description}
	\item[گام اول] 
	برای هر حلقه‌ی ساده‌ی مدار یک جریان حلقه مشخص می‌کنیم.
	\item[گام دوم] 
	با نوشتن KVL در حلقه‌های ساده یک دستگاه چندمعادله چند مجهول درست خواهد شد.
	\item[گام سوم] 
	حل این دستگاه، مجاسبه جریان حلقه‌ها یا شاخه‌ها خواهد بود.
\end{description}

\begin{example}
	جریان حلقه‌ها را بدست آورید.
	
	\begin{center}
		\begin{circuitikz}
			\draw (-3,0) to[R,l=$4\Omega$] (3,0);
			\draw (-3,-1) to[R,l=$1\Omega$] (0,-1);
			\draw (0,-1) to[R,l=$2\Omega$] (3,-1);
			\draw (0,-4) to[american voltage source,l=$5V$] (0,-1);
			\draw (-3,-1) to[american voltage source,l_=$10V$] (-3,-4);
			\draw (3,-1) to[short] (3,-4);
			\draw (0,-4) to[R, l=$3\Omega$] (3,-4);
			
			\draw (-3,0) to[short] (-3,-1);
			\draw (3,0) to[short] (3,-1);
			\draw (-3,-4) -- (0,-4);
		\end{circuitikz}
	\end{center}
\end{example}


\begin{solu}
	\begin{description}
		\item[گام اول] 
		تعیین جریان برای حلقه‌های ساده مدار
		
		\begin{center}
			\begin{circuitikz}
				\draw (-3,0.5) to[R,l=$4\Omega$] (3,0.5);
				\draw (-3,-1) to[R,l=$1\Omega$] (0,-1);
				\draw (0,-1) to[R,l=$2\Omega$] (3,-1);
				\draw (0,-4) to[american voltage source,l=$5V$] (0,-1);
				\draw (-3,-1) to[american voltage source,l_=$10V$] (-3,-4);
				\draw (3,-1) to[short] (3,-4);
				\draw (0,-4) to[R, l=$3\Omega$] (3,-4);
				
				
				\draw (-3,0.5) to[short] (-3,-1);
				\draw (3,0.5) to[short] (3,-1);
				\draw (-3,-4) -- (0,-4);
				\draw[<-,red] (0.25,0) arc  (30:330:0.5) node[xshift=-0.5cm,yshift=0.25cm]{$I_3$};
				\draw[<-,red] (-1.25,-2.25) arc  (30:330:0.5) node[xshift=-0.5cm,yshift=0.25cm]{$I_1$};
				\draw[<-,red] (2,-2.25) arc  (30:330:0.5) node[xshift=-0.5cm,yshift=0.25cm]{$I_2$};
			\end{circuitikz}
		\end{center}
		
		
		
		
		
		\item[گام دوم] 
		نوشتن KVL در حلقه‌های ساده
		
		\begin{minipage}{0.5\textwidth}
			\begin{gather*}
				\Rightarrow
				\begin{cases}
					I_1 - I_3 = 15\\
					5I_2 - 2I_3 = -5 \\
					-I_1-2I_2 + 7I_3 = 0
				\end{cases}
			\end{gather*}
		\end{minipage}
		\begin{minipage}{0.5\textwidth}
			\begin{gather*}
				\begin{cases}
					\textcircled{1} \Rightarrow 1(I_1 - I_3) -5-10 = 0 \\
					\textcircled{2} \Rightarrow 2(I_2-I_3) + 3I_2 + 5 = 0 \\
					\textcircled{3} \Rightarrow 4(I_3) + 2(I_3 - I_2) + 1(I_3 - I_1) = 0
				\end{cases}
			\end{gather*}
		\end{minipage}
		
		
		
		
		\item[گام سوم] 
		حل این دستگاه، مجاسبه جریان حلقه‌ها یا شاخه‌ها خواهد بود.
		
		\begin{gather*}
			\begin{cases}
				I_1 = \frac{35}{2} A \\
				I_2 = 0 \\
				I_3 = \frac{5}{2} A
			\end{cases}
		\end{gather*}
	\end{description}
\end{solu}


\begin{example}
	جریان حلقه‌ها را بدست آورید.
	
	\begin{center}
		\begin{circuitikz}
			\draw (-3,0) to[R,l=$1\Omega$] (3,0);
			\draw (-3,-1) to[R,l=$3\Omega$] (0,-1);
			\draw (0,-1) to[R,l=$4\Omega$] (3,-1);
			\draw (0,-1) to[R,l=$2\Omega$] (0,-4);
			\draw (-3,-1) to[american voltage source,l_=$6V$] (-3,-4);
			\draw (3,-1) to[american controlled voltage source,l=$3i_x$] (3,-4);
			
			
			\draw (-3,0) to[short] (-3,-1);
			\draw (3,0) to[short] (3,-1);
			\draw (-3,-4) -- (3,-4);
		\end{circuitikz}
	\end{center}
\end{example}


\begin{solu}
	\begin{description}
		\item[گام اول] 
		تعیین جریان برای حلقه‌های ساده مدار
		
			\begin{center}
			\begin{circuitikz}
				\draw (-3,0.75) to[R,l=$1\Omega$] (3,0.75);
				\draw (-3,-1) to[R,l=$3\Omega$] (0,-1);
				\draw (0,-1) to[R,l=$4\Omega$] (3,-1);
				\draw (0,-1) to[R,l=$2\Omega$] (0,-4);
				\draw (-3,-1) to[american voltage source,l_=$6V$] (-3,-4);
				\draw (3,-1) to[american controlled voltage source,l=$3i_x$] (3,-4);
				
				
				\draw (-3,0.75) to[short] (-3,-1);
				\draw (3,0.75) to[short] (3,-1);
				\draw (-3,-4) -- (3,-4);
				\draw[<-,red] (0.25,0) arc  (30:330:0.5) node[xshift=-0.5cm,yshift=0.25cm]{$I_3$};
				\draw[<-,red] (-1,-2.25) arc  (30:330:0.5) node[xshift=-0.5cm,yshift=0.25cm]{$I_1$};
				\draw[<-,red] (2,-2.25) arc  (30:330:0.5) node[xshift=-0.5cm,yshift=0.25cm]{$I_2$};
			\end{circuitikz}
		\end{center}
		
		
		
		
		
		\item[گام دوم] 
		نوشتن KVL در حلقه‌های ساده
		
				\begin{minipage}{0.5\textwidth}
			\begin{gather*}
				\Rightarrow
				\begin{cases}
					5I_1 -2I_2 -3I_3 = 6 \\
					-2I_1 + 6I_2 - I_3 = 0 \\
					-3I_1 + 4I_2 + 8I_3 = 0
				\end{cases}
			\end{gather*}
		\end{minipage}
	\begin{minipage}{0.5\textwidth}
			\begin{gather*}
			\begin{cases}
				\textcircled{1} \Rightarrow 3(I_1 - I_3) + 2(I_1 - I_2) - 6 = 0 \\
				\textcircled{2} \Rightarrow 4(I_2-I_3) + 3i_x + 2(I_2 - I_1) = 0 \\
				\textcircled{3} \Rightarrow 1(I_3) + 4(I_3 - I_2) + 3(I_3 - I_1) = 0
			\end{cases}
		\end{gather*}
	\end{minipage}




		\item[گام سوم] 
		حل این دستگاه، مجاسبه جریان حلقه‌ها یا شاخه‌ها خواهد بود.
		
			\begin{gather*}
			\begin{cases}
				I_1 = 1.625 A \\
				I_2 = 0.594 A \\
				I_3 = 0.313 A
			\end{cases}
		\end{gather*}
	\end{description}
\end{solu}

\subsection{اصل برهم نهی(جمع آثار)}

هرگاه یک سیستم خطی با چند منبع مستقل(جریان یا ولتاژ) تعریف شود، پاسخ کامل را می‌توان مجموع تک تک منابع هنگامی که به تنهایی عمل می‌کنند دانست. برای بدست آوردن اثر یک منبع تنها، باید سایر منابع غیرفعال شوند.

\begin{remark}
	سیستم خطی: سیستمی که فقط مقاومت و منبع مستقل باشد.
\end{remark}


\begin{itemize}
	\item
	\textbf{غیرفعال کردن منبع ولتاژ}
	
	یعنی منبع ولتاژ را برمیداریم و به جای آن یک سیم می گذاریم.(اتصال کوتاه)
	
	\item
	\textbf{غیرفعال کردن منبع جریان}
	
	یعنی منبع را از مدار برداشته و جای آن را خالی می‌گذاریم.(اصطلاحاً مدار باز)
\end{itemize}

\begin{example}
	مقدار 
	$i_x$
	را با استفاده از قانون جمع آثار بدست آورید.
	
	\begin{center}
		\begin{circuitikz}
			\draw (-4.5, 3) to[american voltage source, l_=$4V$] (-4.5, 0);
			\draw (-1.5, 3) to[R, l_=$4\Omega$] (-1.5, 0);
			\draw (1.5, 3) to[R, l_=$8\Omega$] (1.5, 0);
			\draw (4.5, 0) to[american current source, l_=$2A$] (4.5, 3);
			
			\draw (-4.5, 3) to[R, l=$1\Omega$] (-1.5, 3);
			\draw (-1.5, 3) to[R, l=$2\Omega$] (1.5, 3);
			
			\draw (-4.5, 0) -- (4.5,0);
			\draw (1.5, 3) to[short] (4.5, 3);
		\end{circuitikz}
	\end{center}
	
\end{example}


\begin{solu}
	\begin{description}
		\item[گام اول] حذف منبع ولتاژ مستقل
		
	\begin{center}	
		\begin{circuitikz}
			\draw (-4.5, 3) to[short] (-4.5, 0);
			\draw (-1.5, 3) to[R, l_=$4\Omega$, i = $i_x$] (-1.5, 0);
			\draw (1.5, 3) to[R, l_=$8\Omega$] (1.5, 0);
			\draw (4.5, 0) to[american current source, l_=$2A$] (4.5, 3);
			
			\draw (-4.5, 3) to[R, l=$1\Omega$] (-1.5, 3) node[yshift=0.25cm,red]{1};
			\draw (-1.5, 3) to[R, l=$2\Omega$,*-*] (1.5, 3) node[yshift=0.25cm,red]{2};
			
			\draw (0,0) to[short,*-] (0,-0.25) node[cground,yshift=0.25cm,red]{3};
			
			
			\draw (-4.5, 0) -- (4.5,0);
			\draw (1.5, 3) to[short] (4.5, 3);
		\end{circuitikz}
	\end{center}
		
		\begin{remark}
			اگر در مدار همه منابع ولتاژ بود بهتر است که از روش تحلیل \textbf{حلقه }استفاده کنید. اگر در مدار همه منابع جریان بود بهتر است که از روش تحلیل \textbf{گره} استفاده کنید. اما اگر هر دوی آنها بود از اصل برهم نهی استفاده می‌کنیم. (هم منبع جریان و هم منبع ولتاژ فقط وابسته)
		\end{remark}
	

	\begin{minipage}{0.5\textwidth}
		\begin{gather*}
			\Rightarrow
			\begin{cases}
				\textcircled{1} \Rightarrow -\frac{1}{2}V_2 + \frac{7}{4}V_1  = 0 \\
				\textcircled{2} \Rightarrow \frac{5}{8}V_2 - \frac{1}{2}V_1 = 2
			\end{cases}
		\end{gather*}
	\end{minipage}
	\begin{minipage}{0.5\textwidth}
			\begin{gather*}
			\begin{cases}
				\textcircled{1} \Rightarrow \frac{V_1}{1} + \frac{V_1}{4} + \frac{V_1 - V_2}{2} = 0 \\
				\textcircled{2} \Rightarrow \frac{V_2 -  V_1}{2} + \frac{V_2}{8} - 2 = 0
			\end{cases}
		\end{gather*}
	\end{minipage}
		\begin{gather*}
			\Rightarrow
			\begin{cases}
				V_1 = 1.18 V \\
				V_2 = 4.15 V
			\end{cases}
		\end{gather*}
	
	\item[گام دوم] حذف منبع جریان مستقل
	
		\begin{center}
		\begin{circuitikz}
			\draw (-4.5, 3) to[american voltage source, l_=$4V$] (-4.5, 0);
			\draw (-1.5, 3) to[R, l_=$4\Omega$, i=$i_{x2}$] (-1.5, 0);
			\draw (1.5, 3) to[R, l_=$8\Omega$] (1.5, 0);
			
			
			\draw (-4.5, 3) to[R, l=$1\Omega$] (-1.5, 3);
			\draw (-1.5, 3) to[R, l=$2\Omega$] (1.5, 3);
			
			\draw (-4.5, 0) to[short,-*] (4.5,0);
			\draw (1.5, 3) to[short,-*] (4.5, 3);
			
			\draw[<-,red] (-2.65,1.75) arc  (30:330:0.5) node[xshift=-0.5cm,yshift=0.25cm]{$I_1$};
			\draw[<-,red] (0.5,1.75) arc  (30:330:0.5) node[xshift=-0.5cm,yshift=0.25cm]{$I_2$};
		\end{circuitikz}
	\end{center}
	
	
	\begin{minipage}{0.5\textwidth}
		\begin{gather*}
			\Rightarrow
			\begin{cases}
				5I_1  - 4I_2 = 4 \\
				 - 4I_1 +14I_2 = 0
			\end{cases}
		\end{gather*}
	\end{minipage}
	\begin{minipage}{0.5\textwidth}
		\begin{gather*}
			\begin{cases}
				1(I_1) + 4(I_1 - I_2) - 4 = 0 \\
				2I_2 + 8I_2 + 4(I_2 - I_1) = 0
			\end{cases}
		\end{gather*}
	\end{minipage}
	
	\begin{gather*}
		\Rightarrow
		\begin{cases}
			I_1= 1.037 \\
			I_2 = 0.296
		\end{cases}
	    \Rightarrow	i_{x2} = I_1 - I_2 = 0.741 \\
		i_x = i_{x1} + i_{x2} = 0.29 + 0.741 = 1.03A
	\end{gather*}
	\end{description}
\end{solu}

\subsection{تبدیل منابع}

قابلیت تعویض منابع ولتاژ و جریان با همدیگر بدون اثرگذاری روی بقیه مدار.

\begin{center}
	\begin{minipage}{0.3\textwidth}
		\begin{center}
			\begin{circuitikz}
				\draw (0,2) to[american voltage source, l_=$V_s$] (0,0);
				\draw (0,2) to[R, l=$R_s$,-*] (3,2) node[xshift=0.25cm]{a};
				\draw (0,0) to[short,-*] (3,0) node[xshift=0.25cm]{b};
			\end{circuitikz}
		\end{center}
	\end{minipage}
	\begin{minipage}{0.3\textwidth}
		\begin{center}
			$\Leftrightarrow$
		\end{center}
	\end{minipage}
	\begin{minipage}{0.3\textwidth}
		\begin{center}
			\begin{circuitikz}
				\draw (0,0) to[american current source, l=$i_s$] (0,2);
				\draw (2,2) to[R, l=$R_p$] (2,0);
				\draw (0,2) to[short,-*] (3,2) node[xshift=0.25cm]{a};
				\draw (0,0) to[short,-*] (3,0) node[xshift=0.25cm]{b};
			\end{circuitikz}
		\end{center}
	\end{minipage}
\end{center}
\begin{gather*}
	\begin{cases}
		R_s = R_p \\
		V_s = R_pi_s
	\end{cases}
\end{gather*}
\begin{remark}
	تبدیل منابع را می‌توان هم برای منابع مستقل و هم منابع وابسته استفاده کرد.
\end{remark}

\begin{example}
	با استفاده از تبدیل منابع مقدار 
	$I_0$
	را بدست آورید.
	
		\begin{center}
		\begin{circuitikz}
			\draw (-4.5, 3) to[american voltage source, l_=$6V$] (-4.5, 0);
			\draw (-1.5, 3) to[R, l_=$6\Omega$] (-1.5, 0);
			\draw (1.5, 0) to[american current source, l_=$2A$] (1.5, 3);
			\draw (4.5, 0) to[R, l_=$2\Omega$] (4.5, 3);
			
			\draw (-4.5, 3) to[R, l=$3\Omega$] (-1.5, 3);
			\draw (-1.5, 3) to[R, l=$2\Omega$] (1.5, 3);
			
			\draw (-4.5, 0) -- (4.5,0);
			\draw (1.5, 3) to[short, i = $I_0$] (4.5, 3);
		\end{circuitikz}
	\end{center}
\end{example}


\begin{solu}
	\ \\
		\begin{center}
		\begin{circuitikz}
			\draw (-4.5, 3) to[american voltage source, l_=$6V$] (-4.5, 0);
			\draw (-1.5, 3) to[R, l_=$6\Omega$] (-1.5, 0);
			\draw (1.5, 0) to[american current source, l_=$2A$] (1.5, 3);
			\draw (4.5, 0) to[R, l_=$2\Omega$] (4.5, 3);
			
			\draw (-4.5, 3) to[R, l=$3\Omega$] (-1.5, 3);
			\draw (-1.5, 3) to[R, l=$2\Omega$] (1.5, 3);
			
			\draw (-4.5, 0) -- (4.5,0);
			\draw (1.5, 3) to[short, i = $I_0$] (4.5, 3);
			\draw[red] (-5,1) arc (250:370:2);
		\end{circuitikz}
	\end{center}
	\begin{gather*}
		\Downarrow \text{\rl{با استفاده از تبدیل منابع}}
	\end{gather*}
	\begin{center}
		\begin{circuitikz}
			\draw (-4.5, 0) to[american current source, l=$2A$] (-4.5, 3);
			\draw (-2.5, 3) to[R, l_=$3\Omega$] (-2.5, 0);
			\draw (-1, 3) to[R, l_=$6\Omega$] (-1, 0);
			\draw (1.5, 0) to[american current source, l_=$2A$] (1.5, 3);
			\draw (4.5, 0) to[R, l_=$2\Omega$] (4.5, 3);
			
			\draw (-4.5, 3) to[short] (-1.5, 3);
			\draw (-1.5, 3) to[R, l=$2\Omega$] (1.5, 3);
			
			\draw (-4.5, 0) -- (4.5,0);
			\draw (1.5, 3) to[short, i = $I_0$] (4.5, 3);
		\end{circuitikz}
	\end{center}
	\begin{gather*}
		\Downarrow \text{\rl{دو مقاومت ۳ و ۶ اهمی موازی هستند.}}
	\end{gather*}
	\begin{center}
		\begin{circuitikz}
			\draw (-4.5, 0) to[american current source, l=$2A$] (-4.5, 3);
			\draw (-1.5, 3) to[R, l_=$2\Omega$] (-1.5, 0);
			\draw (1.5, 0) to[american current source, l_=$2A$] (1.5, 3);
			\draw (4.5, 0) to[R, l_=$2\Omega$] (4.5, 3);
			
			\draw (-4.5, 3) to[short] (-1.5, 3);
			\draw (-1.5, 3) to[R, l=$2\Omega$] (1.5, 3);
			
			\draw (-4.5, 0) -- (4.5,0);
			\draw (1.5, 3) to[short, i = $I_0$] (4.5, 3);
		\end{circuitikz}
	\end{center}
	\begin{gather*}
		\Downarrow \text{\rl{تبدیل منابع}}
	\end{gather*}
	\begin{center}
		\begin{circuitikz}
			\draw (-4.5, 3) to[american voltage source, l_=$4V$] (-4.5, 0);
			\draw (1.5, 0) to[american current source, l_=$2A$] (1.5, 3);
			\draw (4.5, 0) to[R, l_=$2\Omega$] (4.5, 3);
			
			\draw (-4.5, 3) to[R,l=$2\Omega$] (-1.5, 3);
			\draw (-1.5, 3) to[R, l=$2\Omega$] (1.5, 3);
			
			\draw (-4.5, 0) -- (4.5,0);
			\draw (1.5, 3) to[short, i = $I_0$] (4.5, 3);
		\end{circuitikz}
	\end{center}
	\begin{gather*}
		\Downarrow \text{\rl{دو مقاومت دو اهمی سری هستند. باهم جمع و سپس تبدیل منابع}}
	\end{gather*}
		\begin{center}
		\begin{circuitikz}
			\draw (-4.5, 0) to[american current source, l=$1A$] (-4.5, 3);
			\draw (1.5, 0) to[american current source, l_=$2A$] (1.5, 3);
			\draw (4.5, 0) to[R, l_=$2\Omega$] (4.5, 3);
			\draw (-1.5, 0) to[R, l_=$4\Omega$] (-1.5, 3);
			\draw (-4.5, 3) to[short] (-1.5, 3);
			\draw (-1.5, 3) to[short] (1.5, 3);
			
			\draw (-4.5, 0) -- (4.5,0);
			\draw (1.5, 3) to[short, i = $I_0$] (4.5, 3);
		\end{circuitikz}
	\end{center}
	\begin{remark}
		اگر منابع جریان هم جهت و موازی باشند(مانند این شکل) می‌توان آنها را با یکدیگر جمع کرد.
	\end{remark}
	\begin{center}
		\begin{circuitikz}
			\draw (-4.5, 0) to[american current source, l=$3A$] (-4.5, 3);
			\draw (4.5, 0) to[R, l_=$2\Omega$] (4.5, 3);
			\draw (-1.5, 0) to[R, l_=$4\Omega$] (-1.5, 3);
			\draw (-4.5, 3) to[short] (-1.5, 3);
			\draw (-1.5, 3) to[short] (1.5, 3);
			
			\draw (-4.5, 0) -- (4.5,0);
			\draw (1.5, 3) to[short, i = $I_0$] (4.5, 3);
		\end{circuitikz}
	\end{center}
حال با استفاده از فرمول تقسیم جریان می‌توان جریان مورد نظر را محاسبه کرد:
	\begin{gather*}
		I_0 = \frac{4}{2+4} \times 3 = 2A
	\end{gather*}
\end{solu}

\subsection{مدارهای هم ارز تونن و نورتن}

\textbf{هدف: }قرار دادن یک مدار ساده به جای قسمت بزرگی از مدار

برای نقطه‌ای از مدار که می‌خواهیم معادل تونن یا نورتن را بگذاریم، ولتاژ مدار باز
$(V_{th})$
،جریان نورتن
$(I_{sc})$
[جریان اتصال کوتاه] و مقاومت معادل تونن
$(R_{th}=\frac{V_{th}}{I_{sc}})$
را محاسبه می‌کنیم.

\begin{remark}
	\ \\
	\begin{itemize}
		\item 
		جریان همیشه در مسیر بسته برقرار است.
		\item
		وقتی در مسیر بسته‌ای منبع جریان باشد، تعیین کننده جریان آن منبع است.
	\end{itemize}
\end{remark}

\subsubsection*{ولتاژ مدار باز(ولتاژ تونن)}
المانی از مدار را که می‌خواهیم از دوسر آن مدار معادل را بدست آوریم، مدار باز می‌کنیم. مدار را با روش‌های تحلیلی که آموختیم، تحلیل کرده و ولتاژ دو سر مدار باز شده را بدست می‌آوریم.

\subsubsection*{جریان اتصال کوتاه(جریان نورتن)}
المانی از مدار را که می‌خواهیم از دو سر آن مدار معادل را بدست آوریم، اتصال کوتاه می‌کنیم. مدار را با روش‌های تحلیلی که پیش از این آموختیم، تحلیل کرده و  جریان گذرنده از این اتصال کوتاه را محاسبه می‌کنیم.

\begin{example}
	با استفاده از روش معادل سازی تونن و نورتن، ولتاژ 
	$V_0$
	را بدست آورید.
	
	\begin{center}
		\begin{circuitikz}
			\draw (-2,2) to[american voltage source, l_=$9V$] (-2,0);
			\draw (0,2) to[american current source, l_=$1A$] (0,0);
			\draw (2,0) to[R, l=$2\Omega$] (2,2) node[xshift = 0.5cm,yshift=-1cm,red]{$V_0$}
			node[xshift = 0.5cm,yshift=-0.5cm,red]{$+$} node[xshift = 0.5cm,yshift=-1.5cm,red]{$-$};
			\draw (0,2) to[R, l=$3\Omega$] (2,2);
			\draw (-2,2) to[R, l=$3\Omega$] (0,2);
			\draw (-2,0) to[short] (2,0);
		\end{circuitikz}
	\end{center}
\end{example}

\begin{solu}
	\begin{description}
		\item[گام اول] محاسبه ولتاژ تونن:
		
			\begin{center}
			\begin{circuitikz}
				\draw (-3,2) to[american voltage source, l_=$9V$] (-3,0);
					\draw (0,2) to[american current source, l_=$1A$] (0,0);
				\node[xshift = 2.25cm,yshift=1cm,red]{$V_0$}
				node[xshift = 2.25cm,yshift=1.75cm,red]{$+$} node[xshift = 2.25cm,yshift=0.25cm,red]{$-$};
				\draw (0,2) to[R, l=$3\Omega$,-*] (2,2);
				\draw (-3,2) to[R, l=$3\Omega$] (0,2);
				\draw (-3,0) to[short,-*] (2,0);
				\draw[->,blue] (-2,1) arc(150:-60:0.35);
			\end{circuitikz}
		\end{center}
		\begin{gather*}
			3\times 1 + V_{th} -9 = 0 \Rightarrow V_{th} = 6v
		\end{gather*}
		\item[گام دوم:] محاسبه جریان نورتن
			\begin{center}
			\begin{circuitikz}
				\draw (-2,2) to[american voltage source, l_=$9V$,*-] (-2,0) node[red,yshift=2.25cm]{9};
				\draw (0,2) to[american current source, l_=$1A$,-*] (0,0);
				\draw (2,0) to[short] (2,2) node[xshift = 0.5cm,yshift=-1cm,red]{$\downarrow I_{sc}$};
				\draw (0,2) to[R, l=$3\Omega$] (2,2);
				\draw (-2,2) to[R, l=$3\Omega$,-*] (0,2) node[red,yshift=0.25cm]{$V_1$};
				\draw (-2,0) to[short] (2,0);
				\draw (0,0) node[cground]{}(0,-1);	
			\end{circuitikz}
		\end{center}
		\begin{gather*}
			\frac{V_1-9}{3} + \frac{V_1}{3} + 1 = 0 \Rightarrow V_1 = 3v \vspace{0.5cm} \\
			I_{sc} = \frac{V_1}{3} = \frac{3}{3}= 1A
		\end{gather*}
	\item[گام سوم:] 
	\begin{gather*}
		R_{th} = \frac{V_{th}}{I_{sc}} = \frac{6}{1} = 6\Omega
	\end{gather*}
	\begin{center}
		\begin{circuitikz}
			\draw(0,2) to[american voltage source, l_=$6v$] (0,0);
			\draw(2,0) to[R, l=$2\Omega$] (2,2) node[red,xshift=0.5cm,yshift=-1cm]{$V_0$}
			node[red,xshift=0.5cm,yshift=-0.5cm]{$+$} node[red,xshift=0.5cm,yshift=-1.5cm]{$-$};
			\draw(0,2) to[R, l=$6\Omega$] (2,2);
			\draw(0,0) to[short] (2,0);
		\end{circuitikz}
	\end{center}
حال از فرمول تقسیم ولتاژ استفاده می‌کنیم:
	\begin{gather*}
		V_0 = \frac{2}{2+6} \times 6 = 1.5v
	\end{gather*}
	\end{description}
\end{solu}




	\section{القاگر(سلف) و خازن}
القاگر و خازن از عناصر ذخیره‌ای مدار هستند؛ یعنی می‌توانند انرژی محدودی را ذخیره کنند و در موقع لزوم به مدار برگردانند. جرقه سر شمع موتور خودرو نمونه‌ای از ذخیره‌ی انرژی در القاگر و جرقه لازم برای روشن شدن لامپ‌های مهتابی قدیمی‌تر نمونه‌ای از ذخیره انرژی توسط خازن است.

\subsection{القاگر}
القاگر یا سلف انرژی را در میدان مغناطیسی ذخیره می‌کند. القاگر در مدار به صورت زیر نشان داده می‌شود:

\begin{center}
	\begin{circuitikz}
		\draw (0,0) to[L,i=$I$,l=$L$] (2,0);
		\node[yshift=-2cm,xshift=1cm] () {$L=\frac{Q}{I}$};
		\draw[->] (0,-2) -- (-1,-2) node[xshift=-2.25cm]{\text{\rl{ضریب خودالقایی(القاکنایی)}}};
		\draw[->] (0,-2) -- (-1,-3) node[xshift=-1.25cm]{\text{\rl{واحد هانری(H)}}};
		\draw[->] (1.65,-1.75) -- (3,-1) node[xshift=2.25cm,yshift=0.25cm]{\text{\rl{شار مغناطیسی القا شده در القاگر}}};
		\draw[->] (1.35,-1.65) -- (1,-1) node[xshift=-0.5cm,yshift=0.25cm]{\text{\rl{واحد وبر(wb)}}};
		\draw[->] (1.65,-2.25) -- (3,-3) node[xshift=0.5cm,yshift=-0.25cm]{\text{\rl{واحد آمپر(A)}}};
		\draw[->] (1.65,-2.25) -- (3,-2.25) node[xshift=2.5cm]{\text{\rl{جریان الکتریکی عبوری از سلف}}};
	\end{circuitikz}
\end{center}

\subsubsection*{فرمول‌های مربوط به القاگر}
\begin{gather*}
	\text{\rl{رابطه ولتاژ و جریان}}\qquad  V = L \frac{dI}{dt} \Rightarrow \quad I = \frac{1}{L} \int_{0}^{t} Vdt + I(0)
\end{gather*}
\begin{remark}
	از آنجایی که القاگر وابسته به ارتباط جریان با گذر زمان است بنابراین هم می‌تواند خطی باشد و هم غیرخطی.
\end{remark}
\begin{gather*}
	\text{\rl{توان}}\qquad  P = VI = LI \frac{dI}{dt}  
\end{gather*}
\begin{gather*}
	\text{\rl{انرژی ذخیره شده}}\qquad  W = \int p dt = \frac{1}{2}LI^2
\end{gather*}

\subsubsection*{القاگر متقابل(M)}
القاگر متقابل پارامتری است برای مرتبط ساختن ولتاژ القایی در یک مدار.

\begin{minipage}{0.5\textwidth}
	\begin{gather*}
		\begin{cases}
			V_1 = L_1\frac{dI_1}{dt} + M\frac{dI_2}{dt} \vspace{0.5cm}\\
			V_2 = L_2\frac{dI_2}{dt} + M\frac{dI_2}{dt}
		\end{cases}
	\end{gather*}
\end{minipage}
\begin{minipage}{0.5\textwidth}
	\begin{center}
		\begin{circuitikz}
			\draw (0,0) to[short,i=$I_1$] (2,0);
			\draw (2,0) to[L] (2,-2) node[red,xshift=-0.5cm,yshift=1cm]{$V_1$}
			node[red,xshift=-0.5cm,yshift=1.5cm]{$+$} node[red,xshift=-0.5cm,yshift=0.5cm]{$-$};
			\draw (0,-2) to[short] (2,-2);
			\draw[<->,red] (3,0) arc(65:115:1.25) node[red,yshift=0.35cm,xshift=0.5cm]{M};
			\draw (3,0) to[short,i<=$I_2$] (5,0);
			\draw (3,0) to[L] (3,-2) node[red,xshift=0.5cm,yshift=1cm]{$V_2$}
			node[red,xshift=0.5cm,yshift=1.5cm]{$+$} node[red,xshift=0.5cm,yshift=0.5cm]{$-$};
			\draw (3,-2) to[short] (5,-2);
		\end{circuitikz}
	\end{center}
\end{minipage}

\begin{example}
	شکل موج جریان یک القاگر 
	$5mH$
	در شکل زیر داده شده است. شکل موج ولتاژ را رسم کنید.
	\begin{center}
		\includegraphics[scale=0.5]{images/sec04/sec04-fig1}
	\end{center}
\end{example}

\begin{solu}
	\ \\ 
	\begin{gather*}
		V_L = L\frac{di}{dt} \\
		0 \le t \le 2 \rightarrow i(t) = 15t \vspace{0.5cm} \\
		V_L(t) = 15 \times 5 \times 10^{-3} = 75mV \vspace{0.5cm} \\
		2 \le t \le 4 \rightarrow i(t) = -15t +30+30 \vspace{0.5cm} \\
		V_L(t) = -15 \times 5 \times 10^{-3} = -75mV \vspace{0.5cm} \\
	\end{gather*}
	
	\begin{center}
		\includegraphics[scale=0.5]{images/sec04/sec04-fig2}
	\end{center}
	
\end{solu}


\subsubsection*{ترکیب سلف‌ها}

سلف‌های سری:
\begin{center}
	\begin{circuitikz}
		\draw (-2,0) to[L, l=$L_1$] (0,0) ;
		\draw (0,0) to[L, l=$L_2$] (2,0) node[xshift=0.5cm]{$\cdots$};
		\draw (3,0) to[L, l=$L_N$] (5,0);
	\end{circuitikz}
\end{center}
\begin{gather*}
	L_{eq} = L_1 + L_2 + \cdots + L_N
\end{gather*}
سلف‌های موازی:

\begin{center}
	\begin{circuitikz}
		\draw (0,0) to[short] (1,0);
		\draw (1,-1.5) to[short] (1,1.5);
		\draw (1,1.5) to[L, l=$L_1$] (3,1.5);
		\draw (1,0.5) to[L, l=$L_2$] (3,0.5) node[xshift=-1cm,yshift=-0.65cm]{$\vdots$};
		\draw (1,-1.5) to[L, l=$L_N$] (3,-1.5);
		\draw (3,-1.5) to[short] (3,1.5);
		\draw (3,0) to[short] (4,0);
	\end{circuitikz}
\end{center}
\begin{gather*}
	\frac{1}{L_{eq}} = \frac{1}{L_1} + \frac{1}{L_2} + \cdots + \frac{1}{L_N} 
\end{gather*}

\begin{example}
	القاگر معادل شکل زیر را از دید a-b بدست آورید.
	
	\begin{center}
		\begin{circuitikz}
			\draw (0,0) to[L,l=$6H$,*-] (2,0) node[xshift=-2.25cm]{a};
			\draw (2,0) to[L,l=$2H$] (4,0);
			
			\draw (2,0) to[L,l=$6H$] (2,-2);
			\draw (2,-2) to[L,l=$4H$] (2,-4);
			
			\draw (2,-2) to[short] (4,-2);
			
			\draw (4,0) to[L,l=$1H$] (4,-2);
			\draw (4,-2) to[L,l=$8H$] (4,-4);
			
			\draw (0,-4) to[L,l=$10H$,*-] (2,-4) node[xshift=-2.25cm]{b};
			\draw (2,-4) to[L,l=$4H$] (4,-4);
		\end{circuitikz}
	\end{center}
	
\end{example}
\begin{solu}
	\ \\
	\begin{gather*}
		L_{2,1} = 2+1 = 3H \\
		L_{3,6} = \frac{3\times 6}{9} = 2H \vspace{1cm}\\
		L_{4,8} = 4+8 = 12H \\
		L_{12,4} = \frac{4\times 12}{16} = 3H\\
	\end{gather*}
	\begin{center}
		\begin{circuitikz}
			\draw (0,0) to[L,l=$6H$,*-] (2,0) node[xshift=-2.25cm]{a};
			\draw (2,0) to[short] (4,0);
			\draw[red] (4,0) to[L,l=$2H$] (4,-2);
			\draw[red] (4,-2) to[L,l=$3H$] (4,-4);
			\draw (0,-4) to[L,l=$10H$,*-] (2,-4) node[xshift=-2.25cm]{b};
			\draw (2,-4) to[short] (4,-4);
		\end{circuitikz}
	\end{center}
	\begin{gather*}
		L_{eq} = 6+2+3+10=21H
	\end{gather*}
\end{solu}



\subsection{خازن}
عنصری است که برای ذخیره‌ی انرژی در میدان الکتریکی استفاده می‌شود. خازن در مدار با نماد زیر نشان داده می‌شود.

\begin{center}
	\begin{circuitikz}
		\draw (0,0) to[capacitor, i=$I$,l=$C$] (2,0);
	\end{circuitikz}
\end{center}

\subsubsection*{فرمول‌های مربوط به خازن}

\begin{gather*}
	I=C\frac{dv}{dt} \rightarrow V = \frac{1}{C} \int_{0}^{t} I dt + V(0)
\end{gather*}

\begin{gather*}
	P=VI = CV\frac{dV}{dt}
\end{gather*}

\begin{gather*}
	W=\int Pdt = \frac{1}{2}CV^2
\end{gather*}


\begin{example}
	شکل موج ولتاژ اعمال شده به خازن 
	$6\mu F$
	به صورت زیر است. شکل موج جریان را بدست آورید.
		\begin{center}
		\includegraphics[scale=0.5]{images/sec04/sec04-fig3}
	\end{center}
\end{example}

\begin{solu}
	\begin{gather*}
		0 \le t \le 4 \Rightarrow V(t) = 7.5t \times 10^{-3} \vspace{0.5cm} \\
		i(t) = C\frac{dV}{dt} = 6\mu F \times 7.5 \times 10^{3} = 45mA \vspace{0.5cm}\\
		4 \le t \le 6 \Rightarrow V(t) = -15t \times 10^{3} +60+30 \vspace{0.5cm} \\
		i(t) = C\frac{dV}{dt} = 6\mu F \times -15 \times 10^{3} = -90mA \vspace{0.5cm}\\
	\end{gather*}
	\begin{center}
		\includegraphics[scale=0.5]{images/sec04/sec04-fig4}
	\end{center}
\end{solu}
\subsubsection*{ترکیب خازن‌ها}

خازن‌های سری:
\begin{center}
	\begin{circuitikz}
		\draw (-2,0) to[capacitor, l=$C_1$] (0,0) ;
		\draw (0,0) to[capacitor, l=$C_2$] (2,0) node[xshift=0.5cm]{$\cdots$};
		\draw (3,0) to[capacitor, l=$C_N$] (5,0);
	\end{circuitikz}
\end{center}
\begin{gather*}
	\frac{1}{C_{eq}} = \frac{1}{C_1} + \frac{1}{C_2} + \cdots + \frac{1}{C_N} 
\end{gather*}

خازن‌های موازی:

\begin{center}
	\begin{circuitikz}
		\draw (0,0) to[short] (1,0);
		\draw (1,-1.5) to[short] (1,2);
		\draw (1,2) to[capacitor, l=$C_1$] (3,2);
		\draw (1,0.5) to[capacitor, l=$C_2$] (3,0.5) node[xshift=-1cm,yshift=-0.65cm]{$\vdots$};
		\draw (1,-1.5) to[capacitor, l=$C_N$] (3,-1.5);
		\draw (3,-1.5) to[short] (3,2);
		\draw (3,0) to[short] (4,0);
	\end{circuitikz}
\end{center}
\begin{gather*}
	C_{eq} = C_1 + C_2 + \cdots + C_N
\end{gather*}

\begin{example}
	خازن معادل شکل زیر را از دید a-b بدست آورید.
	
	\begin{center}
		\begin{circuitikz}
			\draw (0,0) to[capacitor,l=$6$,*-] (2,0) node[xshift=-2.25cm]{a};
			\draw (2,0) to[capacitor,l=$2$] (4,0);
			
			\draw (2,0) to[capacitor,l_=$6$] (2,-2);
			\draw (2,-2) to[capacitor,l_=$4$] (2,-4);
			
			\draw (2,-2) to[short] (4,-2);
			
			\draw (4,0) to[capacitor,l=$1$] (4,-2);
			\draw (4,-2) to[capacitor,l=$8$] (4,-4);
			
			\draw (0,-4) to[capacitor,l_=$10$,*-] (2,-4) node[xshift=-2.25cm]{b};
			\draw (2,-4) to[capacitor,l_=$4$] (4,-4);
		\end{circuitikz}
	\end{center}
	
\end{example}
\begin{solu}
	\ \\
	\begin{gather*}
		C_{2,1} = \frac{1\times 2}{1+2} = \frac{2}{3}\\
		C_{\frac{2}{3},6} = \frac{2}{3} + 6 = \frac{20}{3} \vspace{1cm}\\
			C_{4,8} = \frac{4\times 8}{12} = \frac{8}{3}\\
		C_{\frac{8}{3},4} = \frac{8}{3} + 4 = \frac{20}{3} \vspace{1cm}\\
	\end{gather*}
	\begin{center}
		\begin{circuitikz}
			\draw (0,0) to[capacitor,l=$6$,*-] (2,0) node[xshift=-2.25cm]{a};
			\draw (2,0) to[short] (4,0);
			\draw[red] (4,0) to[capacitor,l=$\frac{20}{3}$] (4,-2);
			\draw[red] (4,-2) to[capacitor,l=$\frac{20}{3}$] (4,-4);
			\draw (0,-4) to[capacitor,l=$10$,*-] (2,-4) node[xshift=-2.25cm]{b};
			\draw (2,-4) to[short] (4,-4);
		\end{circuitikz}
	\end{center}
	\begin{gather*}
		\frac{1}{C_{eq}} =\frac{1}{6} + \frac{3}{20} + \frac{3}{20} + \frac{1}{10} = \frac{17}{30} \\
		\Rightarrow C_{eq} = 1.7
	\end{gather*}
\end{solu}

	\section{پاسخ طبیعی پله واحد}

\textbf{تابع تحریک پله واحد:}

\begin{gather*}
	u(t) = 
	\begin{cases}
		1 & t>0 \\
		0 & t<0
	\end{cases}
\end{gather*}

\textbf{تابع پله واحد تأخیردار: }

\begin{gather*}
	u(t-t_0) = 
	\begin{cases}
		1 & t>t_0 \\
		0 & t<t_0
	\end{cases}
\end{gather*}

\textbf{تابع تحریک ضربه: }

\begin{gather*}
	\delta(t) = 
	\begin{cases}
		1 & t=0 \\
		0 & otherwise
	\end{cases}
\end{gather*}

\begin{gather*}
	\delta(t-t_0) = 
	\begin{cases}
		1 & t=t_0 \\
		0 & O.W
	\end{cases}
\end{gather*}

\begin{remark}
	برای بدست آوردن پاسخ ضربه(یعنی پاسخ مدار به ورودی ضربه) کافی است ابتدا پاسخ پله(یعنی پاسخ مدار به ورودی پله) را بدست آوریم سپس از این پاسخ مشتق بگیریم.
\end{remark}

\begin{definition}
	پاسخ مدار یعنی به دست آوردن جریان یا ولتاژ در یک نقطه از مدار.
\end{definition}

\begin{remark}
	\ \\
	\begin{itemize}
		\item
		اگر مدار \textbf{مقاومتی} باشد و منبع نیز \textbf{متغیر} باشد، پاسخ مدار \textbf{متغیر}(وابسته به زمان) است.
		\item
		اگر مدار \textbf{مقاومتی} باشد و منبع \textbf{ثابت} باشد، پاسخ \textbf{ثابت} است.
		\item
		اگر مدار شامل \textbf{سلف یا خازن} و یا هردو باشد، پاسخ \textbf{همیشه متغیر}(وابسته به زمان) است.
	\end{itemize}
\end{remark}

\begin{definition}
	\textbf{درجه} به معنای بزرگترین توان  متغیر و \textbf{مرتبه} به معنای تعداد دفعاتی است که می‌توان مشتق گرفت.
\end{definition}

\begin{definition}
	\textbf{مدارهای مرتبه اول}، مدارهایی هستند که برای بدست آوردن رابطه‌ی ولتاژ یا جریان در آن به یک معادله مرتبه اول می‌رسیم.
\end{definition}

مدارهای مرتبه اول دو نوع هستند:
\begin{itemize}
	\item
	\textbf{مدارهای \lr{RL}:} این نوع مدارها شامل مقاومت و سلف هستند.
	\item
	\textbf{مدارهای \lr{RC}:} این نوع مدارها شامل مقاومت و خازن هستند.
\end{itemize}

در مدارهای مرتبه اول، به جای حل معادله‌ی مرتبه اول دیفرانسیل با روش‌های تشریحی می‌توان از فرمول زیر استفاده کرد:

\begin{gather*}
	y(t) = y(\infty) + \Big(y(0) - y(\infty)\Big).e^{-\frac{t}{\tau}}
\end{gather*}
 حال برای انواع مدارها داریم:
 
 \begin{minipage}{0.5\textwidth}
 	\begin{gather*}
 		RL \rightarrow 
 		\begin{cases}
 			y = I \\
 			\tau = \frac{L}{R_{eq}}
 		\end{cases}
 	\end{gather*}
 \end{minipage}
 \begin{minipage}{0.5\textwidth}
	\begin{gather*}
		RC \rightarrow 
		\begin{cases}
			y = V \\
			\tau = R_{eq}.C
		\end{cases}
	\end{gather*}
\end{minipage}


\begin{example}
	رابطه‌ی جریان را برای مدار زیر به دست آورید.
	
	\begin{center}
		\begin{circuitikz}
			\draw (0,2) to[american voltage source, l_=$12^{u(t)}$] (0,0);
			\draw (0,2) to[L,l=$1H$,i=$I$] (2,2);
			\draw (2,2) to[R,l=$1\Omega$] (2,0);
			\draw (0,0) to[short] (2,0);
		\end{circuitikz}
	\end{center}
\end{example}

\begin{solu}
	\begin{gather*}
		\tau = \frac{L}{R_{eq}} = \frac{1}{1} = 1 \\
		I(\infty) = 12A \\
		I(0) = 0 \\
		I(t) = 12 + (0-12)e^{\frac{-t}{1}} \\
		\rightarrow I(t) = 12(1-e^{-t})
	\end{gather*}
\end{solu}

\begin{remark}
	\ \\
	\begin{itemize}
		\item 
		سلف وقتی پر می‌شود به صورت اتصال کوتاه عمل می‌کند.
		\item
		خازن وقتی پر می‌شود، جریانی از خود عبور نمی‌دهد و آن را به صورت مدار باز در نظر بگیرید.
		\item
		سلف را در ابتدای مدار به صورت مدار باز در نظر بگیرید.
		\item
		خازن را در ابتدای مدار به صورت اتصال کوتاه در نظر بگیرید.
	\end{itemize}
\end{remark}

\begin{example}
	جریان گذرنده از سلف را به دست آورید.
	
	\begin{center}
		\begin{circuitikz}
			\draw (0,2) to[american voltage source, l_=$10u(t)$] (0,0);
			\draw (0,2) to[R, l=$2\Omega$] (2,2);
			\draw (2,0) to[L, l=$1H$] (2,2);
			\draw (4,0) to[R, l=$2\Omega$] (4,2);
			\draw (2,2) to[short] (4,2);
			\draw (0,0) to[short] (4,0);
		\end{circuitikz}
	\end{center}
\end{example}

\begin{solu}
	\ \\
	\begin{minipage}{0.4\textwidth}
		\begin{center}
			\begin{circuitikz}
				\draw (0,0) to[R,l=$2\Omega$] (0,2);
				\draw (2,0) to[R,l_=$2\Omega$] (2,2);
				\draw (1,0) to[short,-*] (1,0.5);
				\draw (1,2) to[short,-*] (1,1.5);
				\draw (0,2) to[short] (2,2);
				\draw (0,0) to[short] (2,0);
			\end{circuitikz}
		\end{center}
	\end{minipage}
	\begin{minipage}{0.1\textwidth}
		$\Rightarrow$
	\end{minipage}
	\begin{minipage}{0.4\textwidth}
			\begin{center}
			\begin{circuitikz}
				\draw (0,2) to[american voltage source, l_=$10u(t)$] (0,0);
				\draw (0,2) to[R, l=$2\Omega$] (2,2);
				\draw (2,0) to[L, l=$1H$] (2,2);
				\draw (4,0) to[R, l=$2\Omega$] (4,2);
				\draw (2,2) to[short] (4,2);
				\draw (0,0) to[short] (4,0);
			\end{circuitikz}
		\end{center}
	\end{minipage}
	\begin{gather*}
		R_{eq} = 2 || 2 = 1\Omega \\
		\tau = \frac{L}{R_{eq}} = \frac{1}{1} = 1 \\
		I(\infty) = \frac{10}{2} = 5A \\
		I(0) = 0 \\
		\rightarrow I(t) = 5+(0-5)e^{-t} = 5(1-e^{-t})
	\end{gather*}
\end{solu}

\begin{example}
	در مدار شکل زیر 
	$V_c(t)$
	را برای 
	$t>0$
	بیابید.
	\begin{center}
		\begin{circuitikz}
			\draw (0,2) to[american voltage source, l_=$36v$] (0,0);
			\draw (3,0) to[capacitor, l=$100\mu F$] (3,2) node[red,xshift=1cm,yshift=-1cm]{$V_c(t)$}
			node[red,xshift=1cm,yshift=-0.5cm]{$+$} node[red,xshift=1cm,yshift=-1.5cm]{$-$};
			\draw (6,0) to[switch,l=$t\=0$] (6,2);
			\draw (9,2) to[american voltage source, l=$12v$] (9,0);
			\draw (0,2) to[R,l=$2k$] (3,2);
			\draw (3,2) to[R,l=$6k$] (6,2);
			\draw (6,2) to[R,l=$4k$] (9,2);
			\draw (0,0) to[short] (9,0);
		\end{circuitikz}
	\end{center}
\end{example}

\begin{remark}
	اگر در مداری کلید داشتیم یک مرحله به مراحل قبل اضافه می‌شود.
\end{remark}


\begin{solu}
	مدار برای زمان 
	$t=0^-$
	به صورت زیر است.
		\begin{center}
		\begin{circuitikz}
			\draw (0,2) to[american voltage source, l_=$36v$] (0,0);
			\draw (3,0) to[short,-*] (3,0.5) node[red,xshift=0.5cm,yshift=0.5cm]{$V_c(0^-)$}
			node[red,xshift=0.5cm,yshift=1cm]{$+$} node[red,xshift=0.5cm,yshift=0cm]{$-$};
			\draw (3,2) to[short,-*] (3,1.5);
			\draw (9,2) to[american voltage source, l=$12v$] (9,0);
			\draw (0,2) to[R,l=$2k$] (3,2);
			\draw (3,2) to[R,l=$6k$] (6,2);
			\draw (6,2) to[R,l=$4k$] (9,2);
			\draw (0,0) to[short] (9,0);
		\end{circuitikz}
	\end{center}
 طیق قانون KVL داریم:
	\begin{gather*}
		2^kI + 6^kI = 4^kI +12-36=0 \rightarrow I = \frac{24}{12^k} = 2mA \\
		-36 + 2^k\times 2mA + V_c(0^-) = 0 \rightarrow V_c(0^-) = 32V
	\end{gather*}
	\begin{remark}
		ولتاژ خازن مقداری پیوسته است یعنی ولتاژ خازن در 
		$t=0^-$
		و 
		$t=0^+$
		با هم برابر است.
	\end{remark}
	مدار برای زمان 
	$t=0^+$
	به صورت زیر است.
	\begin{center}
		\begin{circuitikz}
			\draw (0,2) to[american voltage source, l_=$36v$] (0,0);
			\draw (3,0) to[capacitor, l=$100\mu F$] (3,2) node[red,xshift=1.75cm,yshift=-1cm]{{\scriptsize $V_c(0^+)=32V$}}
			node[red,xshift=1cm,yshift=-0.5cm]{$+$} node[red,xshift=1cm,yshift=-1.5cm]{$-$};
			\draw (6,0) to[short] (6,2);
			\draw (9,2) to[american voltage source, l=$12v$] (9,0);
			\draw (0,2) to[R,l=$2k$] (3,2);
			\draw (3,2) to[R,l=$6k$] (6,2);
			\draw (6,2) to[R,l=$4k$] (9,2);
			\draw (0,0) to[short] (9,0);
		\end{circuitikz}
	\end{center}
	\begin{remark}
		\ \\
		\begin{itemize}
			\item
			در زمان 
			$t=0^+$
			خازن نقش یک منبع را دارد.
			\item
			در زمان 
			$t=0^+$
			کلید وصل است.
		\end{itemize}
	\end{remark}
	مدار برای زمان 
	$t>0$
	به صورت زیر خواهد بود:
		\begin{center}
		\begin{circuitikz}
			\draw (0,2) to[american voltage source, l_=$36v$] (0,0);
			\draw (3,0) to[capacitor, l=$100\mu F$] (3,2) node[red,xshift=1cm,yshift=-1cm]{{\scriptsize $V_c(t)$}}
			node[red,xshift=1cm,yshift=-0.5cm]{$+$} node[red,xshift=1cm,yshift=-1.5cm]{$-$};
			\draw (6,0) to[short] (6,2);
		
			\draw (0,2) to[R,l=$2k$] (3,2);
			\draw (3,2) to[R,l=$6k$] (6,2);
		
			\draw (0,0) to[short] (6,0);
		\end{circuitikz}
	\end{center}
	مدار برای زمان 
	$t=+\infty$
	به صورت زیر خواهد بود:
	
	\begin{center}
		\begin{circuitikz}
			\draw (0,2) to[american voltage source, l_=$36v$] (0,0);
			\draw (3,0) to[short,-*] (3,0.5) node[red,xshift=0.5cm,yshift=0.5cm]{$V_c(\infty)$}
			node[red,xshift=0.5cm,yshift=1cm]{$+$} node[red,xshift=0.5cm,yshift=0cm]{$-$};
			\draw (6,0) to[short] (6,2);
			\draw (3,2) to[short,-*] (3,1.5);
			\draw (0,2) to[R,l=$2k$] (3,2);
			\draw (3,2) to[R,l=$6k$] (6,2);
			
			\draw (0,0) to[short] (6,0);
		\end{circuitikz}
	\end{center}
	حال داریم:
	\begin{gather*}
		V_c = \frac{6}{2+6}\times 36 = 27v \\
		\tau = R_{eq}.C \rightarrow R_{eq} = 2 || 6 = \frac{2\times 6}{8} = 1.5 \Omega \\
		\tau = 1.5 \times 10^3 \times 100 \times 10^{-6} = 150 \times 10^{-3}=0.15 \\
		V_c(t) = V_c(\infty) + (V_t(0) - V_t(\infty))e^{\frac{-t}{\tau}} \\
		=27 + (32-27)e^{\frac{-t}{0.15}} \\
		V_c(t)=27 + 5e^{\frac{-t}{0.15}}
	\end{gather*}
\end{solu}

\subsection{پاسخ‌های طبیعی و پله مدارهای RLC}

\begin{definition}
	\textbf{مدارهای \lr{RLC}: }مدارهایی که هم خازن دارند، هم سلف و هم مقاومت. این مدارها از نوع مرتبه دوم هستند(یعنی توصیف آن با معادلات دیفرانسیل خطی مرتبه دوم انجام می‌شود.)
\end{definition}

\noindent
\textbf{روال حل مسائل این قسمت:}

ابتدا مدارهای RLC بدون منبع را در نظر می‌گیریم و پاسخ طبیعی را بدست می‌آوریم. سپس منابع DC کلیدها یا منابع پله را به مدار می‌افزاییم و پاسخ کامل را به صورت مجموع پاسخ ویژه(واداشته) و طبیعی می‌نویسیم و مقدار ثابت‌ها را با اعمال شرایط اولیه پیدا می‌کنیم.


\subsubsection*{پاسخ طبیعی مدار RLC موازی:}

پاسخ طبیعی یعنی فرض می کنیم که مدار بدون منبع است و در سلف و خازن انرژی ذخیره شده است. انرژی ذخیره شده اولیه سلف را 
$I_0$
(جریان) و انرژی ذخیره شده اولیه خازن را 
$V_0$
(ولتاژ) در نظر می‌گیریم.

\begin{center}
	\begin{circuitikz}
		\draw (0,0) to[capacitor,l=$C$] (0,2);
		\draw (0,2) to[short,i<=$i_C$,-*] (2,2);
		\draw (2,2) to[short,i=$i_R$] (4,2);
		\draw (0,0) to[short] (4,0);
		\draw (2,0) to[L,i<=$i_L$,l=$L$] (2,2);
		\draw (4,0) to[R,l=$R$] (4,2);
	\end{circuitikz}
\end{center}
\begin{gather*}
	i_R + i_C + i_L = 0 \\
	\Rightarrow 
	\begin{cases}
		i_R = \frac{V}{R} \\
		i_L = \frac{1}{L} \int_{0}^{t} V dt + I_0 \\
		i_c = C\frac{dv}{dt}
	\end{cases}
\end{gather*}
\begin{gather*}
	i_R + i_C + i_L = 0 \\
	\Rightarrow \frac{V}{R} + C\frac{dV}{dt} + \frac{1}{L} \int_{0}^{t} V dt + I_0 = 0
\end{gather*}
معادلاتی که هم مشتق دارند و هم انتگرال، باید یا مشتق را حذف کنیم و یا انتگرال. برای حذف انتگرال از کل معادله مشتق می‌گیریم.

\begin{gather*}
	\frac{1}{R} \frac{dV}{dt} + C \frac{d^2V}{dt^2} + \frac{1}{L} V = 0 
\end{gather*}

\begin{gather}\label{V-formula}
	\frac{d^2V}{dt^2} + \frac{1}{RC}\frac{dV}{dt} + \frac{V}{LC} = 0 \quad \rightarrow \quad \text{\rl{معادله دیفرانسیل مرتبه دوم}}
\end{gather}
فرض می‌کنیم که 
$V=ke^{st}$
در این صورت خواهیم داشت:
\begin{gather*}
	V=ke^{st} \rightarrow \quad \frac{d'V}{dt} = kse^{st} \rightarrow \quad \frac{d^2V}{dt} = ks^2e^{st} 
\end{gather*}

حال مقدار V را در معادله دیفرانسیل 
\eqref{V-formula}
جایگذاری می‌کنیم و خواهیم داشت:
\begin{gather*}
	ks^2e^{st} + \frac{1}{RC} kse^{st} + \frac{ke^{st}}{LC} = 0 \\
	\overbrace{ke^{st}}^1\Big(s^2 + \frac{1}{RC}s + \frac{1}{LC}\Big) = 0
\end{gather*}
قسمت یک که هیچ‌گاه صفر نمی‌شود بنابراین:
\begin{gather*}
	s^2 + \frac{1}{RC}s + \frac{1}{LC} = 0
\end{gather*}
که به آن معادله مشخصه مدار گفته می‌شود زیر ریشه‌های این معادله رابطه ریاضی 
$V(t)$
را تعیین خواهد کرد. به کمک روش دلتا نیز می‌توان ریشه‌های آن را بدست آورد که ریشه‌های به صورت زیر خواهند بود:
\begin{gather*}
	\begin{cases}
		s_1 = \frac{-1}{2RC} + \sqrt{(\frac{1}{2RC})^2 - \frac{1}{LC}} \vspace{0.5cm} \\
		s_2 = \frac{-1}{2RC} - \sqrt{(\frac{1}{2RC})^2 - \frac{1}{LC}}
	\end{cases}
	V = V_1 + V_2 = k_1e^{s_1t} + k_2e^{s_2t}
\end{gather*}


حال برای ساده‌تر کردن ریشه‌ها می‌توان عبارت‌های زیر را در نظر گرفت:
\begin{gather*}
	\begin{cases}
		\alpha = \frac{1}{2RC} &\quad \rightarrow \quad \text{\rl{ضریب میرایی پاسخ طبیعی}} \\
		\omega_0 = \frac{1}{\sqrt{LC}} &\quad \rightarrow \quad \text{\rl{فرکانش تشدید پاسخ طبیعی(مدار)}}
	\end{cases}
\end{gather*}
بنابراین:
\begin{gather*}
	\begin{cases}
		s_1 = \frac{-1}{2RC} + \sqrt{(\frac{1}{2RC})^2 - \frac{1}{LC}} = -\alpha + \sqrt{\alpha^2 - \omega_0^2} \vspace{0.5cm} \\
		s_2 = \frac{-1}{2RC} - \sqrt{(\frac{1}{2RC})^2 - \frac{1}{LC}} = -\alpha - \sqrt{\alpha^2 - \omega_0^2}
	\end{cases}
\end{gather*}

حال براساس 
$\alpha$
و 
$\omega_0$
سه نوع جواب داریم:

\begin{description}
	\item[$\alpha > \omega_0$:] 
	در این حالت دو ریشه حقیقی و متمایز داریم و به آن پاسخ طبیعی فوق میرا گفته می‌شود. 
	\begin{gather*}
		V(t) = k_1e^{s_1t}+k_2e^{s_2t}
	\end{gather*}

	\item[$\alpha < \omega_0$:] 
	در این حالت دو ریشه مختلط و مزدوج داریم و به آن پاسخ طبیعی زیرمیرا گفته می‌شود. 
	\begin{gather*}
		V(t) = e^{-\alpha t}\Big(k_1 cos\omega_n t+ k_2sin \omega_n t\Big)
	\end{gather*}
	\begin{remark}
		به $\omega_n$ فرکانس تشدید طبیعی گفته می‌شود که از رابطه‌ی زیر محاسبه می‌شود:
		\begin{gather*}
			\omega_n = \sqrt{\omega_0^2 - \alpha^2}
		\end{gather*}
	\end{remark}

	\item[$\alpha = \omega_0$:] 
	در این حالت هر دو ریشه حقیقی و مساوی است و به آن پاسخ طبیعی میرای بحرانی گفته می‌شود.
	\begin{gather*}
		V(t) = k_1te^{-\alpha t} + k_2e^{-\alpha t}
	\end{gather*}
\end{description}

\begin{example}
	پاسخ طبیعی مدار زیر را بدست آورید.
	\begin{center}
		\begin{circuitikz}
			\draw (0,0) to[capacitor,l=$\frac{1}{16}F$] (0,2);
			\draw (0,2) to[short] (2,2);
			\draw (2,2) to[short] (4,2);
			\draw (0,0) to[short] (4,0);
			\draw (2,0) to[L,l=$4H$] (2,2);
			\draw (4,0) to[R,l=$2\Omega$] (4,2);
		\end{circuitikz}
	\end{center}
\end{example}

\begin{solu}
	\ \\
	\begin{gather*}
		\alpha = \frac{16}{2\times 2 } = 4 \\
		\omega_0 = \frac{1}{\sqrt{\frac{1}{4}}} = 2
	\end{gather*}
با توجه به حالات بالا پاسخ طبیعی فوق میراست. لذا:
\begin{gather*}
		V(t) = k_1e^{s_1t}+k_2e^{s_2t} \\
		s_1 = -4 + \sqrt{16-4} = -0.54 \\
		s_2 = -4 - \sqrt{16-4} = -7.46 \\
		V(t) = k_1e^{-0.54t}+k_2e^{-7.46t} \\
\end{gather*}
\end{solu}

\begin{example}
	در مدار شکل زیر اگر 
	$V_c(0) = 2V$
	و
	$i_L(0) = -2A$
	باشد، مطلوب است مقدار 
	$V(t)$
	برای
	$t>0$.

	\begin{center}
		\begin{circuitikz}
			\draw (0,0) to[capacitor,l=$\frac{1}{4}F$] (0,2);
			\draw (0,2) to[short] (2,2);
			\draw (2,2) to[short] (4,2);
			\draw (0,0) to[short] (4,0);
			\draw (2,0) to[L,l=$1H$] (2,2);
			\draw (4,0) to[R,l=$2\Omega$] (4,2) node[yshift=-0.25cm,xshift=0.75cm](){+} node[yshift=-1.75cm, xshift=0.75cm]{-}
			node[yshift = -1cm, xshift=0.75cm]{$V(t)$};
		\end{circuitikz}
	\end{center}

\end{example}

\begin{solu}
	اولین کار برای حل این تیپ سوالات بدست آوردن 
	$\alpha, \omega$
	است. بنابراین:
	\begin{gather*}
		\alpha = \frac{1}{2RC} = \frac{1}{2\times2\times\frac{1}{4}} = 1 \\
		\omega_0 = \frac{1}{\sqrt{1\times\frac{1}{4}}} = 2
	\end{gather*}
	از آنجایی که 
	$\alpha < \omega_0$
	است پاسخ زیرمیراست. پس:
	\begin{gather*}
		V(t) = e^{-\alpha t}\Big(k_1 cos\omega_n t+ k_2sin \omega_n t\Big) \\
		\omega_n = \sqrt{\omega_0^2 - \alpha^2} = \sqrt{4-1} = \sqrt{3}\\
		\Rightarrow V(t) = e^{-\alpha t}\Big(k_1 cos\sqrt{3}t+ k_2sin \sqrt{3} t\Big) \\
	\end{gather*}
	حال باید مقادیر 
	$k_1, k_2$
	را محاسبه کنیم:
	\begin{gather*}
		V(0) = 1\Big(k_1 cos 0+ k_2 sin 0\Big) = k_1 = 2\\
	\end{gather*}
	و برای بدست آوردن 
	$k_2$
	مدار را در زمان صفر در نظر می‌گیریم یعنی به صورت زیر:
	\begin{center}
		\begin{circuitikz}
			\draw (0,0) to[capacitor,l=$\frac{1}{4}F$] (0,2);
			\draw (0,2) to[short, i<=$c\frac{dv}{dt}$] (2,2);
			\draw (2,2) to[short,i=$\frac{V}{R}$] (4,2);
			\draw (0,0) to[short] (4,0);
			\draw (2,0) to[L,l=$1H$, i<=$-2A$] (2,2);
			\draw (4,0) to[R,l=$2\Omega$] (4,2) node[yshift=-0.25cm,xshift=0.75cm](){+} node[yshift=-1.75cm, xshift=0.75cm]{-}
			node[yshift = -1cm, xshift=0.75cm]{$V(t)$};
		\end{circuitikz}
	\end{center}
	\begin{gather*}
		\frac{dv}{dt} = -e^{-t}\Big(k_1 cos\sqrt{3}t+ k_2sin \sqrt{3}t\Big) 
		+ e^{-t}\Big(-\sqrt{3}k_1 sin\sqrt{3}t+ \sqrt{3}k_2 cos \sqrt{3}t\Big) \\
		i_c = C\frac{dv}{dt} = \frac{1}{4}\Big[ -1(2+0) + 1(0+\sqrt{3}k_2) \Big] = -\frac{1}{2} + \frac{\sqrt{3}}{4}k_2
	\end{gather*}
	با توجه به قانون KCL داریم:
	\begin{gather*}
		i_R + i_L + i_C = 0 \\
		i_R = \frac{V}{R} = \frac{2}{2} = 1 \\
		1 -2 -\frac{1}{2} + \frac{\sqrt{3}}{4}k_2 = 0 \rightarrow k_2 = 3.47 \\
		\Rightarrow V(t) = e^{-t}\Big(2 cos\sqrt{3}t+ 3.47 sin \sqrt{3} t\Big) 
	\end{gather*}
\end{solu}
\subsubsection*{پاسخ طبیعی مدار RLC سری:}
همانطور که در شکل زیر هم مشاهده می‌کنید در این مدارها در هر سه المان یک جریان برقرار است.
بنابراین در این مدارها به دنبال به دست آوردن \textit{جریان} هستیم.
\begin{center}
	\begin{circuitikz}
		\draw (0,0) to[L,l_=$L$] (3,0)
		node[yshift=0.5cm,xshift = -1cm]{-}
		node[yshift = 0.5cm, xshift = -2cm]{+}
		node[yshift = 0.5cm, xshift=-1.5cm]{$V_L$}
		;
		\draw (0,0) to[R,l=$R$, i<=$i$] (0,-3)
		node[yshift = 1cm, xshift = -0.5cm]{+}
		node[yshift = 2cm, xshift = -0.5cm]{-}
		node[yshift = 1.5cm, xshift = -0.5cm]{$V_R$}
		;
		\draw (3,0) to[capacitor,l_=$C$] (3,-3)
		node[yshift = 1cm, xshift = 0.75cm]{-}
		node[yshift = 2cm, xshift = 0.75cm]{+}
		node[yshift = 1.5cm, xshift = 0.75cm]{$V_C$}
		;
		\draw (0,-3) to[short] (3,-3);
		\draw[<-,blue] (1.5,-1.5) arc (30:330:0.25);
	\end{circuitikz}
\end{center}
فرض می‌کنیم که انرژی اولیه ذخیره شده در سلف 
$I_0$
و انرژی اولیه ذخیره شده در خازن هم 
$V_0$
باشد. از طرفی با توجه به جریان مشخص شده و قانون 
KVL
داریم:
\begin{gather*}
	V_L + V_C + V_R = 0 \Rightarrow L\frac{di}{dt} + \frac{1}{C}\int i\ dt + V_0 + Ri = 0 \\
	L\frac{d^2i}{dt^2} + \frac{1}{C}i + R\frac{di}{dt} = 0 \\
	\frac{d^2i}{dt^2} + \frac{R}{L}\frac{di}{dt} + \frac{1}{LC}i = 0 \\
	\Rightarrow s^2 + \frac{R}{L}s + \frac{1}{LC} = 0 \\
	\begin{cases}
		s_1 = \frac{-R}{2L} + \sqrt{(\frac{R}{2L})^2-\frac{1}{LC}} \\
		s_2 = \frac{-R}{2L} - \sqrt{(\frac{R}{2L})^2-\frac{1}{LC}}
	\end{cases}
	\rightarrow 
	\begin{cases}
		\alpha = \frac{R}{2L} \rightarrow \text{\rl{ضریب میرایی پاسخ طبیعی}}\\
		\omega_0 = \frac{1}{\sqrt{LC}} \rightarrow \text{\rl{فرکانس تشدید}}
	\end{cases}
\end{gather*}
به مانند قبل با توجه به 
$\alpha, \omega_0$
سه جواب متمایز خواهیم داشت. 
\begin{description}
	\item[$\alpha > \omega_0$:] 
	در این حالت دو ریشه حقیقی و متمایز داریم و به آن پاسخ طبیعی فوق میرا گفته می‌شود. 
	\begin{gather*}
		i(t) = k_1e^{s_1t}+k_2e^{s_2t}
	\end{gather*}

	\item[$\alpha < \omega_0$:] 
	در این حالت دو ریشه مختلط و مزدوج داریم و به آن پاسخ طبیعی زیرمیرا گفته می‌شود. 
	\begin{gather*}
		i(t) = e^{-\alpha t}\Big(k_1 cos\omega_n t+ k_2sin \omega_n t\Big)
	\end{gather*}
	\begin{remark}
		به $\omega_n$ فرکانس تشدید طبیعی گفته می‌شود که از رابطه‌ی زیر محاسبه می‌شود:
		\begin{gather*}
			\omega_n = \sqrt{\omega_0^2 - \alpha^2}
		\end{gather*}
	\end{remark}

	\item[$\alpha = \omega_0$:] 
	در این حالت هر دو ریشه حقیقی و مساوی است و به آن پاسخ طبیعی میرای بحرانی گفته می‌شود.
	\begin{gather*}
		i(t) = k_1te^{-\alpha t} + k_2e^{-\alpha t}
	\end{gather*}
\end{description}

\begin{example}
	در شکل زیر اگر 
	$R=1\Omega, L = 1H, C=\frac{1}{25} F, i_L(0) = 2A, V_C(0) = 2V$
	مقدار 
	$i(t)$
	را برای 
	$t>0$
	بیابید.

	\begin{center}
		\begin{circuitikz}
			\draw (0,0) to[L,l_=$L$] (3,0)
			node[yshift=0.5cm,xshift = -1cm]{-}
			node[yshift = 0.5cm, xshift = -2cm]{+}
			node[yshift = 0.5cm, xshift=-1.5cm]{$V_L$}
			;
			\draw (0,0) to[R,l=$R$, i<=$i$] (0,-3)
			node[yshift = 1cm, xshift = -0.5cm]{+}
			node[yshift = 2cm, xshift = -0.5cm]{-}
			node[yshift = 1.5cm, xshift = -0.5cm]{$V_R$}
			;
			\draw (3,0) to[capacitor,l_=$C$] (3,-3)
			node[yshift = 1cm, xshift = 0.75cm]{-}
			node[yshift = 2cm, xshift = 0.75cm]{+}
			node[yshift = 1.5cm, xshift = 0.75cm]{$V_C$}
			;
			\draw (0,-3) to[short] (3,-3);
		\end{circuitikz}
	\end{center}
\end{example}

\begin{solu}
	اولین گام برای حل مسائل 
	RLC 
	بدست آوردن مقادیر
	$\alpha, \omega_0$
	است. پس:
	\begin{gather*}
		\alpha = \frac{R}{2L} = \frac{6}{2\times1} = 3 \\
		\omega_0 = \frac{1}{\sqrt{LC}} = \frac{1}{\sqrt{1\times\frac{1}{25}}} = 5
	\end{gather*}
	با توجه به اینکه
	$\alpha < \omega_0$
	است نوع پاسخ زیرمیراست بنابراین داریم:
	\begin{gather*}
		i(t) = e^{-\alpha t}\Big(k_1 cos\omega_n t+ k_2sin \omega_n t\Big) \\
		\omega_n = \sqrt{\omega_0^2 - \alpha^2} = \sqrt{25-9} = 4 \\
		i(0) = 2A \rightarrow 2 = k_1 \\
		KVL \rightarrow V_R + V_L + V_C = 0 \rightarrow R\times i(0) + L\frac{di}{dt} + V_c(0) = 0 \\
		\rightarrow k_2 = -2 \\
		\Rightarrow i(t) = e^{-2t}\Big(2 cos4t - 2sin4t\Big) 
	\end{gather*}
\end{solu}

\begin{example}
	در مدار شکل زیر اگر
	$V_c(0) = -6V, i_L(0) = 4A$
	باشد،
	$V_c(t)$
	را در 
	$t>0$
	محاسبه نمایید.

	\begin{center}
		\begin{circuitikz}
			\draw (0,2) to[american voltage source,l_=$10u(t)$] (0,0);
			\draw (0,2) to[R,l=$6\Omega$] (3,2);
			\draw (3,2) to[L,l=$1H$] (6,2);
			\draw (6,2) to[capacitor,l_=$\frac{1}{25}F$] (6,0)
			node[xshift=0.75cm, yshift=1.5cm]{+}
			node[xshift=0.75cm, yshift=1cm]{$V_c$}
			node[xshift=0.75cm, yshift=0.5cm]{-}
			;
			\draw (0,0) to[short] (6,0);
		\end{circuitikz}
	\end{center}
\end{example}

\begin{solu}
	ابتدا مقادیر
	$\alpha, \omega_0$
	را محاسبه می‌کنیم:
	\begin{gather*}
		\begin{cases}
			\alpha = \frac{R}{2L} = \frac{6}{2} = 3 \\
			\omega_0 = \frac{1}{\sqrt{LC}} = \frac{1}{1\times\frac{1}{25}} = 5
		\end{cases}
	\end{gather*}
	
	با توجه به اینکه
	$\alpha < \omega_0$
	است نوع پاسخ زیرمیراست. از طرفی وقتی مدار منبع داشته باشد و مقدار آن تابع پله‌ای باشد، داریم:
	\begin{gather*}
		V_c(t) = 10 + e^{-3 t}\Big(k_1 cos4 t+ k_2sin 4 t\Big) \\
		\omega_n = \sqrt{\omega_0^2 - \alpha^2} = \sqrt{25-9} = 4 \\
		V_c(0) = -6 = 10 + k_1 \rightarrow k_1 = -16 \\
		i(t) = C\frac{dv(t)}{dt} = \frac{1}{25}\Big( 0 + (-3e^{-3t})(k_1 cos4 t+ k_2sin 4 t)+(e^{-3t})(-4k_1sin4t + 4k_2 cos 4t) \Big) \\
		\text{\rl{جریان مقاومت}} = \text{\rl{جریان خازن}}=\text{\rl{جریان سلف}} \\
		i_L(0) = 4A = \frac{1}{25}\Bigl(-3(-16) + 4k_2\Bigr) \rightarrow k_2 = 13
	\end{gather*}
\end{solu}
	\chapter{پاسخ‌های طبیعی و پله مدارهای RLC}

\begin{definition}
	\textbf{مدارهای \lr{RLC}: }مدارهایی که هم خازن دارند، هم سلف و هم مقاومت. این مدارها از نوع مرتبه دوم هستند(یعنی توصیف آن با معادلات دیفرانسیل خطی مرتبه دوم انجام می‌شود.)
\end{definition}

\noindent
\textbf{روال حل مسائل این قسمت:}

ابتدا مدارهای RLC بدون منبع را در نظر می‌گیریم و پاسخ طبیعی را بدست می‌آوریم. سپس منابع DC کلیدها یا منابع پله را به مدار می‌افزاییم و پاسخ کامل را به صورت مجموع پاسخ ویژه(واداشته) و طبیعی می‌نویسیم و مقدار ثابت‌ها را با اعمال شرایط اولیه پیدا می‌کنیم.


\subsection*{پاسخ طبیعی مدار RLC موازی:}

پاسخ طبیعی یعنی فرض می کنیم که مدار بدون منبع است و در سلف و خازن انرژی ذخیره شده است. انرژی ذخیره شده اولیه سلف را 
$I_0$
(جریان) و انرژی ذخیره شده اولیه خازن را 
$V_0$
(ولتاژ) در نظر می‌گیریم.

\begin{center}
	\begin{circuitikz}
		\draw (0,0) to[capacitor,l=$C$] (0,2);
		\draw (0,2) to[short,i<=$i_C$,-*] (2,2);
		\draw (2,2) to[short,i=$i_R$] (4,2);
		\draw (0,0) to[short] (4,0);
		\draw (2,0) to[L,i<=$i_L$,l=$L$] (2,2);
		\draw (4,0) to[R,l=$R$] (4,2);
	\end{circuitikz}
\end{center}
\begin{gather*}
	i_R + i_C + i_L = 0 \\
	\Rightarrow 
	\begin{cases}
		i_R = \frac{V}{R} \\
		i_L = \frac{1}{L} \int_{0}^{t} V dt + I_0 \\
		i_c = C\frac{dv}{dt}
	\end{cases}
\end{gather*}
\begin{gather*}
	i_R + i_C + i_L = 0 \\
	\Rightarrow \frac{V}{R} + C\frac{dV}{dt} + \frac{1}{L} \int_{0}^{t} V dt + I_0 = 0
\end{gather*}
معادلاتی که هم مشتق دارند و هم انتگرال، باید یا مشتق را حذف کنیم و یا انتگرال. برای حذف انتگرال از کل معادله مشتق می‌گیریم.

\begin{gather*}
	\frac{1}{R} \frac{dV}{dt} + C \frac{d^2V}{dt^2} + \frac{1}{L} V = 0 
\end{gather*}

\begin{gather}\label{V-formula}
	\frac{d^2V}{dt^2} + \frac{1}{RC}\frac{dV}{dt} + \frac{V}{LC} = 0 \quad \rightarrow \quad \text{\rl{معادله دیفرانسیل مرتبه دوم}}
\end{gather}
فرض می‌کنیم که 
$V=ke^{st}$
در این صورت خواهیم داشت:
\begin{gather*}
	V=ke^{st} \rightarrow \quad \frac{d'V}{dt} = kse^{st} \rightarrow \quad \frac{d^2V}{dt} = ks^2e^{st} 
\end{gather*}

حال مقدار V را در معادله دیفرانسیل 
\eqref{V-formula}
جایگذاری می‌کنیم و خواهیم داشت:
\begin{gather*}
	ks^2e^{st} + \frac{1}{RC} kse^{st} + \frac{ke^{st}}{LC} = 0 \\
	\overbrace{ke^{st}}^1\Big(s^2 + \frac{1}{RC}s + \frac{1}{LC}\Big) = 0
\end{gather*}
قسمت یک که هیچ‌گاه صفر نمی‌شود بنابراین:
\begin{gather*}
	s^2 + \frac{1}{RC}s + \frac{1}{LC} = 0
\end{gather*}
که به آن معادله مشخصه مدار گفته می‌شود زیر ریشه‌های این معادله رابطه ریاضی 
$V(t)$
را تعیین خواهد کرد. به کمک روش دلتا نیز می‌توان ریشه‌های آن را بدست آورد که ریشه‌های به صورت زیر خواهند بود:
\begin{gather*}
	\begin{cases}
		s_1 = \frac{-1}{2RC} + \sqrt{(\frac{1}{2RC})^2 - \frac{1}{LC}} \vspace{0.5cm} \\
		s_2 = \frac{-1}{2RC} - \sqrt{(\frac{1}{2RC})^2 - \frac{1}{LC}}
	\end{cases}
	V = V_1 + V_2 = k_1e^{s_1t} + k_2e^{s_2t}
\end{gather*}


حال برای ساده‌تر کردن ریشه‌ها می‌توان عبارت‌های زیر را در نظر گرفت:
\begin{gather*}
	\begin{cases}
		\alpha = \frac{1}{2RC} &\quad \rightarrow \quad \text{\rl{ضریب میرایی پاسخ طبیعی}} \\
		\omega_0 = \frac{1}{\sqrt{LC}} &\quad \rightarrow \quad \text{\rl{فرکانش تشدید پاسخ طبیعی(مدار)}}
	\end{cases}
\end{gather*}
بنابراین:
\begin{gather*}
	\begin{cases}
		s_1 = \frac{-1}{2RC} + \sqrt{(\frac{1}{2RC})^2 - \frac{1}{LC}} = -\alpha + \sqrt{\alpha^2 - \omega_0^2} \vspace{0.5cm} \\
		s_2 = \frac{-1}{2RC} - \sqrt{(\frac{1}{2RC})^2 - \frac{1}{LC}} = -\alpha - \sqrt{\alpha^2 - \omega_0^2}
	\end{cases}
\end{gather*}

حال براساس 
$\alpha$
و 
$\omega_0$
سه نوع جواب داریم:

\begin{description}
	\item[$\alpha > \omega_0$:] 
	در این حالت دو ریشه حقیقی و متمایز داریم و به آن پاسخ طبیعی فوق میرا گفته می‌شود. 
	\begin{gather*}
		V(t) = k_1e^{s_1t}+k_2e^{s_2t}
	\end{gather*}
	
	\item[$\alpha < \omega_0$:] 
	در این حالت دو ریشه مختلط و مزدوج داریم و به آن پاسخ طبیعی زیرمیرا گفته می‌شود. 
	\begin{gather*}
		V(t) = e^{-\alpha t}\Big(k_1 cos\omega_n t+ k_2sin \omega_n t\Big)
	\end{gather*}
	\begin{remark}
		به $\omega_n$ فرکانس تشدید طبیعی گفته می‌شود که از رابطه‌ی زیر محاسبه می‌شود:
		\begin{gather*}
			\omega_n = \sqrt{\omega_0^2 - \alpha^2}
		\end{gather*}
	\end{remark}
	
	\item[$\alpha = \omega_0$:] 
	در این حالت هر دو ریشه حقیقی و مساوی است و به آن پاسخ طبیعی میرای بحرانی گفته می‌شود.
	\begin{gather*}
		V(t) = k_1te^{-\alpha t} + k_2e^{-\alpha t}
	\end{gather*}
\end{description}

\begin{example}
	پاسخ طبیعی مدار زیر را بدست آورید.
	\begin{center}
		\begin{circuitikz}
			\draw (0,0) to[capacitor,l=$\frac{1}{16}F$] (0,2);
			\draw (0,2) to[short] (2,2);
			\draw (2,2) to[short] (4,2);
			\draw (0,0) to[short] (4,0);
			\draw (2,0) to[L,l=$4H$] (2,2);
			\draw (4,0) to[R,l=$2\Omega$] (4,2);
		\end{circuitikz}
	\end{center}
\end{example}

\begin{solu}
	\ \\
	\begin{gather*}
		\alpha = \frac{16}{2\times 2 } = 4 \\
		\omega_0 = \frac{1}{\sqrt{\frac{1}{4}}} = 2
	\end{gather*}
	با توجه به حالات بالا پاسخ طبیعی فوق میراست. لذا:
	\begin{gather*}
		V(t) = k_1e^{s_1t}+k_2e^{s_2t} \\
		s_1 = -4 + \sqrt{16-4} = -0.54 \\
		s_2 = -4 - \sqrt{16-4} = -7.46 \\
		V(t) = k_1e^{-0.54t}+k_2e^{-7.46t} \\
	\end{gather*}
\end{solu}

\begin{example}
	در مدار شکل زیر اگر 
	$V_c(0) = 2V$
	و
	$i_L(0) = -2A$
	باشد، مطلوب است مقدار 
	$V(t)$
	برای
	$t>0$.
	
	\begin{center}
		\begin{circuitikz}
			\draw (0,0) to[capacitor,l=$\frac{1}{4}F$] (0,2);
			\draw (0,2) to[short] (2,2);
			\draw (2,2) to[short] (4,2);
			\draw (0,0) to[short] (4,0);
			\draw (2,0) to[L,l=$1H$] (2,2);
			\draw (4,0) to[R,l=$2\Omega$] (4,2) node[yshift=-0.25cm,xshift=0.75cm](){+} node[yshift=-1.75cm, xshift=0.75cm]{-}
			node[yshift = -1cm, xshift=0.75cm]{$V(t)$};
		\end{circuitikz}
	\end{center}
	
\end{example}

\begin{solu}
	اولین کار برای حل این تیپ سوالات بدست آوردن 
	$\alpha, \omega$
	است. بنابراین:
	\begin{gather*}
		\alpha = \frac{1}{2RC} = \frac{1}{2\times2\times\frac{1}{4}} = 1 \\
		\omega_0 = \frac{1}{\sqrt{1\times\frac{1}{4}}} = 2
	\end{gather*}
	از آنجایی که 
	$\alpha < \omega_0$
	است پاسخ زیرمیراست. پس:
	\begin{gather*}
		V(t) = e^{-\alpha t}\Big(k_1 cos\omega_n t+ k_2sin \omega_n t\Big) \\
		\omega_n = \sqrt{\omega_0^2 - \alpha^2} = \sqrt{4-1} = \sqrt{3}\\
		\Rightarrow V(t) = e^{-\alpha t}\Big(k_1 cos\sqrt{3}t+ k_2sin \sqrt{3} t\Big) \\
	\end{gather*}
	حال باید مقادیر 
	$k_1, k_2$
	را محاسبه کنیم:
	\begin{gather*}
		V(0) = 1\Big(k_1 cos 0+ k_2 sin 0\Big) = k_1 = 2\\
	\end{gather*}
	و برای بدست آوردن 
	$k_2$
	مدار را در زمان صفر در نظر می‌گیریم یعنی به صورت زیر:
	\begin{center}
		\begin{circuitikz}
			\draw (0,0) to[capacitor,l=$\frac{1}{4}F$] (0,2);
			\draw (0,2) to[short, i<=$c\frac{dv}{dt}$] (2,2);
			\draw (2,2) to[short,i=$\frac{V}{R}$] (4,2);
			\draw (0,0) to[short] (4,0);
			\draw (2,0) to[L,l=$1H$, i<=$-2A$] (2,2);
			\draw (4,0) to[R,l=$2\Omega$] (4,2) node[yshift=-0.25cm,xshift=0.75cm](){+} node[yshift=-1.75cm, xshift=0.75cm]{-}
			node[yshift = -1cm, xshift=0.75cm]{$V(t)$};
		\end{circuitikz}
	\end{center}
	\begin{gather*}
		\frac{dv}{dt} = -e^{-t}\Big(k_1 cos\sqrt{3}t+ k_2sin \sqrt{3}t\Big) 
		+ e^{-t}\Big(-\sqrt{3}k_1 sin\sqrt{3}t+ \sqrt{3}k_2 cos \sqrt{3}t\Big) \\
		i_c = C\frac{dv}{dt} = \frac{1}{4}\Big[ -1(2+0) + 1(0+\sqrt{3}k_2) \Big] = -\frac{1}{2} + \frac{\sqrt{3}}{4}k_2
	\end{gather*}
	با توجه به قانون KCL داریم:
	\begin{gather*}
		i_R + i_L + i_C = 0 \\
		i_R = \frac{V}{R} = \frac{2}{2} = 1 \\
		1 -2 -\frac{1}{2} + \frac{\sqrt{3}}{4}k_2 = 0 \rightarrow k_2 = 3.47 \\
		\Rightarrow V(t) = e^{-t}\Big(2 cos\sqrt{3}t+ 3.47 sin \sqrt{3} t\Big) 
	\end{gather*}
\end{solu}
\subsection*{پاسخ طبیعی مدار RLC سری:}
همانطور که در شکل زیر هم مشاهده می‌کنید در این مدارها در هر سه المان یک جریان برقرار است.
بنابراین در این مدارها به دنبال به دست آوردن \textit{جریان} هستیم.
\begin{center}
	\begin{circuitikz}
		\draw (0,0) to[L,l_=$L$] (3,0)
		node[yshift=0.5cm,xshift = -1cm]{-}
		node[yshift = 0.5cm, xshift = -2cm]{+}
		node[yshift = 0.5cm, xshift=-1.5cm]{$V_L$}
		;
		\draw (0,0) to[R,l=$R$, i<=$i$] (0,-3)
		node[yshift = 1cm, xshift = -0.5cm]{+}
		node[yshift = 2cm, xshift = -0.5cm]{-}
		node[yshift = 1.5cm, xshift = -0.5cm]{$V_R$}
		;
		\draw (3,0) to[capacitor,l_=$C$] (3,-3)
		node[yshift = 1cm, xshift = 0.75cm]{-}
		node[yshift = 2cm, xshift = 0.75cm]{+}
		node[yshift = 1.5cm, xshift = 0.75cm]{$V_C$}
		;
		\draw (0,-3) to[short] (3,-3);
		\draw[<-,blue] (1.5,-1.5) arc (30:330:0.25);
	\end{circuitikz}
\end{center}
فرض می‌کنیم که انرژی اولیه ذخیره شده در سلف 
$I_0$
و انرژی اولیه ذخیره شده در خازن هم 
$V_0$
باشد. از طرفی با توجه به جریان مشخص شده و قانون 
KVL
داریم:
\begin{gather*}
	V_L + V_C + V_R = 0 \Rightarrow L\frac{di}{dt} + \frac{1}{C}\int i\ dt + V_0 + Ri = 0 \\
	L\frac{d^2i}{dt^2} + \frac{1}{C}i + R\frac{di}{dt} = 0 \\
	\frac{d^2i}{dt^2} + \frac{R}{L}\frac{di}{dt} + \frac{1}{LC}i = 0 \\
	\Rightarrow s^2 + \frac{R}{L}s + \frac{1}{LC} = 0 \\
	\begin{cases}
		s_1 = \frac{-R}{2L} + \sqrt{(\frac{R}{2L})^2-\frac{1}{LC}} \\
		s_2 = \frac{-R}{2L} - \sqrt{(\frac{R}{2L})^2-\frac{1}{LC}}
	\end{cases}
	\rightarrow 
	\begin{cases}
		\alpha = \frac{R}{2L} \rightarrow \text{\rl{ضریب میرایی پاسخ طبیعی}}\\
		\omega_0 = \frac{1}{\sqrt{LC}} \rightarrow \text{\rl{فرکانس تشدید}}
	\end{cases}
\end{gather*}
به مانند قبل با توجه به 
$\alpha, \omega_0$
سه جواب متمایز خواهیم داشت. 
\begin{description}
	\item[$\alpha > \omega_0$:] 
	در این حالت دو ریشه حقیقی و متمایز داریم و به آن پاسخ طبیعی فوق میرا گفته می‌شود. 
	\begin{gather*}
		i(t) = k_1e^{s_1t}+k_2e^{s_2t}
	\end{gather*}
	
	\item[$\alpha < \omega_0$:] 
	در این حالت دو ریشه مختلط و مزدوج داریم و به آن پاسخ طبیعی زیرمیرا گفته می‌شود. 
	\begin{gather*}
		i(t) = e^{-\alpha t}\Big(k_1 cos\omega_n t+ k_2sin \omega_n t\Big)
	\end{gather*}
	\begin{remark}
		به $\omega_n$ فرکانس تشدید طبیعی گفته می‌شود که از رابطه‌ی زیر محاسبه می‌شود:
		\begin{gather*}
			\omega_n = \sqrt{\omega_0^2 - \alpha^2}
		\end{gather*}
	\end{remark}
	
	\item[$\alpha = \omega_0$:] 
	در این حالت هر دو ریشه حقیقی و مساوی است و به آن پاسخ طبیعی میرای بحرانی گفته می‌شود.
	\begin{gather*}
		i(t) = k_1te^{-\alpha t} + k_2e^{-\alpha t}
	\end{gather*}
\end{description}

\begin{example}
	در شکل زیر اگر 
	$R=1\Omega, L = 1H, C=\frac{1}{25} F, i_L(0) = 2A, V_C(0) = 2V$
	مقدار 
	$i(t)$
	را برای 
	$t>0$
	بیابید.
	
	\begin{center}
		\begin{circuitikz}
			\draw (0,0) to[L,l_=$L$] (3,0)
			node[yshift=0.5cm,xshift = -1cm]{-}
			node[yshift = 0.5cm, xshift = -2cm]{+}
			node[yshift = 0.5cm, xshift=-1.5cm]{$V_L$}
			;
			\draw (0,0) to[R,l=$R$, i<=$i$] (0,-3)
			node[yshift = 1cm, xshift = -0.5cm]{+}
			node[yshift = 2cm, xshift = -0.5cm]{-}
			node[yshift = 1.5cm, xshift = -0.5cm]{$V_R$}
			;
			\draw (3,0) to[capacitor,l_=$C$] (3,-3)
			node[yshift = 1cm, xshift = 0.75cm]{-}
			node[yshift = 2cm, xshift = 0.75cm]{+}
			node[yshift = 1.5cm, xshift = 0.75cm]{$V_C$}
			;
			\draw (0,-3) to[short] (3,-3);
		\end{circuitikz}
	\end{center}
\end{example}

\begin{solu}
	اولین گام برای حل مسائل 
	RLC 
	بدست آوردن مقادیر
	$\alpha, \omega_0$
	است. پس:
	\begin{gather*}
		\alpha = \frac{R}{2L} = \frac{6}{2\times1} = 3 \\
		\omega_0 = \frac{1}{\sqrt{LC}} = \frac{1}{\sqrt{1\times\frac{1}{25}}} = 5
	\end{gather*}
	با توجه به اینکه
	$\alpha < \omega_0$
	است نوع پاسخ زیرمیراست بنابراین داریم:
	\begin{gather*}
		i(t) = e^{-\alpha t}\Big(k_1 cos\omega_n t+ k_2sin \omega_n t\Big) \\
		\omega_n = \sqrt{\omega_0^2 - \alpha^2} = \sqrt{25-9} = 4 \\
		i(0) = 2A \rightarrow 2 = k_1 \\
		KVL \rightarrow V_R + V_L + V_C = 0 \rightarrow R\times i(0) + L\frac{di}{dt} + V_c(0) = 0 \\
		\rightarrow k_2 = -2 \\
		\Rightarrow i(t) = e^{-2t}\Big(2 cos4t - 2sin4t\Big) 
	\end{gather*}
\end{solu}

\begin{example}
	در مدار شکل زیر اگر
	$V_c(0) = -6V, i_L(0) = 4A$
	باشد،
	$V_c(t)$
	را در 
	$t>0$
	محاسبه نمایید.
	
	\begin{center}
		\begin{circuitikz}
			\draw (0,2) to[american voltage source,l_=$10u(t)$] (0,0);
			\draw (0,2) to[R,l=$6\Omega$] (3,2);
			\draw (3,2) to[L,l=$1H$] (6,2);
			\draw (6,2) to[capacitor,l_=$\frac{1}{25}F$] (6,0)
			node[xshift=0.75cm, yshift=1.5cm]{+}
			node[xshift=0.75cm, yshift=1cm]{$V_c$}
			node[xshift=0.75cm, yshift=0.5cm]{-}
			;
			\draw (0,0) to[short] (6,0);
		\end{circuitikz}
	\end{center}
\end{example}

\begin{solu}
	ابتدا مقادیر
	$\alpha, \omega_0$
	را محاسبه می‌کنیم:
	\begin{gather*}
		\begin{cases}
			\alpha = \frac{R}{2L} = \frac{6}{2} = 3 \\
			\omega_0 = \frac{1}{\sqrt{LC}} = \frac{1}{1\times\frac{1}{25}} = 5
		\end{cases}
	\end{gather*}
	
	با توجه به اینکه
	$\alpha < \omega_0$
	است نوع پاسخ زیرمیراست. از طرفی وقتی مدار منبع داشته باشد و مقدار آن تابع پله‌ای باشد، داریم:
	\begin{gather*}
		V_c(t) = 10 + e^{-3 t}\Big(k_1 cos4 t+ k_2sin 4 t\Big) \\
		\omega_n = \sqrt{\omega_0^2 - \alpha^2} = \sqrt{25-9} = 4 \\
		V_c(0) = -6 = 10 + k_1 \rightarrow k_1 = -16 \\
		i(t) = C\frac{dv(t)}{dt} = \frac{1}{25}\Big( 0 + (-3e^{-3t})(k_1 cos4 t+ k_2sin 4 t)+(e^{-3t})(-4k_1sin4t + 4k_2 cos 4t) \Big) \\
		\text{\rl{جریان مقاومت}} = \text{\rl{جریان خازن}}=\text{\rl{جریان سلف}} \\
		i_L(0) = 4A = \frac{1}{25}\Bigl(-3(-16) + 4k_2\Bigr) \rightarrow k_2 = 13
	\end{gather*}
\end{solu}



















	\include{section07}
	\chapter{توان در حالت ماندگار سینوسی}

دو توان کلی داریم:

\section{توان لحظه‌ای}
توان لحظه‌ای برای هر عنصر الکتریکی یا الکترونیکی برابر با حاصلضرب ولتاژ لحظه‌ای در جریان لحظه‌ای گذرنده از آن. برای مثال در مدار ساده‌ی زیر اگر جریان عبوری از آن i(t)  باشد داریم:

\begin{minipage}{0.4\textwidth}
	\begin{center}
		\begin{circuitikz}
			\draw (0,2) to[american voltage source] (0,0);
			\draw (2,2) to[generic, l=$Z$] (2,0);
			\draw (0,2) to[short, i=$i_(t)$] (2,2);
			\draw (0,0) to[short] (2,0);
		\end{circuitikz}
	\end{center}
\end{minipage}
\begin{minipage}{0.4\textwidth}
	\begin{gather*}
		V(t) = V_m \cos (\omega t + \theta_v) \\
		I(t) = I_m \cos (\omega t + \theta_i)
	\end{gather*}
\end{minipage}
\vspace{1cm}

بنابراین رابطه محاسبه توان در حالت ماندگار سینوسی:


\begin{gather}\label{eq:8-1}
	P(t) = V(t) I(t) = V_m I_m \cos (\omega t + \theta_v).\cos(\omega t + \theta_i)
\end{gather}

با استفاده از اتحاد مثلثاتی زیر می‌توان رابطه \ref{eq:8-1} را ساده تر کرد:

\begin{gather*}
	\cos \alpha \cos \beta = \frac{1}{2}\Big[\cos(\alpha + \beta) + \cos(\alpha - \beta) \Big]
\end{gather*}

\begin{gather*}
	P(t) = \frac{1}{2} V_m I_m \Big( \overbrace{\cos(\theta_v - \theta_i)}^{\text{\rl{مستقل از زمان و مقدار ثابتی دارد.}}} + \cos (2\omega t + \theta_v + \theta_i) \Big)
\end{gather*}



\begin{example}
	در مدار شکل زیر 
	$ V(t) = 12 \cos (\omega t + 52^\circ) $
	و 
	$Z = 6 \angle 27^\circ$
	توان لحظه‌ای را بدست آورید.
		\begin{center}
		\begin{circuitikz}
			\draw (0,2) to[american voltage source] (0,0);
			\draw (2,2) to[generic, l=$Z$] (2,0);
			\draw (0,2) to[short, i=$i_(t)$] (2,2);
			\draw (0,0) to[short] (2,0);
		\end{circuitikz}
	\end{center}
	
\end{example}

\begin{solu}
	برای حل باید استفاده 
	$i(t)$
	را محاسبه کرد:
	
	\begin{gather*}
		i(t) = \frac{V(t)}{Z} = \frac{12 \angle 52^\circ}{6 \angle 27^\circ} = 2(52-27) = 2\angle 25^\circ
	\end{gather*}

	\begin{gather*}
		P(t) = \frac{1}{2} V_m I_m \Big( \cos(\theta_v - \theta_i) + \cos (2\omega t + \theta_v + \theta_i) \Big) \\
			P(t) = \frac{1}{2}\times 12 \times 2 \Big( \cos(52- 25) + \cos (2\omega t + 52 + 25) \Big) \\
			= 12 \Big( \cos(27^\circ) + \cos (2\omega t + 77^\circ) \Big) \\
			= 10.7 + 12 \cos(2\omega t + 77^\circ) (w) \rightarrow \text{\rl{واحد آن وات است}}
	\end{gather*}
\end{solu}

\section{توان متوسط}

توان متوسط با استفاده از رابطه‌ی زیر محاسبه خواهد شد:
\begin{gather*}
	P_{av} = \frac{1}{T}\int_{t_0}^{t_0+T} P(t)\ dt = \frac{1}{2} V_m I_m \cos(\theta_v - \theta_i)
\end{gather*}

مفهومی به نام ضریب توان
\LTRfootnote{Power Factor}
وجود دارد که با استفاده از رابطه‌ی زیر محاسبه می‌شود:
\begin{gather*}
	PF=\cos(\theta_v - \theta_i)
\end{gather*}
حال اگر:
\begin{itemize}
	\item
	مدار کاملاً مقاومتی باشد 
	$ \theta_v - \theta_i = 0 $
	یعنی 
	$PF = 1$.
	
		\item
	مدار کاملاً القایی(مداری که فقط سلف دارد) باشد 
	$ \theta_v - \theta_i = 90^\circ $
	یعنی 
	$PF = 0$.
	
		\item
	مدار کاملاً خازنی باشد 
	$ \theta_v - \theta_i =- 90^\circ $
	یعنی 
	$PF = 0$.
\end{itemize}

\begin{remark}
	ضریب توان(PF) یکی از معیارهای هر مداری است.
\end{remark}

\section*{توان واکشی}

توان مدارهای کاملاً القایی یا کاملاً خازنی را توان واکشی می نامند و با علامت Q نمایش می‌دهند و داریم که:
\begin{gather*}
	Q = \frac{1}{2} V_m I_m \sin(\theta_v - \theta_i)
\end{gather*}

\begin{remark}
	در رابطه با توان واکشی توجه داشته باشید که:
	\begin{itemize}
		\item
		واحد اندازه گیری آن 
		\lr{VAR\LTRfootnote{Volt Amper Reactive}}
		است.
		
		\item
		به توان واکشی، توان واکنشی هم گفته می‌شود.
	\end{itemize}
\end{remark}


\begin{example}
	به مداری ولتاژ
	$ V(t) = 50 \cos (\omega t + 30^\circ) $
	اعمال و از آن جریانی معادل
	$ I(t) = 5 \sin (\omega t - 30^\circ) $
	ناشی شده است. توان متوسط و توان واکشی مدار را پیدا کنید.
\end{example}


\begin{solu}
ابتدا رابطه جریان را به یک رابطه cos تبدیل کنید:
\begin{gather*}
	 I(t) = 5 \cos(\omega t - 30^\circ - 90^\circ) = 5 \cos(\omega t - 120^\circ)
\end{gather*}
حال برای محاسبه توان متوسط داریم که:
\begin{gather*}
	P_{av} = \frac{1}{2} V_m I_m \cos(\theta_v - \theta_i)\\
	= \frac{1}{2} \times 50 \times 5 \cos(30 - (-120))=125\cos(150) = -108.25 w
\end{gather*}
و برای محاسبه توان واکشی:
\begin{gather*}
	Q = \frac{1}{2} V_m I_m \sin(\theta_v - \theta_i) \\
	\frac{1}{2} \times 50 \times 5 \sin(30 - (-120)) = 125.\sin(150^\circ) = 62.5 VAR
\end{gather*}
\end{solu}

\section{مقادیر مؤثر ولتاژ و جریان متناوب}

مقادیر مؤثر برای جریان‌های متناوب استفاده می‌شود و به آن rms نیز گفته می‌شود.

\begin{definition}[مقدار مؤثر جریان متناوب($I_{rms}$)]
	مقدار جریان مستقیمی که اگر از مقاومت R بگذرد، توانی که به آن می‌دهد با توانی که جریان متناوب به آن می‌دهد یکی باشد.
\end{definition}

\begin{gather*}
	\begin{cases}
		I = I_m \cos(\omega t +\phi)\\
		V = V_m \cos(\omega t +\phi)
	\end{cases}
	\Rightarrow I_{rms} = \frac{I_m}{\sqrt{2}}
\end{gather*}

\begin{definition}[مقدار مؤثر ولتاژ متناوب($V_{rms}$)]
مقدار ولتاژ مستقیمی که اگر از مقاومت R بگذرد، توانی که به آن می‌دهد با توانی که ولتاژ متناوب به آن می‌دهد یکی باشد.
\end{definition}

\begin{gather*}
	\begin{cases}
		I = I_m \cos(\omega t +\phi)\\
		V = V_m \cos(\omega t +\phi)
	\end{cases}
	\Rightarrow V_{rms} = \frac{V_m}{\sqrt{2}}
\end{gather*}

با استفاده از ولتاژ و جریان مؤثر می‌توان رابطه توان متوسط و توان واکنشی را بازنویسی کرد. برای محاسبه توان متوسط داریم که:

\begin{gather*}
	P_{av} =  V_{rms} I_{rms} \cos(\theta_v - \theta_i)
\end{gather*}
و برای محاسبه توان واکشی:
\begin{gather*}
	Q =  V_{rms} I_{rms} \sin(\theta_v - \theta_i) 
\end{gather*}

\section{توان ظاهری}

توان ظاهری برابر است با حاصل ضرب مقادیر مؤثر ولتاژ و جریان که با p(کوچک) نشان داده می‌شود و واحد آن ولت‌آمپر (V.A) می‌باشد.

\begin{gather*}
	p = V_{rms} I_{rms}
\end{gather*}
ضریب توان را هم می‌توان با استفاده از توان ظاهری و توان متوسط به صورت زیر بازنویسی کرد:


\begin{gather*}
	PF = \cos(\theta_v - \theta_i) = \frac{P_{av}}{V_{rms} I_{rms}}
\end{gather*}

\begin{example}
	در مدار شکل زیر توان متوسط، توان ظاهری و ضریب توان را محاسبه کنید. 
	\begin{center}
		\begin{circuitikz}
			\draw (0,2) to[american voltage source, l_=$20 \angle 0^\circ\ V_{rms}$] (0,0);
			\draw (0,2) to[generic, l=$2+5j$] (2,2);
			\draw (2,2) to[generic, l=$1-j$] (2,0);
			\draw (0,0) to[short] (2,0);
		\end{circuitikz}
	\end{center}	
\end{example}

\begin{solu}
	ابتدا جریان را محاسبه می‌کنیم:
	\begin{gather*}
		I = \frac{V}{Z} = \frac{20 \angle 0^\circ}{(2+j5)+(1-j1)} = \frac{20 \angle 0^\circ}{3+j4} = \frac{20 \angle 0^\circ}{5 \angle 53.13^\circ}\\
		=4 \angle (0-53.13) = 4 \angle (-53.13^\circ) \Rightarrow I_{rms} = 4
	\end{gather*}
	\begin{gather*}
		p = V_{rms} I_{rms} = 20 \times 4 = 80
	\end{gather*}
	\begin{gather*}
		P_{av} =  V_{rms} I_{rms} \cos(\theta_v - \theta_i) = 20 \times 4 \times \cos(+53.13^\circ) = 48w
	\end{gather*}
	\begin{gather*}
		PF = \cos(\theta_v - \theta_i) = \frac{P_{av}}{V_{rms} I_{rms}} = \frac{48}{80} = 0.6
	\end{gather*}
\end{solu}


\begin{remark}
	مقادیر مؤثر فاز ندارد و تنها یک عدد می‌باشد چون طبق تعریف یک جریان مستقیم است.
\end{remark}


\section{توان مختلط}

توان مختلط را با S نمایش می‌دهد و برابر است با:

\begin{gather*}
	S = P_{av} + jQ
\end{gather*}

واحد آن VA (ولت-آمپر) می‌باشد.

\begin{example}
	در مدار شکل زیر باری با توان متوسط $ 8kw $ با ضریب توان پس‌افتی $ 0.8 $ به کمک خطی با امپدانس
	$ 0.05+j0.2\Omega $
	  از منبع
	   $ Vـs $
	    تغذیه می‌شود. ولتاژ مؤثر بار
	 $ 220V_{rms} $
	 	 است. ولتاژ منبع $ V_s $ را تعیین کنید. 
	 	 
	 	 \begin{center}
	 	 	\begin{circuitikz}
	 	 		\draw (0,2.5) to[american voltage source,l_=$V_s$] (0,0);
	 	 		\node[rectangle,draw,xshift=4cm, yshift=1.25cm,align=center] (bar) {$P_{av} = 8kw$\\$PF = 0.8$};
	 	 		\draw (0,2.5) to[R, l=$0.05$] (2,2.5);
	 	 		\draw (2,2.5) to[L, l=$j0.2$] (4,2.5);
	 	 		\draw(4,2.5) -- (bar);
	 	 		\draw (0,0) -- (4,0) -- (bar);
	 	 		
	 	 		\draw[<->] (2.5,0.5) -- (2.5,2) node[xshift=-1cm,yshift=-0.75cm]{$220V_{rms}$};
	 	 	\end{circuitikz}
	 	 \end{center}
\end{example}

\begin{solu}
	\ \\
	\begin{gather*}
		V_{rms}. I_{rms} = \frac{8000}{0.8} \Rightarrow I_{rms} = \frac{8000}{0.8\times 220} = 45.45\ A_{rms}\\
		\cos(\theta_v - \theta_i) = 0.8 \rightarrow \theta_v - \theta_i = 36.87^\circ \Rightarrow 	\sin(\theta_v - \theta_i) = 0.6\\
		\text{بار} Q =  V_{rms} I_{rms} \sin(\theta_v - \theta_i) = 220 \times 45.45 \times 0.6 = 6\ kVAR
	\end{gather*}
	\begin{gather*}
		\text{بار}\rightarrow
		\begin{cases}
			P_{av} = 8kw \\
			Q = 6kVAR
		\end{cases}
	\end{gather*}
	
	\begin{remark}
		\ \\
		\begin{center}
			\begin{tikzpicture}
				\node[]{$S = P + jQ$}
				node[xshift=-2cm,yshift=-1cm]{\text{\rl{توان مربوط به مقاومت}}}
				node[xshift=2cm,yshift=-1cm]{\text{\rl{توان مربوط به سلف و خازن}}};
				\draw[->] (-0.2,-0.2) -- (-0.7,-0.7);
				\draw[->] (1,-0.2) -- (1.7,-0.7);
			\end{tikzpicture}
		\end{center}
	\end{remark}
	\begin{gather*}
		\text{خط}\ P_{av} = R(I_{rms})^2 = (0.05) (45.45)^2 = 103.28\ w \\
		\text{خط}\ Q = X(I_{rms})^2 = (0.2) (45.45)^2 = 413.14\ VAR
	\end{gather*}
	\begin{gather*}
		\text{\rl{توان کل حقیقی متوسط}} \Rightarrow P_{av_{T}} = P_{av_{\text{خط}}} + P_{av_{\text{بار}}} 
		= 8000 + 103.28 = 8103.28\ w \\
		\text{\rl{توان کل واکشی}} \Rightarrow Q_{T} = Q_{\text{خط}} + Q_{\text{بار}} 
		= 6000 + 413.14 = 6413.14\ VAR \\ 
		\text{\rl{توان کل مختلط}} \Rightarrow S_{T} = P_{av_{T}} + jQ_{T}
		= 8103.28 + j6413.14 = 10333.99 \angle 38.35^\circ\ VAR
	\end{gather*}
	\begin{gather*}
		|s| = V_{s_{rms}} \times I_{rms} \rightarrow V_{s_{rms}} = \frac{|s|}{I_{rms}} \\
		V_{s_{rms}} = \frac{10333.99}{45.45} = 227.37\ rms
	\end{gather*}
\end{solu}


\section{انتقال ماکزیمم توان}

منظور آن انتقال ماکزیمم توان متوسط به بار است. 

زمانی انتقال ماکزیمم توان از مدار بر بار اتفاق می‌افتد که مدار فقط مقاومتی باشد، یعنی اگر بار به صورت 
$Z_L = R_L +jX_L$
و امپدانس دیده شده مدار توسط بار به صورت 
$Z = R + jX$
باشد حالت ماکزیمم توان زمانی اتفاق می‌افتد که 
$X_L = -X$
یعنی:

\begin{flushleft}
	$\text{\rl{کل مدار}}Z_t = R + R_L$
\end{flushleft}

\begin{center}
	\begin{minipage}{0.3\textwidth}
		\begin{center}
			\begin{circuitikz}
				\draw(0,2) to[american voltage source,l_=$V_{th}$] (0,0);
				\draw(0,2) to[generic,l=$Z_{th}$, i=$I_m$] (3,2);
				\draw(3,2) to[generic,l=$Z_L$] (3,0);
				\draw(0,0) to[short] (3,0);
			\end{circuitikz}
		\end{center}
	\end{minipage}
	\begin{minipage}{0.1\textwidth}
		\begin{center}
			$ \equiv $
		\end{center}
	\end{minipage}
	\begin{minipage}{0.3\textwidth}
		\begin{center}
			\begin{circuitikz}
				\node[rectangle,draw, minimum height=3cm, minimum width=2cm]{\text{مدار}};
				\draw (2,-1) to[generic,l_=$Z_L$] (2,1);
				\draw (1,1) to[short,i=$I_m$] (2,1);
				\draw (1,-1) to[short] (2,-1);
				\draw[->] (1.5,-1.5) -- (1.5,-0.5) -- (1.25,-0.5)
				node[xshift=0.5cm,yshift=-1cm]{$ th $};
			\end{circuitikz}
		\end{center}
	\end{minipage}
\end{center}

\begin{center}
	\begin{tabular}{lc}
		$ P_{av} = \frac{1}{2}R_L\ |I_m|^2 $ & \text{\rl{ماکزیمم توان متوسط تحویلی بار}} \\
		$ I_m = \frac{V_{th}}{2R_L} $ \\
		\fbox{$ Z_L = Z_{th}^\star $} & \tikz \draw[->] (0,1) -- (0,0) -- (1,0);
	\end{tabular}
\end{center}



\begin{example}
	در مدار شکل زیر،
	$ Z_L $
	چقدر باید باشد تا بیشترین توان متوسط را جذب کند؟ مقدار ماکزیمم توان متوسط بار را محاسبه کنید.
	
	\begin{center}
		\begin{circuitikz}
			\draw (0,2) to[american voltage source, l_=$5\angle 0^\circ$] (0,0);
			\draw (2,2) to[R, l=$4\Omega$] (2,0);
			\draw (4,2) to[generic, l=$Z_L$] (4,0);
			\draw (0,2) to[L, l=$j3\Omega$] (2,2);
			\draw (0,0) to[short] (4,0);
			\draw (2,2) to[short] (4,2);
		\end{circuitikz}
	\end{center}

\end{example}

\begin{solu}
	اولین گام: بدست آوردن معادل تونن دیده شده از دو سر بار
	
	\begin{gather*}
		Z_{th} = 4\ ||\ j3 = \frac{j12}{4+j3} = \frac{36+j48}{25} = 1.44 + j1.92 \\
		V_{th} = \frac{4}{4+j3}\times(5\angle 0^\circ) = \frac{20}{4+j3} = \frac{16-j12}{5} = 4\ \angle \ -36.86^\circ
	\end{gather*}

	\begin{center}
		\begin{circuitikz}
			\draw (0,2) to[american voltage source, l_=$4\angle -36.86^\circ$] (0,0);
			\draw (2,2) to[generic, l=$Z_L$] (2,0);
			\draw (0,2) to[generic, l=$Z_{th}$] (2,2);
			\draw (0,0) to[short] (2,0);
		\end{circuitikz}
	\end{center}
	\begin{gather*}
		Z_L = Z_{th}^* = 1.44-j1.92 \\
		I_m = \frac{V_{th}}{Z_{th} + Z_L} = \frac{4\ \angle\  -36.86^\circ}{2.88} = 1.73\ \angle \ -36.86^\circ \\
		\Rightarrow P_{av} = \frac{1}{2}R_L I_m^2 = \frac{1}{2}(1.44)(1.73)^2 = 2.15\ w
	\end{gather*}
\end{solu}























\end{document}
