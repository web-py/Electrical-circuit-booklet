\section{القاگر(سلف) و خازن}
القاگر و خازن از عناصر ذخیره‌ای مدار هستند؛ یعنی می‌توانند انرژی محدودی را ذخیره کنند و در موقع لزوم به مدار برگردانند. جرقه سر شمع موتور خودرو نمونه‌ای از ذخیره‌ی انرژی در القاگر و جرقه لازم برای روشن شدن لامپ‌های مهتابی قدیمی‌تر نمونه‌ای از ذخیره انرژی توسط خازن است.

\subsection{القاگر}
القاگر یا سلف انرژی را در میدان مغناطیسی ذخیره می‌کند. القاگر در مدار به صورت زیر نشان داده می‌شود:

\begin{center}
	\begin{circuitikz}
		\draw (0,0) to[L,i=$I$,l=$L$] (2,0);
		\node[yshift=-2cm,xshift=1cm] () {$L=\frac{Q}{I}$};
		\draw[->] (0,-2) -- (-1,-2) node[xshift=-2.25cm]{\text{\rl{ضریب خودالقایی(القاکنایی)}}};
		\draw[->] (0,-2) -- (-1,-3) node[xshift=-1.25cm]{\text{\rl{واحد هانری(H)}}};
		\draw[->] (1.65,-1.75) -- (3,-1) node[xshift=2.25cm,yshift=0.25cm]{\text{\rl{شار مغناطیسی القا شده در القاگر}}};
		\draw[->] (1.35,-1.65) -- (1,-1) node[xshift=-0.5cm,yshift=0.25cm]{\text{\rl{واحد وبر(wb)}}};
		\draw[->] (1.65,-2.25) -- (3,-3) node[xshift=0.5cm,yshift=-0.25cm]{\text{\rl{واحد آمپر(A)}}};
		\draw[->] (1.65,-2.25) -- (3,-2.25) node[xshift=2.5cm]{\text{\rl{جریان الکتریکی عبوری از سلف}}};
	\end{circuitikz}
\end{center}

\subsubsection*{فرمول‌های مربوط به القاگر}
\begin{gather*}
	\text{\rl{رابطه ولتاژ و جریان}}\qquad  V = L \frac{dI}{dt} \Rightarrow \quad I = \frac{1}{L} \int_{0}^{t} Vdt + I(0)
\end{gather*}
\begin{remark}
	از آنجایی که القاگر وابسته به ارتباط جریان با گذر زمان است بنابراین هم می‌تواند خطی باشد و هم غیرخطی.
\end{remark}
\begin{gather*}
	\text{\rl{توان}}\qquad  P = VI = LI \frac{dI}{dt}  
\end{gather*}
\begin{gather*}
	\text{\rl{انرژی ذخیره شده}}\qquad  W = \int p dt = \frac{1}{2}LI^2
\end{gather*}

\subsubsection*{القاگر متقابل(M)}
القاگر متقابل پارامتری است برای مرتبط ساختن ولتاژ القایی در یک مدار.

\begin{minipage}{0.5\textwidth}
	\begin{gather*}
		\begin{cases}
			V_1 = L_1\frac{dI_1}{dt} + M\frac{dI_2}{dt} \vspace{0.5cm}\\
			V_2 = L_2\frac{dI_2}{dt} + M\frac{dI_2}{dt}
		\end{cases}
	\end{gather*}
\end{minipage}
\begin{minipage}{0.5\textwidth}
	\begin{center}
		\begin{circuitikz}
			\draw (0,0) to[short,i=$I_1$] (2,0);
			\draw (2,0) to[L] (2,-2) node[red,xshift=-0.5cm,yshift=1cm]{$V_1$}
			node[red,xshift=-0.5cm,yshift=1.5cm]{$+$} node[red,xshift=-0.5cm,yshift=0.5cm]{$-$};
			\draw (0,-2) to[short] (2,-2);
			\draw[<->,red] (3,0) arc(65:115:1.25) node[red,yshift=0.35cm,xshift=0.5cm]{M};
			\draw (3,0) to[short,i<=$I_2$] (5,0);
			\draw (3,0) to[L] (3,-2) node[red,xshift=0.5cm,yshift=1cm]{$V_2$}
			node[red,xshift=0.5cm,yshift=1.5cm]{$+$} node[red,xshift=0.5cm,yshift=0.5cm]{$-$};
			\draw (3,-2) to[short] (5,-2);
		\end{circuitikz}
	\end{center}
\end{minipage}

\begin{example}
	شکل موج جریان یک القاگر 
	$5mH$
	در شکل زیر داده شده است. شکل موج ولتاژ را رسم کنید.
	\begin{center}
		\includegraphics[scale=0.5]{images/sec04/sec04-fig1}
	\end{center}
\end{example}

\begin{solu}
	\ \\ 
	\begin{gather*}
		V_L = L\frac{di}{dt} \\
		0 \le t \le 2 \rightarrow i(t) = 15t \vspace{0.5cm} \\
		V_L(t) = 15 \times 5 \times 10^{-3} = 75mV \vspace{0.5cm} \\
		2 \le t \le 4 \rightarrow i(t) = -15t +30+30 \vspace{0.5cm} \\
		V_L(t) = -15 \times 5 \times 10^{-3} = -75mV \vspace{0.5cm} \\
	\end{gather*}
	
	\begin{center}
		\includegraphics[scale=0.5]{images/sec04/sec04-fig2}
	\end{center}
	
\end{solu}


\subsubsection*{ترکیب سلف‌ها}

سلف‌های سری:
\begin{center}
	\begin{circuitikz}
		\draw (-2,0) to[L, l=$L_1$] (0,0) ;
		\draw (0,0) to[L, l=$L_2$] (2,0) node[xshift=0.5cm]{$\cdots$};
		\draw (3,0) to[L, l=$L_N$] (5,0);
	\end{circuitikz}
\end{center}
\begin{gather*}
	L_{eq} = L_1 + L_2 + \cdots + L_N
\end{gather*}
سلف‌های موازی:

\begin{center}
	\begin{circuitikz}
		\draw (0,0) to[short] (1,0);
		\draw (1,-1.5) to[short] (1,1.5);
		\draw (1,1.5) to[L, l=$L_1$] (3,1.5);
		\draw (1,0.5) to[L, l=$L_2$] (3,0.5) node[xshift=-1cm,yshift=-0.65cm]{$\vdots$};
		\draw (1,-1.5) to[L, l=$L_N$] (3,-1.5);
		\draw (3,-1.5) to[short] (3,1.5);
		\draw (3,0) to[short] (4,0);
	\end{circuitikz}
\end{center}
\begin{gather*}
	\frac{1}{L_{eq}} = \frac{1}{L_1} + \frac{1}{L_2} + \cdots + \frac{1}{L_N} 
\end{gather*}

\begin{example}
	القاگر معادل شکل زیر را از دید a-b بدست آورید.
	
	\begin{center}
		\begin{circuitikz}
			\draw (0,0) to[L,l=$6H$,*-] (2,0) node[xshift=-2.25cm]{a};
			\draw (2,0) to[L,l=$2H$] (4,0);
			
			\draw (2,0) to[L,l=$6H$] (2,-2);
			\draw (2,-2) to[L,l=$4H$] (2,-4);
			
			\draw (2,-2) to[short] (4,-2);
			
			\draw (4,0) to[L,l=$1H$] (4,-2);
			\draw (4,-2) to[L,l=$8H$] (4,-4);
			
			\draw (0,-4) to[L,l=$10H$,*-] (2,-4) node[xshift=-2.25cm]{b};
			\draw (2,-4) to[L,l=$4H$] (4,-4);
		\end{circuitikz}
	\end{center}
	
\end{example}
\begin{solu}
	\ \\
	\begin{gather*}
		L_{2,1} = 2+1 = 3H \\
		L_{3,6} = \frac{3\times 6}{9} = 2H \vspace{1cm}\\
		L_{4,8} = 4+8 = 12H \\
		L_{12,4} = \frac{4\times 12}{16} = 3H\\
	\end{gather*}
	\begin{center}
		\begin{circuitikz}
			\draw (0,0) to[L,l=$6H$,*-] (2,0) node[xshift=-2.25cm]{a};
			\draw (2,0) to[short] (4,0);
			\draw[red] (4,0) to[L,l=$2H$] (4,-2);
			\draw[red] (4,-2) to[L,l=$3H$] (4,-4);
			\draw (0,-4) to[L,l=$10H$,*-] (2,-4) node[xshift=-2.25cm]{b};
			\draw (2,-4) to[short] (4,-4);
		\end{circuitikz}
	\end{center}
	\begin{gather*}
		L_{eq} = 6+2+3+10=21H
	\end{gather*}
\end{solu}



\subsection{خازن}
عنصری است که برای ذخیره‌ی انرژی در میدان الکتریکی استفاده می‌شود. خازن در مدار با نماد زیر نشان داده می‌شود.

\begin{center}
	\begin{circuitikz}
		\draw (0,0) to[capacitor, i=$I$,l=$C$] (2,0);
	\end{circuitikz}
\end{center}

\subsubsection*{فرمول‌های مربوط به خازن}

\begin{gather*}
	I=C\frac{dv}{dt} \rightarrow V = \frac{1}{C} \int_{0}^{t} I dt + V(0)
\end{gather*}

\begin{gather*}
	P=VI = CV\frac{dV}{dt}
\end{gather*}

\begin{gather*}
	W=\int Pdt = \frac{1}{2}CV^2
\end{gather*}


\begin{example}
	شکل موج ولتاژ اعمال شده به خازن 
	$6\mu F$
	به صورت زیر است. شکل موج جریان را بدست آورید.
		\begin{center}
		\includegraphics[scale=0.5]{images/sec04/sec04-fig3}
	\end{center}
\end{example}

\begin{solu}
	\begin{gather*}
		0 \le t \le 4 \Rightarrow V(t) = 7.5t \times 10^{-3} \vspace{0.5cm} \\
		i(t) = C\frac{dV}{dt} = 6\mu F \times 7.5 \times 10^{3} = 45mA \vspace{0.5cm}\\
		4 \le t \le 6 \Rightarrow V(t) = -15t \times 10^{3} +60+30 \vspace{0.5cm} \\
		i(t) = C\frac{dV}{dt} = 6\mu F \times -15 \times 10^{3} = -90mA \vspace{0.5cm}\\
	\end{gather*}
	\begin{center}
		\includegraphics[scale=0.5]{images/sec04/sec04-fig4}
	\end{center}
\end{solu}
\subsubsection*{ترکیب خازن‌ها}

خازن‌های سری:
\begin{center}
	\begin{circuitikz}
		\draw (-2,0) to[capacitor, l=$C_1$] (0,0) ;
		\draw (0,0) to[capacitor, l=$C_2$] (2,0) node[xshift=0.5cm]{$\cdots$};
		\draw (3,0) to[capacitor, l=$C_N$] (5,0);
	\end{circuitikz}
\end{center}
\begin{gather*}
	\frac{1}{C_{eq}} = \frac{1}{C_1} + \frac{1}{C_2} + \cdots + \frac{1}{C_N} 
\end{gather*}

خازن‌های موازی:

\begin{center}
	\begin{circuitikz}
		\draw (0,0) to[short] (1,0);
		\draw (1,-1.5) to[short] (1,2);
		\draw (1,2) to[capacitor, l=$C_1$] (3,2);
		\draw (1,0.5) to[capacitor, l=$C_2$] (3,0.5) node[xshift=-1cm,yshift=-0.65cm]{$\vdots$};
		\draw (1,-1.5) to[capacitor, l=$C_N$] (3,-1.5);
		\draw (3,-1.5) to[short] (3,2);
		\draw (3,0) to[short] (4,0);
	\end{circuitikz}
\end{center}
\begin{gather*}
	C_{eq} = C_1 + C_2 + \cdots + C_N
\end{gather*}

\begin{example}
	خازن معادل شکل زیر را از دید a-b بدست آورید.
	
	\begin{center}
		\begin{circuitikz}
			\draw (0,0) to[capacitor,l=$6$,*-] (2,0) node[xshift=-2.25cm]{a};
			\draw (2,0) to[capacitor,l=$2$] (4,0);
			
			\draw (2,0) to[capacitor,l_=$6$] (2,-2);
			\draw (2,-2) to[capacitor,l_=$4$] (2,-4);
			
			\draw (2,-2) to[short] (4,-2);
			
			\draw (4,0) to[capacitor,l=$1$] (4,-2);
			\draw (4,-2) to[capacitor,l=$8$] (4,-4);
			
			\draw (0,-4) to[capacitor,l_=$10$,*-] (2,-4) node[xshift=-2.25cm]{b};
			\draw (2,-4) to[capacitor,l_=$4$] (4,-4);
		\end{circuitikz}
	\end{center}
	
\end{example}
\begin{solu}
	\ \\
	\begin{gather*}
		C_{2,1} = \frac{1\times 2}{1+2} = \frac{2}{3}\\
		C_{\frac{2}{3},6} = \frac{2}{3} + 6 = \frac{20}{3} \vspace{1cm}\\
			C_{4,8} = \frac{4\times 8}{12} = \frac{8}{3}\\
		C_{\frac{8}{3},4} = \frac{8}{3} + 4 = \frac{20}{3} \vspace{1cm}\\
	\end{gather*}
	\begin{center}
		\begin{circuitikz}
			\draw (0,0) to[capacitor,l=$6$,*-] (2,0) node[xshift=-2.25cm]{a};
			\draw (2,0) to[short] (4,0);
			\draw[red] (4,0) to[capacitor,l=$\frac{20}{3}$] (4,-2);
			\draw[red] (4,-2) to[capacitor,l=$\frac{20}{3}$] (4,-4);
			\draw (0,-4) to[capacitor,l=$10$,*-] (2,-4) node[xshift=-2.25cm]{b};
			\draw (2,-4) to[short] (4,-4);
		\end{circuitikz}
	\end{center}
	\begin{gather*}
		\frac{1}{C_{eq}} =\frac{1}{6} + \frac{3}{20} + \frac{3}{20} + \frac{1}{10} = \frac{17}{30} \\
		\Rightarrow C_{eq} = 1.7
	\end{gather*}
\end{solu}
