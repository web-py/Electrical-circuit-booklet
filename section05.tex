\chapter{پاسخ طبیعی پله واحد}

\textbf{تابع تحریک پله واحد:}

\begin{gather*}
	u(t) = 
	\begin{cases}
		1 & t>0 \\
		0 & t<0
	\end{cases}
\end{gather*}

\textbf{تابع پله واحد تأخیردار: }

\begin{gather*}
	u(t-t_0) = 
	\begin{cases}
		1 & t>t_0 \\
		0 & t<t_0
	\end{cases}
\end{gather*}

\textbf{تابع تحریک ضربه: }

\begin{gather*}
	\delta(t) = 
	\begin{cases}
		1 & t=0 \\
		0 & otherwise
	\end{cases}
\end{gather*}

\begin{gather*}
	\delta(t-t_0) = 
	\begin{cases}
		1 & t=t_0 \\
		0 & O.W
	\end{cases}
\end{gather*}

\begin{remark}
	برای بدست آوردن پاسخ ضربه(یعنی پاسخ مدار به ورودی ضربه) کافی است ابتدا پاسخ پله(یعنی پاسخ مدار به ورودی پله) را بدست آوریم سپس از این پاسخ مشتق بگیریم.
\end{remark}

\begin{definition}
	پاسخ مدار یعنی به دست آوردن جریان یا ولتاژ در یک نقطه از مدار.
\end{definition}

\begin{remark}
	\ \\
	\begin{itemize}
		\item
		اگر مدار \textbf{مقاومتی} باشد و منبع نیز \textbf{متغیر} باشد، پاسخ مدار \textbf{متغیر}(وابسته به زمان) است.
		\item
		اگر مدار \textbf{مقاومتی} باشد و منبع \textbf{ثابت} باشد، پاسخ \textbf{ثابت} است.
		\item
		اگر مدار شامل \textbf{سلف یا خازن} و یا هردو باشد، پاسخ \textbf{همیشه متغیر}(وابسته به زمان) است.
	\end{itemize}
\end{remark}

\begin{definition}
	\textbf{درجه} به معنای بزرگترین توان  متغیر و \textbf{مرتبه} به معنای تعداد دفعاتی است که می‌توان مشتق گرفت.
\end{definition}

\begin{definition}
	\textbf{مدارهای مرتبه اول}، مدارهایی هستند که برای بدست آوردن رابطه‌ی ولتاژ یا جریان در آن به یک معادله مرتبه اول می‌رسیم.
\end{definition}

مدارهای مرتبه اول دو نوع هستند:
\begin{itemize}
	\item
	\textbf{مدارهای \lr{RL}:} این نوع مدارها شامل مقاومت و سلف هستند.
	\item
	\textbf{مدارهای \lr{RC}:} این نوع مدارها شامل مقاومت و خازن هستند.
\end{itemize}

در مدارهای مرتبه اول، به جای حل معادله‌ی مرتبه اول دیفرانسیل با روش‌های تشریحی می‌توان از فرمول زیر استفاده کرد:

\begin{gather*}
	y(t) = y(\infty) + \Big(y(0) - y(\infty)\Big).e^{-\frac{t}{\tau}}
\end{gather*}
 حال برای انواع مدارها داریم:
 
 \begin{minipage}{0.5\textwidth}
 	\begin{gather*}
 		RL \rightarrow 
 		\begin{cases}
 			y = I \\
 			\tau = \frac{L}{R_{eq}}
 		\end{cases}
 	\end{gather*}
 \end{minipage}
 \begin{minipage}{0.5\textwidth}
	\begin{gather*}
		RC \rightarrow 
		\begin{cases}
			y = V \\
			\tau = R_{eq}.C
		\end{cases}
	\end{gather*}
\end{minipage}


\begin{example}
	رابطه‌ی جریان را برای مدار زیر به دست آورید.
	
	\begin{center}
		\begin{circuitikz}
			\draw (0,2) to[american voltage source, l_=$12^{u(t)}$] (0,0);
			\draw (0,2) to[L,l=$1H$,i=$I$] (2,2);
			\draw (2,2) to[R,l=$1\Omega$] (2,0);
			\draw (0,0) to[short] (2,0);
		\end{circuitikz}
	\end{center}
\end{example}

\begin{solu}
	\begin{gather*}
		\tau = \frac{L}{R_{eq}} = \frac{1}{1} = 1 \\
		I(\infty) = 12A \\
		I(0) = 0 \\
		I(t) = 12 + (0-12)e^{\frac{-t}{1}} \\
		\rightarrow I(t) = 12(1-e^{-t})
	\end{gather*}
\end{solu}

\begin{remark}
	\ \\
	\begin{itemize}
		\item 
		سلف وقتی پر می‌شود به صورت اتصال کوتاه عمل می‌کند.
		\item
		خازن وقتی پر می‌شود، جریانی از خود عبور نمی‌دهد و آن را به صورت مدار باز در نظر بگیرید.
		\item
		سلف را در ابتدای مدار به صورت مدار باز در نظر بگیرید.
		\item
		خازن را در ابتدای مدار به صورت اتصال کوتاه در نظر بگیرید.
	\end{itemize}
\end{remark}

\begin{example}
	جریان گذرنده از سلف را به دست آورید.
	
	\begin{center}
		\begin{circuitikz}
			\draw (0,2) to[american voltage source, l_=$10u(t)$] (0,0);
			\draw (0,2) to[R, l=$2\Omega$] (2,2);
			\draw (2,0) to[L, l=$1H$] (2,2);
			\draw (4,0) to[R, l=$2\Omega$] (4,2);
			\draw (2,2) to[short] (4,2);
			\draw (0,0) to[short] (4,0);
		\end{circuitikz}
	\end{center}
\end{example}

\begin{solu}
	\ \\
	\begin{minipage}{0.4\textwidth}
		\begin{center}
			\begin{circuitikz}
				\draw (0,0) to[R,l=$2\Omega$] (0,2);
				\draw (2,0) to[R,l_=$2\Omega$] (2,2);
				\draw (1,0) to[short,-*] (1,0.5);
				\draw (1,2) to[short,-*] (1,1.5);
				\draw (0,2) to[short] (2,2);
				\draw (0,0) to[short] (2,0);
			\end{circuitikz}
		\end{center}
	\end{minipage}
	\begin{minipage}{0.1\textwidth}
		$\Rightarrow$
	\end{minipage}
	\begin{minipage}{0.4\textwidth}
			\begin{center}
			\begin{circuitikz}
				\draw (0,2) to[american voltage source, l_=$10u(t)$] (0,0);
				\draw (0,2) to[R, l=$2\Omega$] (2,2);
				\draw (2,0) to[L, l=$1H$] (2,2);
				\draw (4,0) to[R, l=$2\Omega$] (4,2);
				\draw (2,2) to[short] (4,2);
				\draw (0,0) to[short] (4,0);
			\end{circuitikz}
		\end{center}
	\end{minipage}
	\begin{gather*}
		R_{eq} = 2 || 2 = 1\Omega \\
		\tau = \frac{L}{R_{eq}} = \frac{1}{1} = 1 \\
		I(\infty) = \frac{10}{2} = 5A \\
		I(0) = 0 \\
		\rightarrow I(t) = 5+(0-5)e^{-t} = 5(1-e^{-t})
	\end{gather*}
\end{solu}

\begin{example}
	در مدار شکل زیر 
	$V_c(t)$
	را برای 
	$t>0$
	بیابید.
	\begin{center}
		\begin{circuitikz}
			\draw (0,2) to[american voltage source, l_=$36v$] (0,0);
			\draw (3,0) to[capacitor, l=$100\mu F$] (3,2) node[red,xshift=1cm,yshift=-1cm]{$V_c(t)$}
			node[red,xshift=1cm,yshift=-0.5cm]{$+$} node[red,xshift=1cm,yshift=-1.5cm]{$-$};
			\draw (6,0) to[switch,l=$t\=0$] (6,2);
			\draw (9,2) to[american voltage source, l=$12v$] (9,0);
			\draw (0,2) to[R,l=$2k$] (3,2);
			\draw (3,2) to[R,l=$6k$] (6,2);
			\draw (6,2) to[R,l=$4k$] (9,2);
			\draw (0,0) to[short] (9,0);
		\end{circuitikz}
	\end{center}
\end{example}

\begin{remark}
	اگر در مداری کلید داشتیم یک مرحله به مراحل قبل اضافه می‌شود.
\end{remark}


\begin{solu}
	مدار برای زمان 
	$t=0^-$
	به صورت زیر است.
		\begin{center}
		\begin{circuitikz}
			\draw (0,2) to[american voltage source, l_=$36v$] (0,0);
			\draw (3,0) to[short,-*] (3,0.5) node[red,xshift=0.5cm,yshift=0.5cm]{$V_c(0^-)$}
			node[red,xshift=0.5cm,yshift=1cm]{$+$} node[red,xshift=0.5cm,yshift=0cm]{$-$};
			\draw (3,2) to[short,-*] (3,1.5);
			\draw (9,2) to[american voltage source, l=$12v$] (9,0);
			\draw (0,2) to[R,l=$2k$] (3,2);
			\draw (3,2) to[R,l=$6k$] (6,2);
			\draw (6,2) to[R,l=$4k$] (9,2);
			\draw (0,0) to[short] (9,0);
		\end{circuitikz}
	\end{center}
 طیق قانون KVL داریم:
	\begin{gather*}
		2^kI + 6^kI = 4^kI +12-36=0 \rightarrow I = \frac{24}{12^k} = 2mA \\
		-36 + 2^k\times 2mA + V_c(0^-) = 0 \rightarrow V_c(0^-) = 32V
	\end{gather*}
	\begin{remark}
		ولتاژ خازن مقداری پیوسته است یعنی ولتاژ خازن در 
		$t=0^-$
		و 
		$t=0^+$
		با هم برابر است.
	\end{remark}
	مدار برای زمان 
	$t=0^+$
	به صورت زیر است.
	\begin{center}
		\begin{circuitikz}
			\draw (0,2) to[american voltage source, l_=$36v$] (0,0);
			\draw (3,0) to[capacitor, l=$100\mu F$] (3,2) node[red,xshift=1.75cm,yshift=-1cm]{{\scriptsize $V_c(0^+)=32V$}}
			node[red,xshift=1cm,yshift=-0.5cm]{$+$} node[red,xshift=1cm,yshift=-1.5cm]{$-$};
			\draw (6,0) to[short] (6,2);
			\draw (9,2) to[american voltage source, l=$12v$] (9,0);
			\draw (0,2) to[R,l=$2k$] (3,2);
			\draw (3,2) to[R,l=$6k$] (6,2);
			\draw (6,2) to[R,l=$4k$] (9,2);
			\draw (0,0) to[short] (9,0);
		\end{circuitikz}
	\end{center}
	\begin{remark}
		\ \\
		\begin{itemize}
			\item
			در زمان 
			$t=0^+$
			خازن نقش یک منبع را دارد.
			\item
			در زمان 
			$t=0^+$
			کلید وصل است.
		\end{itemize}
	\end{remark}
	مدار برای زمان 
	$t>0$
	به صورت زیر خواهد بود:
		\begin{center}
		\begin{circuitikz}
			\draw (0,2) to[american voltage source, l_=$36v$] (0,0);
			\draw (3,0) to[capacitor, l=$100\mu F$] (3,2) node[red,xshift=1cm,yshift=-1cm]{{\scriptsize $V_c(t)$}}
			node[red,xshift=1cm,yshift=-0.5cm]{$+$} node[red,xshift=1cm,yshift=-1.5cm]{$-$};
			\draw (6,0) to[short] (6,2);
		
			\draw (0,2) to[R,l=$2k$] (3,2);
			\draw (3,2) to[R,l=$6k$] (6,2);
		
			\draw (0,0) to[short] (6,0);
		\end{circuitikz}
	\end{center}
	مدار برای زمان 
	$t=+\infty$
	به صورت زیر خواهد بود:
	
	\begin{center}
		\begin{circuitikz}
			\draw (0,2) to[american voltage source, l_=$36v$] (0,0);
			\draw (3,0) to[short,-*] (3,0.5) node[red,xshift=0.5cm,yshift=0.5cm]{$V_c(\infty)$}
			node[red,xshift=0.5cm,yshift=1cm]{$+$} node[red,xshift=0.5cm,yshift=0cm]{$-$};
			\draw (6,0) to[short] (6,2);
			\draw (3,2) to[short,-*] (3,1.5);
			\draw (0,2) to[R,l=$2k$] (3,2);
			\draw (3,2) to[R,l=$6k$] (6,2);
			
			\draw (0,0) to[short] (6,0);
		\end{circuitikz}
	\end{center}
	حال داریم:
	\begin{gather*}
		V_c = \frac{6}{2+6}\times 36 = 27v \\
		\tau = R_{eq}.C \rightarrow R_{eq} = 2 || 6 = \frac{2\times 6}{8} = 1.5 \Omega \\
		\tau = 1.5 \times 10^3 \times 100 \times 10^{-6} = 150 \times 10^{-3}=0.15 \\
		V_c(t) = V_c(\infty) + (V_t(0) - V_t(\infty))e^{\frac{-t}{\tau}} \\
		=27 + (32-27)e^{\frac{-t}{0.15}} \\
		V_c(t)=27 + 5e^{\frac{-t}{0.15}}
	\end{gather*}
\end{solu}

