\chapter{پاسخ‌های طبیعی و پله مدارهای RLC}

\begin{definition}
	\textbf{مدارهای \lr{RLC}: }مدارهایی که هم خازن دارند، هم سلف و هم مقاومت. این مدارها از نوع مرتبه دوم هستند(یعنی توصیف آن با معادلات دیفرانسیل خطی مرتبه دوم انجام می‌شود.)
\end{definition}

\noindent
\textbf{روال حل مسائل این قسمت:}

ابتدا مدارهای RLC بدون منبع را در نظر می‌گیریم و پاسخ طبیعی را بدست می‌آوریم. سپس منابع DC کلیدها یا منابع پله را به مدار می‌افزاییم و پاسخ کامل را به صورت مجموع پاسخ ویژه(واداشته) و طبیعی می‌نویسیم و مقدار ثابت‌ها را با اعمال شرایط اولیه پیدا می‌کنیم.


\subsection*{پاسخ طبیعی مدار RLC موازی:}

پاسخ طبیعی یعنی فرض می کنیم که مدار بدون منبع است و در سلف و خازن انرژی ذخیره شده است. انرژی ذخیره شده اولیه سلف را 
$I_0$
(جریان) و انرژی ذخیره شده اولیه خازن را 
$V_0$
(ولتاژ) در نظر می‌گیریم.

\begin{center}
	\begin{circuitikz}
		\draw (0,0) to[capacitor,l=$C$] (0,2);
		\draw (0,2) to[short,i<=$i_C$,-*] (2,2);
		\draw (2,2) to[short,i=$i_R$] (4,2);
		\draw (0,0) to[short] (4,0);
		\draw (2,0) to[L,i<=$i_L$,l=$L$] (2,2);
		\draw (4,0) to[R,l=$R$] (4,2);
	\end{circuitikz}
\end{center}
\begin{gather*}
	i_R + i_C + i_L = 0 \\
	\Rightarrow 
	\begin{cases}
		i_R = \frac{V}{R} \\
		i_L = \frac{1}{L} \int_{0}^{t} V dt + I_0 \\
		i_c = C\frac{dv}{dt}
	\end{cases}
\end{gather*}
\begin{gather*}
	i_R + i_C + i_L = 0 \\
	\Rightarrow \frac{V}{R} + C\frac{dV}{dt} + \frac{1}{L} \int_{0}^{t} V dt + I_0 = 0
\end{gather*}
معادلاتی که هم مشتق دارند و هم انتگرال، باید یا مشتق را حذف کنیم و یا انتگرال. برای حذف انتگرال از کل معادله مشتق می‌گیریم.

\begin{gather*}
	\frac{1}{R} \frac{dV}{dt} + C \frac{d^2V}{dt^2} + \frac{1}{L} V = 0 
\end{gather*}

\begin{gather}\label{V-formula}
	\frac{d^2V}{dt^2} + \frac{1}{RC}\frac{dV}{dt} + \frac{V}{LC} = 0 \quad \rightarrow \quad \text{\rl{معادله دیفرانسیل مرتبه دوم}}
\end{gather}
فرض می‌کنیم که 
$V=ke^{st}$
در این صورت خواهیم داشت:
\begin{gather*}
	V=ke^{st} \rightarrow \quad \frac{d'V}{dt} = kse^{st} \rightarrow \quad \frac{d^2V}{dt} = ks^2e^{st} 
\end{gather*}

حال مقدار V را در معادله دیفرانسیل 
\eqref{V-formula}
جایگذاری می‌کنیم و خواهیم داشت:
\begin{gather*}
	ks^2e^{st} + \frac{1}{RC} kse^{st} + \frac{ke^{st}}{LC} = 0 \\
	\overbrace{ke^{st}}^1\Big(s^2 + \frac{1}{RC}s + \frac{1}{LC}\Big) = 0
\end{gather*}
قسمت یک که هیچ‌گاه صفر نمی‌شود بنابراین:
\begin{gather*}
	s^2 + \frac{1}{RC}s + \frac{1}{LC} = 0
\end{gather*}
که به آن معادله مشخصه مدار گفته می‌شود زیر ریشه‌های این معادله رابطه ریاضی 
$V(t)$
را تعیین خواهد کرد. به کمک روش دلتا نیز می‌توان ریشه‌های آن را بدست آورد که ریشه‌های به صورت زیر خواهند بود:
\begin{gather*}
	\begin{cases}
		s_1 = \frac{-1}{2RC} + \sqrt{(\frac{1}{2RC})^2 - \frac{1}{LC}} \vspace{0.5cm} \\
		s_2 = \frac{-1}{2RC} - \sqrt{(\frac{1}{2RC})^2 - \frac{1}{LC}}
	\end{cases}
	V = V_1 + V_2 = k_1e^{s_1t} + k_2e^{s_2t}
\end{gather*}


حال برای ساده‌تر کردن ریشه‌ها می‌توان عبارت‌های زیر را در نظر گرفت:
\begin{gather*}
	\begin{cases}
		\alpha = \frac{1}{2RC} &\quad \rightarrow \quad \text{\rl{ضریب میرایی پاسخ طبیعی}} \\
		\omega_0 = \frac{1}{\sqrt{LC}} &\quad \rightarrow \quad \text{\rl{فرکانش تشدید پاسخ طبیعی(مدار)}}
	\end{cases}
\end{gather*}
بنابراین:
\begin{gather*}
	\begin{cases}
		s_1 = \frac{-1}{2RC} + \sqrt{(\frac{1}{2RC})^2 - \frac{1}{LC}} = -\alpha + \sqrt{\alpha^2 - \omega_0^2} \vspace{0.5cm} \\
		s_2 = \frac{-1}{2RC} - \sqrt{(\frac{1}{2RC})^2 - \frac{1}{LC}} = -\alpha - \sqrt{\alpha^2 - \omega_0^2}
	\end{cases}
\end{gather*}

حال براساس 
$\alpha$
و 
$\omega_0$
سه نوع جواب داریم:

\begin{description}
	\item[$\alpha > \omega_0$:] 
	در این حالت دو ریشه حقیقی و متمایز داریم و به آن پاسخ طبیعی فوق میرا گفته می‌شود. 
	\begin{gather*}
		V(t) = k_1e^{s_1t}+k_2e^{s_2t}
	\end{gather*}
	
	\item[$\alpha < \omega_0$:] 
	در این حالت دو ریشه مختلط و مزدوج داریم و به آن پاسخ طبیعی زیرمیرا گفته می‌شود. 
	\begin{gather*}
		V(t) = e^{-\alpha t}\Big(k_1 cos\omega_n t+ k_2sin \omega_n t\Big)
	\end{gather*}
	\begin{remark}
		به $\omega_n$ فرکانس تشدید طبیعی گفته می‌شود که از رابطه‌ی زیر محاسبه می‌شود:
		\begin{gather*}
			\omega_n = \sqrt{\omega_0^2 - \alpha^2}
		\end{gather*}
	\end{remark}
	
	\item[$\alpha = \omega_0$:] 
	در این حالت هر دو ریشه حقیقی و مساوی است و به آن پاسخ طبیعی میرای بحرانی گفته می‌شود.
	\begin{gather*}
		V(t) = k_1te^{-\alpha t} + k_2e^{-\alpha t}
	\end{gather*}
\end{description}

\begin{example}
	پاسخ طبیعی مدار زیر را بدست آورید.
	\begin{center}
		\begin{circuitikz}
			\draw (0,0) to[capacitor,l=$\frac{1}{16}F$] (0,2);
			\draw (0,2) to[short] (2,2);
			\draw (2,2) to[short] (4,2);
			\draw (0,0) to[short] (4,0);
			\draw (2,0) to[L,l=$4H$] (2,2);
			\draw (4,0) to[R,l=$2\Omega$] (4,2);
		\end{circuitikz}
	\end{center}
\end{example}

\begin{solu}
	\ \\
	\begin{gather*}
		\alpha = \frac{16}{2\times 2 } = 4 \\
		\omega_0 = \frac{1}{\sqrt{\frac{1}{4}}} = 2
	\end{gather*}
	با توجه به حالات بالا پاسخ طبیعی فوق میراست. لذا:
	\begin{gather*}
		V(t) = k_1e^{s_1t}+k_2e^{s_2t} \\
		s_1 = -4 + \sqrt{16-4} = -0.54 \\
		s_2 = -4 - \sqrt{16-4} = -7.46 \\
		V(t) = k_1e^{-0.54t}+k_2e^{-7.46t} \\
	\end{gather*}
\end{solu}

\begin{example}
	در مدار شکل زیر اگر 
	$V_c(0) = 2V$
	و
	$i_L(0) = -2A$
	باشد، مطلوب است مقدار 
	$V(t)$
	برای
	$t>0$.
	
	\begin{center}
		\begin{circuitikz}
			\draw (0,0) to[capacitor,l=$\frac{1}{4}F$] (0,2);
			\draw (0,2) to[short] (2,2);
			\draw (2,2) to[short] (4,2);
			\draw (0,0) to[short] (4,0);
			\draw (2,0) to[L,l=$1H$] (2,2);
			\draw (4,0) to[R,l=$2\Omega$] (4,2) node[yshift=-0.25cm,xshift=0.75cm](){+} node[yshift=-1.75cm, xshift=0.75cm]{-}
			node[yshift = -1cm, xshift=0.75cm]{$V(t)$};
		\end{circuitikz}
	\end{center}
	
\end{example}

\begin{solu}
	اولین کار برای حل این تیپ سوالات بدست آوردن 
	$\alpha, \omega$
	است. بنابراین:
	\begin{gather*}
		\alpha = \frac{1}{2RC} = \frac{1}{2\times2\times\frac{1}{4}} = 1 \\
		\omega_0 = \frac{1}{\sqrt{1\times\frac{1}{4}}} = 2
	\end{gather*}
	از آنجایی که 
	$\alpha < \omega_0$
	است پاسخ زیرمیراست. پس:
	\begin{gather*}
		V(t) = e^{-\alpha t}\Big(k_1 cos\omega_n t+ k_2sin \omega_n t\Big) \\
		\omega_n = \sqrt{\omega_0^2 - \alpha^2} = \sqrt{4-1} = \sqrt{3}\\
		\Rightarrow V(t) = e^{-\alpha t}\Big(k_1 cos\sqrt{3}t+ k_2sin \sqrt{3} t\Big) \\
	\end{gather*}
	حال باید مقادیر 
	$k_1, k_2$
	را محاسبه کنیم:
	\begin{gather*}
		V(0) = 1\Big(k_1 cos 0+ k_2 sin 0\Big) = k_1 = 2\\
	\end{gather*}
	و برای بدست آوردن 
	$k_2$
	مدار را در زمان صفر در نظر می‌گیریم یعنی به صورت زیر:
	\begin{center}
		\begin{circuitikz}
			\draw (0,0) to[capacitor,l=$\frac{1}{4}F$] (0,2);
			\draw (0,2) to[short, i<=$c\frac{dv}{dt}$] (2,2);
			\draw (2,2) to[short,i=$\frac{V}{R}$] (4,2);
			\draw (0,0) to[short] (4,0);
			\draw (2,0) to[L,l=$1H$, i<=$-2A$] (2,2);
			\draw (4,0) to[R,l=$2\Omega$] (4,2) node[yshift=-0.25cm,xshift=0.75cm](){+} node[yshift=-1.75cm, xshift=0.75cm]{-}
			node[yshift = -1cm, xshift=0.75cm]{$V(t)$};
		\end{circuitikz}
	\end{center}
	\begin{gather*}
		\frac{dv}{dt} = -e^{-t}\Big(k_1 cos\sqrt{3}t+ k_2sin \sqrt{3}t\Big) 
		+ e^{-t}\Big(-\sqrt{3}k_1 sin\sqrt{3}t+ \sqrt{3}k_2 cos \sqrt{3}t\Big) \\
		i_c = C\frac{dv}{dt} = \frac{1}{4}\Big[ -1(2+0) + 1(0+\sqrt{3}k_2) \Big] = -\frac{1}{2} + \frac{\sqrt{3}}{4}k_2
	\end{gather*}
	با توجه به قانون KCL داریم:
	\begin{gather*}
		i_R + i_L + i_C = 0 \\
		i_R = \frac{V}{R} = \frac{2}{2} = 1 \\
		1 -2 -\frac{1}{2} + \frac{\sqrt{3}}{4}k_2 = 0 \rightarrow k_2 = 3.47 \\
		\Rightarrow V(t) = e^{-t}\Big(2 cos\sqrt{3}t+ 3.47 sin \sqrt{3} t\Big) 
	\end{gather*}
\end{solu}
\subsection*{پاسخ طبیعی مدار RLC سری:}
همانطور که در شکل زیر هم مشاهده می‌کنید در این مدارها در هر سه المان یک جریان برقرار است.
بنابراین در این مدارها به دنبال به دست آوردن \textit{جریان} هستیم.
\begin{center}
	\begin{circuitikz}
		\draw (0,0) to[L,l_=$L$] (3,0)
		node[yshift=0.5cm,xshift = -1cm]{-}
		node[yshift = 0.5cm, xshift = -2cm]{+}
		node[yshift = 0.5cm, xshift=-1.5cm]{$V_L$}
		;
		\draw (0,0) to[R,l=$R$, i<=$i$] (0,-3)
		node[yshift = 1cm, xshift = -0.5cm]{+}
		node[yshift = 2cm, xshift = -0.5cm]{-}
		node[yshift = 1.5cm, xshift = -0.5cm]{$V_R$}
		;
		\draw (3,0) to[capacitor,l_=$C$] (3,-3)
		node[yshift = 1cm, xshift = 0.75cm]{-}
		node[yshift = 2cm, xshift = 0.75cm]{+}
		node[yshift = 1.5cm, xshift = 0.75cm]{$V_C$}
		;
		\draw (0,-3) to[short] (3,-3);
		\draw[<-,blue] (1.5,-1.5) arc (30:330:0.25);
	\end{circuitikz}
\end{center}
فرض می‌کنیم که انرژی اولیه ذخیره شده در سلف 
$I_0$
و انرژی اولیه ذخیره شده در خازن هم 
$V_0$
باشد. از طرفی با توجه به جریان مشخص شده و قانون 
KVL
داریم:
\begin{gather*}
	V_L + V_C + V_R = 0 \Rightarrow L\frac{di}{dt} + \frac{1}{C}\int i\ dt + V_0 + Ri = 0 \\
	L\frac{d^2i}{dt^2} + \frac{1}{C}i + R\frac{di}{dt} = 0 \\
	\frac{d^2i}{dt^2} + \frac{R}{L}\frac{di}{dt} + \frac{1}{LC}i = 0 \\
	\Rightarrow s^2 + \frac{R}{L}s + \frac{1}{LC} = 0 \\
	\begin{cases}
		s_1 = \frac{-R}{2L} + \sqrt{(\frac{R}{2L})^2-\frac{1}{LC}} \\
		s_2 = \frac{-R}{2L} - \sqrt{(\frac{R}{2L})^2-\frac{1}{LC}}
	\end{cases}
	\rightarrow 
	\begin{cases}
		\alpha = \frac{R}{2L} \rightarrow \text{\rl{ضریب میرایی پاسخ طبیعی}}\\
		\omega_0 = \frac{1}{\sqrt{LC}} \rightarrow \text{\rl{فرکانس تشدید}}
	\end{cases}
\end{gather*}
به مانند قبل با توجه به 
$\alpha, \omega_0$
سه جواب متمایز خواهیم داشت. 
\begin{description}
	\item[$\alpha > \omega_0$:] 
	در این حالت دو ریشه حقیقی و متمایز داریم و به آن پاسخ طبیعی فوق میرا گفته می‌شود. 
	\begin{gather*}
		i(t) = k_1e^{s_1t}+k_2e^{s_2t}
	\end{gather*}
	
	\item[$\alpha < \omega_0$:] 
	در این حالت دو ریشه مختلط و مزدوج داریم و به آن پاسخ طبیعی زیرمیرا گفته می‌شود. 
	\begin{gather*}
		i(t) = e^{-\alpha t}\Big(k_1 cos\omega_n t+ k_2sin \omega_n t\Big)
	\end{gather*}
	\begin{remark}
		به $\omega_n$ فرکانس تشدید طبیعی گفته می‌شود که از رابطه‌ی زیر محاسبه می‌شود:
		\begin{gather*}
			\omega_n = \sqrt{\omega_0^2 - \alpha^2}
		\end{gather*}
	\end{remark}
	
	\item[$\alpha = \omega_0$:] 
	در این حالت هر دو ریشه حقیقی و مساوی است و به آن پاسخ طبیعی میرای بحرانی گفته می‌شود.
	\begin{gather*}
		i(t) = k_1te^{-\alpha t} + k_2e^{-\alpha t}
	\end{gather*}
\end{description}

\begin{example}
	در شکل زیر اگر 
	$R=1\Omega, L = 1H, C=\frac{1}{25} F, i_L(0) = 2A, V_C(0) = 2V$
	مقدار 
	$i(t)$
	را برای 
	$t>0$
	بیابید.
	
	\begin{center}
		\begin{circuitikz}
			\draw (0,0) to[L,l_=$L$] (3,0)
			node[yshift=0.5cm,xshift = -1cm]{-}
			node[yshift = 0.5cm, xshift = -2cm]{+}
			node[yshift = 0.5cm, xshift=-1.5cm]{$V_L$}
			;
			\draw (0,0) to[R,l=$R$, i<=$i$] (0,-3)
			node[yshift = 1cm, xshift = -0.5cm]{+}
			node[yshift = 2cm, xshift = -0.5cm]{-}
			node[yshift = 1.5cm, xshift = -0.5cm]{$V_R$}
			;
			\draw (3,0) to[capacitor,l_=$C$] (3,-3)
			node[yshift = 1cm, xshift = 0.75cm]{-}
			node[yshift = 2cm, xshift = 0.75cm]{+}
			node[yshift = 1.5cm, xshift = 0.75cm]{$V_C$}
			;
			\draw (0,-3) to[short] (3,-3);
		\end{circuitikz}
	\end{center}
\end{example}

\begin{solu}
	اولین گام برای حل مسائل 
	RLC 
	بدست آوردن مقادیر
	$\alpha, \omega_0$
	است. پس:
	\begin{gather*}
		\alpha = \frac{R}{2L} = \frac{6}{2\times1} = 3 \\
		\omega_0 = \frac{1}{\sqrt{LC}} = \frac{1}{\sqrt{1\times\frac{1}{25}}} = 5
	\end{gather*}
	با توجه به اینکه
	$\alpha < \omega_0$
	است نوع پاسخ زیرمیراست بنابراین داریم:
	\begin{gather*}
		i(t) = e^{-\alpha t}\Big(k_1 cos\omega_n t+ k_2sin \omega_n t\Big) \\
		\omega_n = \sqrt{\omega_0^2 - \alpha^2} = \sqrt{25-9} = 4 \\
		i(0) = 2A \rightarrow 2 = k_1 \\
		KVL \rightarrow V_R + V_L + V_C = 0 \rightarrow R\times i(0) + L\frac{di}{dt} + V_c(0) = 0 \\
		\rightarrow k_2 = -2 \\
		\Rightarrow i(t) = e^{-2t}\Big(2 cos4t - 2sin4t\Big) 
	\end{gather*}
\end{solu}

\begin{example}
	در مدار شکل زیر اگر
	$V_c(0) = -6V, i_L(0) = 4A$
	باشد،
	$V_c(t)$
	را در 
	$t>0$
	محاسبه نمایید.
	
	\begin{center}
		\begin{circuitikz}
			\draw (0,2) to[american voltage source,l_=$10u(t)$] (0,0);
			\draw (0,2) to[R,l=$6\Omega$] (3,2);
			\draw (3,2) to[L,l=$1H$] (6,2);
			\draw (6,2) to[capacitor,l_=$\frac{1}{25}F$] (6,0)
			node[xshift=0.75cm, yshift=1.5cm]{+}
			node[xshift=0.75cm, yshift=1cm]{$V_c$}
			node[xshift=0.75cm, yshift=0.5cm]{-}
			;
			\draw (0,0) to[short] (6,0);
		\end{circuitikz}
	\end{center}
\end{example}

\begin{solu}
	ابتدا مقادیر
	$\alpha, \omega_0$
	را محاسبه می‌کنیم:
	\begin{gather*}
		\begin{cases}
			\alpha = \frac{R}{2L} = \frac{6}{2} = 3 \\
			\omega_0 = \frac{1}{\sqrt{LC}} = \frac{1}{1\times\frac{1}{25}} = 5
		\end{cases}
	\end{gather*}
	
	با توجه به اینکه
	$\alpha < \omega_0$
	است نوع پاسخ زیرمیراست. از طرفی وقتی مدار منبع داشته باشد و مقدار آن تابع پله‌ای باشد، داریم:
	\begin{gather*}
		V_c(t) = 10 + e^{-3 t}\Big(k_1 cos4 t+ k_2sin 4 t\Big) \\
		\omega_n = \sqrt{\omega_0^2 - \alpha^2} = \sqrt{25-9} = 4 \\
		V_c(0) = -6 = 10 + k_1 \rightarrow k_1 = -16 \\
		i(t) = C\frac{dv(t)}{dt} = \frac{1}{25}\Big( 0 + (-3e^{-3t})(k_1 cos4 t+ k_2sin 4 t)+(e^{-3t})(-4k_1sin4t + 4k_2 cos 4t) \Big) \\
		\text{\rl{جریان مقاومت}} = \text{\rl{جریان خازن}}=\text{\rl{جریان سلف}} \\
		i_L(0) = 4A = \frac{1}{25}\Bigl(-3(-16) + 4k_2\Bigr) \rightarrow k_2 = 13
	\end{gather*}
\end{solu}


















