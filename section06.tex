\section{تجزیه و تحلیل حالت ماندگاری سینوسی}
\subsection*{مقدمات}
\begin{gather*}
    V(t) = V_m\ Sin(\omega t) \\
    t \rightarrow \text{\rl{زمان بر حسب ثانیه}}\\
    V_m \rightarrow \text{\rl{دامنه ولتاژ}} \\
    \omega \rightarrow \text{\rl{فرکانس زاویه‌ای}}
    \begin{cases}
        \omega = 2\pi f = \frac{2\pi}{T} \\
        f \rightarrow \text{\rl{فرکانس یا بسامد}} - \text{\rl{واحد آن هرتز}}\\
        T \rightarrow \text{\rl{دوره‌ی تناوب}} - \text{\rl{واحد آن ثانیه}}
    \end{cases}
\end{gather*}

\subsection*{توصیف اعداد مختلط}

\begin{center}
    \begin{minipage}{0.4\textwidth}
        \begin{gather*}
            z = \sqrt{x^2+y^2} \\
            \theta = tan^{-1}(\frac{y}{x})
        \end{gather*}
    \end{minipage}
    \begin{minipage}{0.4\textwidth}
        \begin{gather*}
            \begin{cases}
                z = x + jy \rightarrow \text{\rl{نمایش معمول}}\\
                z = |z|e^{j\theta} \rightarrow \text{\rl{نمایش قطبی}}\\
                z = |z|\measuredangle \theta \rightarrow \text{\rl{نمایش برداری}}
            \end{cases}
        \end{gather*}
    \end{minipage}
\end{center}

\subsubsection*{جمع و تفریق اعداد مختلط}
\begin{center}
    \begin{minipage}{0.4\textwidth}
        \begin{gather*}
            z_1 \pm z_2 = (x_1\pm x_2) + j(y_1\pm y_2)
        \end{gather*}
    \end{minipage}
    \begin{minipage}{0.4\textwidth}
        \begin{gather*}
            \begin{cases}
                z_1 = x_1 + jy_1 \\
                z_2 = x_2 + jy_2 
            \end{cases}
        \end{gather*}
    \end{minipage}
\end{center}

\subsubsection*{ضرب اعداد مختلط}
\begin{gather*}
    z_1 . z_2 = (x_1 + jy_1) . (x_2 + jy_2) \\
    = x_1x_2 - y_1y_2 +j(x_1y_2 + x_2y_1) \\
    = r_1r_2 \measuredangle (\theta_1 + \theta_2)
    \rightarrow
    \begin{cases}
        r_1 = \sqrt{x_1^2+y_1^2} \\
        r_2 = \sqrt{x_2^2+y_2^2}\\
        \theta_1 = tan^{-1}(\frac{y_1}{x_1}) \\
        \theta_2 = tan^{-1}(\frac{y_2}{x_2})
    \end{cases}
\end{gather*}

\subsubsection*{تقسیم اعداد مختلط}
\begin{gather*}
    \frac{z_1}{z_2} = \frac{x_1 + jy_1}{x_2 + jy_2} = 
    \frac{r_1 \measuredangle \theta_1}{r_2 \measuredangle \theta_2}\\
    = \frac{r_1}{r_2} \measuredangle (\theta_1 - \theta_2)
    \rightarrow
    \begin{cases}
        r_1 = \sqrt{x_1^2+y_1^2} \\
        r_2 = \sqrt{x_2^2+y_2^2}\\
        \theta_1 = tan^{-1}(\frac{y_1}{x_1}) \\
        \theta_2 = tan^{-1}(\frac{y_2}{x_2})
    \end{cases}
\end{gather*}

\subsection*{نمایش در فرم فاز برداری}
جریان‌ها و ولتاژهای 
$cos$
را برای ساده‌تر کردن تحلیل می‌توان به صورت زیر نوشت. دقت شود چنین نحوه‌ی نمایشی فقط در مورد توابع 
$cos$
است و اگر تابع 
$sin$
باشد باید ابتدا آن را به یک تابع 
$cos$
تبدیل کنید.

\begin{gather*}
    V(t) = I_m\ cos(\omega t + \phi) = I_m\measuredangle \phi
\end{gather*}

\begin{example}
    جریان و ولتاژ زیر را به فرم فاز برداری نمایش دهید.

    الف)
    $i(t) = 10\ cos(377t+20^{\circ})$
    
    ب)
    $V(t) = 20\ cos(377t+15^{\circ})$
\end{example}

\begin{solu}
    الف:
    \begin{gather*}
        i(t) = 10\ cos(377t+20^{\circ}) = 10 \measuredangle 20^{\circ}
    \end{gather*}
    ب:
    \begin{gather*}
        V(t) = 20\ cos(377t+15^{\circ}) = 20 \measuredangle 15^{\circ} 
    \end{gather*}
\end{solu}

\begin{example}
    اگر 
    $f=60\ Hz$
    و نمایش فاز برداری ولتاژ به صورت 
    $V = 25 \measuredangle 45^{\circ}  $
    باشد. معادله ولتاژ را در حوزه زمانی بدست آورید.
\end{example}

\begin{solu}
   \begin{center}
    \begin{minipage}{0.4\textwidth}
        \begin{gather*}
            V(t) = 25\ cos(377t + 45^{\circ})
        \end{gather*}
    \end{minipage}
    \begin{minipage}{0.4\textwidth}
        \begin{gather*}
            \begin{cases}
                V_m = 25 \\
                \omega = 2 \times 3.14 \times 60 = 377 \\
                \phi = 45^{\circ}
            \end{cases}
        \end{gather*}
    \end{minipage}
   \end{center}
\end{solu}