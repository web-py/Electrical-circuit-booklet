\section{تجزیه و تحلیل حالت ماندگاری سینوسی}
\subsection*{مقدمات}
\begin{gather*}
    V(t) = V_m\ Sin(\omega t) \\
    t \rightarrow \text{\rl{زمان بر حسب ثانیه}}\\
    V_m \rightarrow \text{\rl{دامنه ولتاژ}} \\
    \omega \rightarrow \text{\rl{فرکانس زاویه‌ای}}
    \begin{cases}
        \omega = 2\pi f = \frac{2\pi}{T} \\
        f \rightarrow \text{\rl{فرکانس یا بسامد}} - \text{\rl{واحد آن هرتز}}\\
        T \rightarrow \text{\rl{دوره‌ی تناوب}} - \text{\rl{واحد آن ثانیه}}
    \end{cases}
\end{gather*}

\subsection*{توصیف اعداد مختلط}

\begin{center}
    \begin{minipage}{0.4\textwidth}
        \begin{gather*}
            z = \sqrt{x^2+y^2} \\
            \theta = tan^{-1}(\frac{y}{x})
        \end{gather*}
    \end{minipage}
    \begin{minipage}{0.4\textwidth}
        \begin{gather*}
            \begin{cases}
                z = x + jy \rightarrow \text{\rl{نمایش معمول}}\\
                z = |z|e^{j\theta} \rightarrow \text{\rl{نمایش قطبی}}\\
                z = |z|\measuredangle \theta \rightarrow \text{\rl{نمایش برداری}}
            \end{cases}
        \end{gather*}
    \end{minipage}
\end{center}

\subsubsection*{جمع و تفریق اعداد مختلط}
\begin{center}
    \begin{minipage}{0.4\textwidth}
        \begin{gather*}
            z_1 \pm z_2 = (x_1\pm x_2) + j(y_1\pm y_2)
        \end{gather*}
    \end{minipage}
    \begin{minipage}{0.4\textwidth}
        \begin{gather*}
            \begin{cases}
                z_1 = x_1 + jy_1 \\
                z_2 = x_2 + jy_2 
            \end{cases}
        \end{gather*}
    \end{minipage}
\end{center}

\subsubsection*{ضرب اعداد مختلط}
\begin{gather*}
    z_1 . z_2 = (x_1 + jy_1) . (x_2 + jy_2) \\
    = x_1x_2 - y_1y_2 +j(x_1y_2 + x_2y_1) \\
    = r_1r_2 \measuredangle (\theta_1 + \theta_2)
    \rightarrow
    \begin{cases}
        r_1 = \sqrt{x_1^2+y_1^2} \\
        r_2 = \sqrt{x_2^2+y_2^2}\\
        \theta_1 = tan^{-1}(\frac{y_1}{x_1}) \\
        \theta_2 = tan^{-1}(\frac{y_2}{x_2})
    \end{cases}
\end{gather*}

\subsubsection*{تقسیم اعداد مختلط}
\begin{gather*}
    \frac{z_1}{z_2} = \frac{x_1 + jy_1}{x_2 + jy_2} = 
    \frac{r_1 \measuredangle \theta_1}{r_2 \measuredangle \theta_2}\\
    = \frac{r_1}{r_2} \measuredangle (\theta_1 - \theta_2)
    \rightarrow
    \begin{cases}
        r_1 = \sqrt{x_1^2+y_1^2} \\
        r_2 = \sqrt{x_2^2+y_2^2}\\
        \theta_1 = tan^{-1}(\frac{y_1}{x_1}) \\
        \theta_2 = tan^{-1}(\frac{y_2}{x_2})
    \end{cases}
\end{gather*}

\subsection*{نمایش در فرم فاز برداری}
جریان‌ها و ولتاژهای 
$cos$
را برای ساده‌تر کردن تحلیل می‌توان به صورت زیر نوشت. دقت شود چنین نحوه‌ی نمایشی فقط در مورد توابع 
$cos$
است و اگر تابع 
$sin$
باشد باید ابتدا آن را به یک تابع 
$cos$
تبدیل کنید.

\begin{gather*}
    V(t) = I_m\ cos(\omega t + \phi) = I_m\measuredangle \phi
\end{gather*}

\begin{example}
    جریان و ولتاژ زیر را به فرم فاز برداری نمایش دهید.

    الف)
    $i(t) = 10\ cos(377t+20^{\circ})$
    
    ب)
    $V(t) = 20\ cos(377t+15^{\circ})$
\end{example}

\begin{solu}
    الف:
    \begin{gather*}
        i(t) = 10\ cos(377t+20^{\circ}) = 10 \measuredangle 20^{\circ}
    \end{gather*}
    ب:
    \begin{gather*}
        V(t) = 20\ cos(377t+15^{\circ}) = 20 \measuredangle 15^{\circ} 
    \end{gather*}
\end{solu}

\begin{example}
    اگر 
    $f=60\ Hz$
    و نمایش فاز برداری ولتاژ به صورت 
    $V = 25 \measuredangle 45^{\circ}  $
    باشد. معادله ولتاژ را در حوزه زمانی بدست آورید.
\end{example}

\begin{solu}
   \begin{center}
    \begin{minipage}{0.4\textwidth}
        \begin{gather*}
            V(t) = 25\ cos(377t + 45^{\circ})
        \end{gather*}
    \end{minipage}
    \begin{minipage}{0.4\textwidth}
        \begin{gather*}
            \begin{cases}
                V_m = 25 \\
                \omega = 2 \times 3.14 \times 60 = 377 \\
                \phi = 45^{\circ}
            \end{cases}
        \end{gather*}
    \end{minipage}
   \end{center}
\end{solu}

\subsection{امپدانس و ادمیتانس}

در مهندسی، تحلیل در دو فضا انجام می‌شود:

\begin{itemize}
	\item
	فضای فرکانسی: فضای متغیر در واحد فرکانس
	\item
	فضای زمانی: فضای متغیر در واحد زمانی
\end{itemize}

با استفاده از نمایش در فرم فرکانس می‌توان با استفاده از مفهوم دیگری به نام امپدانس (یا ادمیتانس) به سلف و خازن به چشم یک مقاومت نگاه کرد. در این حالت از معادلات دیفرانسیلی خبری نیست. امپدانس را معمولاً با Z  و ادمیتانس را با Y نمایش می‌دهند. داریم که:
\begin{gather*}
	Z = \frac{1}{Y}
\end{gather*}

امپدانس خازن:

\vspace{0.5cm}
\begin{minipage}{0.5\textwidth}
	\begin{gather*}
		Z_c = \frac{1}{jwc} = \frac{-j}{wc} \rightarrow Y_c = jwc
	\end{gather*}
\end{minipage}
\begin{minipage}{0.3\textwidth}
	\begin{center}
		\begin{circuitikz}
			\draw (0,0) to[capacitor] (0,2);
		\end{circuitikz}
	\end{center}
\end{minipage}

\vspace{1cm}

امپدانس سلف:

\vspace{0.5cm}
\begin{minipage}{0.5\textwidth}
	\begin{gather*}
		Z_L = jwL \rightarrow Y_L = \frac{1}{jwL} = \frac{-j}{wL}
	\end{gather*}
\end{minipage}
\begin{minipage}{0.3\textwidth}
	\begin{center}
		\begin{circuitikz}
			\draw (0,0) to[L] (0,2);
		\end{circuitikz}
	\end{center}
\end{minipage}

\begin{example}
	مقدار جریان را در شکل زیر بیابید.
	
	\begin{minipage}{0.3\textwidth}
		\begin{gather*}
			V_s = 17.9 \measuredangle 30^\circ	
		\end{gather*}
	\end{minipage}
\begin{minipage}{0.5\textwidth}
	\vspace{0.5cm}
		\begin{center}
		\begin{circuitikz}
			\draw (0,2) to[american voltage source] (0,0);
			\draw (2,2) to[R, l=$2\Omega$] (2,0);
			\draw (4,2) to[L,l=$4j$] (4,0);
			\draw (0,2) to[short] (4,2);
			\draw (0,0) to[short] (4,0);
		\end{circuitikz}
	\end{center}
\end{minipage}
\end{example}

\begin{solu}
	\ \\
	\begin{gather*}
		Z_{eq} = 2 \textbardbl j4 = \frac{2(j4)}{2+j4} = \frac{8j}{4.47\measuredangle \tan^{-1}(2)}\\
		= \frac{8 \measuredangle 90^\circ}{4.47\measuredangle \tan^{-1}(2)} =
		\frac{8}{4.47} \measuredangle (90 - 63.43) = \\
		= 1.79 \measuredangle 26.57^\circ
	\end{gather*}
	\begin{gather*}
		I = \frac{V_s}{Z_{eq}} = \frac{17.9 \measuredangle 30^\circ}{1.79 \measuredangle 26.57^\circ}
		=10 \measuredangle 3.43^\circ
	\end{gather*}
	\begin{gather*}
	\Rightarrow i(t) = 10\cos (\omega t + 3.43^\circ)
\end{gather*}
\end{solu}



\begin{remark}
	\begin{itemize}
		\item
		امپدانس سلف همیشه مثبت و ادمیتانس سلف منفی است.
		\item
		امپدانس خازن همیشه منفی و ادمیتانس خازن مثبت است.
	\end{itemize}
\end{remark}

\begin{example}
	در مدار شکل زیر امپدانس معادل را در فرکانس 
	$\omega = 10^3 (\frac{rad}{s})$
	بدست آوردید.
	
		\begin{center}
		\begin{circuitikz}
			\draw (2,2) to[L,l=$5mH$] (2,0);
			\draw (0,2) to[capacitor,l=$100\mu F$,*-] (2,2);
			\draw (0,0) to[short,*-] (2,0);
			\draw[->] (-0.25,-0.5) -- (-0.25,0.5) -- (0.5,0.5) node[xshift=-1.25cm]{$Z_{eq}$};
		\end{circuitikz}
	\end{center}
\end{example}

\begin{solu}
	\begin{gather*}
		Z_{eq} = Z_c + Z_L\\
		Z_c = \frac{-j}{\omega c} = \frac{-j}{10^3 \times 100 \times 10^{-6}}= -j10 \Omega \\
		Z_L = j\omega L = j(10^3)(5\times 10^{-3}) = j5
	\end{gather*}
	\begin{gather*}
	Z_{eq} = Z_c + Z_L = -j10 + j5 = -j5 = -5 \measuredangle 90^\circ
	\end{gather*}
\end{solu}




\begin{remark}
	واحد امپدانس \textbf{اهم} و واحد ادمیتانس \textbf{مهو} است.
\end{remark}


\begin{example}
	امپدانس معادل را به دست آورید.
	
		\begin{center}
		\begin{circuitikz}
			\draw (2,2) to[L, l=$5mH$] (2,0);
			\draw (4,2) to[capacitor,l=$100\mu F$] (4,0);
			\draw (0,2) to[short,*-] (4,2);
			\draw (0,0) to[short,*-] (4,0);
			\draw[->] (-0.25,-0.5) -- (-0.25,0.5) -- (0.5,0.5) node[xshift=-1.25cm]{$Z_{eq}$};
		\end{circuitikz}
	\end{center}
\end{example}

\begin{solu}
	\begin{gather*}
		Z_c = \frac{-j}{\omega c} = \frac{-j}{10^3 \times 100 \times 10^{-6}}= -j10 \Omega \\
		Z_L = j\omega L = j(10^3)(5\times 10^{-3}) = j5
	\end{gather*}
	\begin{gather*}
		Z_{eq} = \frac{Z_c.Z_L}{Z_c + Z_L} \\
		Z_{eq} = \frac{-j10.j5}{-j10 + j5} = \frac{50}{-j5} = j10 = 10 \measuredangle 90^\circ
	\end{gather*}
\end{solu}






















